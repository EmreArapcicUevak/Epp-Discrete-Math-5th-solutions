\documentclass[14pt]{extarticle}

\usepackage[table]{xcolor} % colored lines for tables
\usepackage[normalem]{ulem} % strike through text
\usepackage{amsmath,mathtools,amsfonts,amsthm,amssymb,hyperref,pifont}
\usepackage{parskip,geometry,latexsym,bookmark,mathtools,float,cancel}
\usepackage{minted,tcolorbox,bm}

\usepackage[mathscr]{euscript}
\let\euscr\mathscr
\let\mathscr\relax
\usepackage[scr]{rsfso}
\newcommand{\ps}{\mathscr{P}} % for the Power set P

\newtheorem{defn}{Definition}
\newtheorem{thm}{Theorem}
\newtheorem{claim}{Claim}
\newtheorem{lemma}{Lemma}

\newcommand{\dps}{\displaystyle}
\newcommand{\es}{\varnothing}
\newcommand{\fbl}{\underline{\hspace{1cm}}\,\,}
\newcommand{\R}{\mathbb{R}}
\newcommand{\Q}{\mathbb{Q}}
\newcommand{\Z}{\mathbb{Z}}
\newcommand{\from}{\leftarrow}
\newcommand{\true}{{\bf t}}
\newcommand{\false}{{\bf c}}
\newcommand{\bic}{\leftrightarrow}
\newcommand{\da}{\downarrow}
\newcommand{\fa}{\forall}
\newcommand{\te}{\exists}
\newcommand{\cy}{\color{cyan}}

\newcommand{\colsq}[1]{{\color{#1} $\blacksquare$}}

\newcommand{\base}[1]{{\cy #1}} % for log bases
\newcommand{\floor}[1]{{\left\lfloor#1\right\rfloor}}
\newcommand{\ceil}[1]{{\left\lceil#1\right\rceil}}
\newcommand\Ccancel[2][black]{\renewcommand\CancelColor{\color{#1}}\cancel{#2}}
\newcommand\Cbcancel[2][black]{\renewcommand\CancelColor{\color{#1}}\bcancel{#2}}

\renewcommand{\arraystretch}{1.2}

\hypersetup{colorlinks,allcolors=blue,linktoc=all}
\geometry{a4paper}
\geometry{margin=0.42in}

\title{Solutions to Chapter 7, Susanna Epp Discrete Math 5th Edition}

\author{https://github.com/spamegg1}

\begin{document}
\maketitle
\tableofcontents

\section{Exercise Set 7.1}

\subsection{Exercise 1}
Let \(X = \{1, 3, 5\}\) and \(Y = \{s, t, u, v\}\). Define \(f: X \to Y\) by the following arrow diagram.

\begin{figure}[ht!]
    \centering
    \includegraphics[scale=0.5]{../images/7.1.1.png}
\end{figure}

\subsubsection{(a)}
Write the domain of $f$ and the co-domain of $f$.

\begin{proof}
    domain of \(f = \{1, 3, 5\}\), co-domain of \(f = \{s, t, u, v\}\)
\end{proof}

\subsubsection{(b)}
Find $f(1), f(3)$, and $f(5)$.

\begin{proof}
    \(f(1) = v, f(3) = s, f(5) = v\)
\end{proof}

\subsubsection{(c)}
What is the range of $f$?

\begin{proof}
    range of \(f = \{s, v\}\)
\end{proof}

\subsubsection{(d)}
Is 3 an inverse image of $s$? Is 1 an inverse image of $u$?

\begin{proof}
    yes, no
\end{proof}

\subsubsection{(e)}
What is the inverse image of $s$? of $u$? of $v$?

\begin{proof}
    inverse image of \(s = \{3\}\), inverse image of \(u = \es\), inverse image of \(v = \{1, 5\}\)
\end{proof}

\subsubsection{(f)}
Represent $f$ as a set of ordered pairs.

\begin{proof}
    \(\{(1, v), (3, s), (5, v)\}\)
\end{proof}

\subsection{Exercise 2}
Let \(X = \{1, 3, 5\}\) and \(Y = \{a, b, c, d\}\). Define \(g: X \to Y\) by the following arrow diagram.

\begin{figure}[ht!]
    \centering
    \includegraphics[scale=0.6]{../images/7.1.2.png}
\end{figure}

\subsubsection{(a)}
Write the domain of $g$ and the co-domain of $g$.

\begin{proof}
    domain: \(\{1, 3, 5\}\) co-domain: \(\{a,b,c,d\}\)
\end{proof}

\subsubsection{(b)}
Find $g(1), g(3)$, and $g(5)$.

\begin{proof}
    \(g(1) = b, g(3) = b, g(5) = b\)
\end{proof}

\subsubsection{(c)}
What is the range of $g$?

\begin{proof}
    \(\{b\}\)
\end{proof}

\subsubsection{(d)}
Is 3 an inverse image of $a$? Is 1 an inverse image of $b$?

\begin{proof}
    no, yes
\end{proof}

\subsubsection{(e)}
What is the inverse image of $b$? of $c$?

\begin{proof}
    \(\{1,3,5\}\) and $\es$
\end{proof}

\subsubsection{(f)}
Represent $g$ as a set of ordered pairs.

\begin{proof}
    \(\{(1,b),(3,b),(5,b)\}\)
\end{proof}


\subsection{Exercise 3}
Indicate whether the statements in parts (a)–(d) are true or false for all functions. Justify your answers.

\subsubsection{(a)}
If two elements in the domain of a function are equal, then their images in the co-domain are equal.

\begin{proof}
    True. The definition of function says that for any input there is one and only one output, so if two inputs are
    equal, their outputs must also be equal.
\end{proof}

\subsubsection{(b)}
If two elements in the co-domain of a function are equal, then their preimages in the domain are also equal.

\begin{proof}
    Not necessarily true. A function can have the same output for more than one input.
\end{proof}

\subsubsection{(c)}
A function can have the same output for more than one input.

\begin{proof}
    True. The definition of function does not prohibit this occurrence.
\end{proof}

\subsubsection{(d)}
A function can have the same input for more than one output.

\begin{proof}
    False, this is ruled out by the definition of a function. Functions are single valued. Every input corresponds to
    only one output.
\end{proof}

\subsection{Exercise 4}
\subsubsection{(a)}
Find all functions from \(X = \{a, b\}\) to \(Y = \{u, v\}\).

\begin{proof}
    There are four functions from $X$ to $Y$ as shown {\it on the next page}.

    \begin{figure}[ht!]
        \centering
        \includegraphics[scale=0.5]{../images/7.1.4.a.png}
    \end{figure}

\end{proof}

\subsubsection{(b)}
Find all functions from \(X = \{a, b, c\}\) to \(Y = \{u\}\).

\begin{proof}
    There is only one function $f: X \to Y$ given by the set \(\{(a,u), (b,u), (c,u)\}\).
\end{proof}

\subsubsection{(c)}
Find all functions from \(X = \{a, b, c\}\) to \(Y = \{u, v\}\).

\begin{proof}
    There are 8 functions:

    \(\{(a,u), (b,u), (c,u)\}\)

    \(\{(a,u), (b,u), (c,v)\}\)

    \(\{(a,u), (b,v), (c,u)\}\)

    \(\{(a,u), (b,v), (c,v)\}\)

    \(\{(a,v), (b,u), (c,u)\}\)

    \(\{(a,v), (b,u), (c,v)\}\)

    \(\{(a,v), (b,v), (c,u)\}\)

    \(\{(a,v), (b,v), (c,v)\}\)
\end{proof}

\subsection{Exercise 5}
Let $I_{\Z}$ be the identity function defined on the set of all integers, and suppose that \(e, b_{i}^{jk}, K(t)\), and
\(u_{kj}\) all represent integers. Find the following:

\subsubsection{(a)}
\(I_{\Z}(e)\)

\begin{proof}
    $e$ (because \(I_{\Z}\) is the identity function).
\end{proof}

\subsubsection{(b)}
\(I_{\Z}(b_{i}^{jk})\)

\begin{proof}
    $b_{i}^{jk}$ (because \(I_{\Z}\) is the identity function).
\end{proof}

\subsubsection{(c)}
\(I_{\Z}(K(t))\)

\begin{proof}
    $K(t)$ (because \(I_{\Z}\) is the identity function).
\end{proof}

\subsubsection{(d)}
\(I_{\Z}(u_{kj})\)

\begin{proof}
    $u_{kj}$ (because \(I_{\Z}\) is the identity function).
\end{proof}

\subsection{Exercise 6}
Find functions defined on the set of nonnegative integers that can be used to define the sequences whose first six
terms are given below.

\subsubsection{(a)}
\(\dps 1, -\frac{1}{3}, \frac{1}{5}, -\frac{1}{7}, \frac{1}{9}, -\frac{1}{11}\)

\begin{proof}
    The sequence is given by the function \(f: \Z^{nonneg} \to \R\) defined by the rule \(\dps f(n)=\frac{(-1)^n}{2n+1}\)
    {\cy for each nonnegative integer $n$.}
\end{proof}

\subsubsection{(b)}
\(0, -2, 4, -6, 8, -10\)

\begin{proof}
    The sequence is given by the function \(f: \Z^{nonneg} \to \Z\) defined by the rule \(\dps f(n) = (-1)^n \cdot (2n)\)
    {\cy for each nonnegative integer $n$.}
\end{proof}

\subsection{Exercise 7}
Let \(A = \{1, 2, 3, 4, 5\}\), and define a function \(F: \ps(A) \to \Z\) as follows: For each set $X$ in $\ps(A)$,
\[
    F(x) =
    \left\{
    \begin{tabular}{ll}
        \(0\) & if $X$ has an even number of elements \\
        \(1\) & if $X$ has an odd number of elements
    \end{tabular}
    \right.
\]
Find the following:

\subsubsection{(a)}
\(F(\{1, 3, 4\})\)

\begin{proof}
    \(F(\{1, 3, 4\}) = 1\) {\it [because \(\{1, 3, 4\}\) has an odd number of elements]}

\end{proof}

\subsubsection{(b)}
\(F(\es)\)

\begin{proof}
    \(F(\{\es\}) = 0\) {\it [because \(\{\es\}\) has an even number of elements]}
\end{proof}

\subsubsection{(c)}
\(F(\{2, 3\})\)

\begin{proof}
    \(F(\{2, 3\}) = 0\) {\it [because \(\{2, 3\}\) has an even number of elements]}
\end{proof}

\subsubsection{(d)}
\(F(\{2, 3, 4, 5\})\)

\begin{proof}
    \(F(\{2, 3, 4, 5\}) = 0\) {\it [because \(\{2, 3, 4, 5\}\) has an even number of elements]}
\end{proof}

\subsection{Exercise 8}
Let \(J_5 = \{0, 1, 2, 3, 4\}\), and define a function \(F: J_5 \to J_5\) as follows: For each \(x \in J_5, F(x) = (x^3
+ 2x + 4) \mod 5\). Find the following:

\subsubsection{(a)}
$F(0)$

\begin{proof}
    \(F(0) = (0^3 + 2 \cdot 0 + 4) \mod 5 = 4 \mod 5 = 4\)

\end{proof}

\subsubsection{(b)}
$F(1)$

\begin{proof}
    \(F(1) = (1^3 + 2 \cdot 1 + 4) \mod 5 = 7 \mod 5 = 2\)
\end{proof}

\subsubsection{(c)}
$F(2)$

\begin{proof}
    \(F(2) = (2^3 + 2 \cdot 2 + 4) \mod 5 = 16 \mod 5 = 1\)
\end{proof}

\subsubsection{(d)}
$F(3)$

\begin{proof}
    \(F(3) = (3^3 + 2 \cdot 3 + 4) \mod 5 = 37 \mod 5 = 2\)
\end{proof}

\subsubsection{(e)}
$F(4)$

\begin{proof}
    \(F(4) = (4^3 + 2 \cdot 4 + 4) \mod 5 = 76 \mod 5 = 1\)
\end{proof}

\subsection{Exercise 9}
Define a function \(S: \Z^+ \to \Z^+\) as follows: For each positive integer \(n, S(n) =\) the sum of the positive
divisors of $n$. Find the following:

\subsubsection{(a)}
$S(1)$

\begin{proof}
    $S(1) = 1$

\end{proof}

\subsubsection{(b)}
$S(15)$

\begin{proof}
    \(S(15) = 1 + 3 + 5 + 15 = 24\)
\end{proof}

\subsubsection{(c)}
$S(17)$

\begin{proof}
    \(S(15) = 1 + 17 = 18\)
\end{proof}

\subsubsection{(d)}
$S(5)$

\begin{proof}
    \(S(5) = 1 + 5 = 6\)
\end{proof}

\subsubsection{(e)}
$S(18)$

\begin{proof}
    \(S(18) = 1 + 2 + 3 + 6 + 9 + 18 = 39\)
\end{proof}

\subsubsection{(f)}
$S(21)$

\begin{proof}
    \(S(21) = 1 + 3 + 7 + 21 = 32\)
\end{proof}

\subsection{Exercise 10}
Let $D$ be the set of all finite subsets of positive integers. Define a function \(T: \Z^+ \to D\) as follows:
For each positive integer \(n, T(n) =\) the set of positive divisors of $n$. Find the following:

\subsubsection{(a)}
$T(1)$

\begin{proof}
    \(T(1) = \{1\}\)

\end{proof}

\subsubsection{(b)}
$T(15)$

\begin{proof}
    \(T(15) = \{1, 3, 5, 15\}\)
\end{proof}

\subsubsection{(c)}
$T(17)$

\begin{proof}
    \(T(17) = \{1, 17\}\)
\end{proof}

\subsubsection{(d)}
$T(5)$

\begin{proof}
    \(T(1) = \{1\}\)
\end{proof}

\subsubsection{(e)}
$T(18)$

\begin{proof}
    \(T(18) = \{1, 2, 3, 6, 9, 18\}\)
\end{proof}

\subsubsection{(f)}
$T(21)$

\begin{proof}
    \(T(21) = \{1, 3, 7, 21\}\)
\end{proof}

\subsection{Exercise 11}
Define \(F: \Z \times \Z \to \Z \times \Z\) as follows: For every ordered pair \((a,b)\) of integers, \(F(a,b) = (2a+1,
3b-2)\). Find the following:

\subsubsection{(a)}
$F(4,4)$

\begin{proof}
    \(F(4, 4) = (2 \cdot 4 + 1, 3 \cdot 4 - 2) = (9, 10)\)
\end{proof}

\subsubsection{(b)}
$F(2,1)$

\begin{proof}
    \(F(2, 1) = (2 \cdot 2 + 1, 3 \cdot 1 - 2) = (5, 1)\)
\end{proof}

\subsubsection{(c)}
$F(3,2)$

\begin{proof}
    \(F(3, 3) = (2 \cdot 3 + 1, 3 \cdot 3 - 2) = (7, 7)\)
\end{proof}

\subsubsection{(d)}
$F(1,5)$

\begin{proof}
    \(F(1, 5) = (2 \cdot 1 + 1, 3 \cdot 5 - 2) = (3, 13)\)
\end{proof}

\subsection{Exercise 12}
Let \(J_5 = \{0, 1, 2, 3, 4\}\), and define \(G: J_5 \times J_5 \to J_5 \times J_5\) as follows: For each \((a,b) \in
J_5 \times J_5, G(a,b) = ((2a+1) \mod 5, (3b-2) \mod 5)\). Find the following:

\subsubsection{(a)}
$G(4,4)$

\begin{proof}
    \(G(4, 4) = ((2 \cdot 4 + 1) \mod 5, (3 \cdot 4 - 2) \mod 5) = (9 \mod 5, 10 \mod 5) = (4, 0)\)
\end{proof}

\subsubsection{(b)}
$G(2,1)$

\begin{proof}
    \(G(2, 1) = ((2 \cdot 2 + 1) \mod 5, (3 \cdot 1 - 2) \mod 5) = (5 \mod 5, 1 \mod 5) = (0, 1)\)
\end{proof}

\subsubsection{(c)}
$G(3,2)$

\begin{proof}
    \(G(3, 2) = ((2 \cdot 3 + 1) \mod 5, (3 \cdot 2 - 2) \mod 5) = (7 \mod 5, 4 \mod 5) = (2, 4)\)
\end{proof}

\subsubsection{(d)}
$G(1,5)$

\begin{proof}
    \(G(1, 5) = ((2 \cdot 1 + 1) \mod 5, (3 \cdot 5 - 2) \mod 5) = (3 \mod 5, 13 \mod 5) = (3, 3)\)
\end{proof}

\subsection{Exercise 13}
Let \(J_5 = \{0, 1, 2, 3, 4\}\), and define functions \(f: J_5 \to J_5\) and \(g: J_5 \to J_5\) as follows: For each
\(x \in J_5, f(x) = (x + 4)^2 \mod 5\) and \(g(x) = (x^2 + 3x + 1) \mod 5\). Is $f = g$? Explain.

\begin{proof}
    \begin{center}
        \arrayrulecolor{cyan}
        \begin{tabular}{|c|c|c|}
            \hline
            $x$ & $f(x)$             & $g(x)$                               \\
            \hline
            0   & \(4^2 \mod 5 = 1\) & \((0^2 + 3 \cdot 0 + 1) \mod 5 = 1\) \\
            \hline
            1   & \(5^2 \mod 5 = 0\) & \((1^2 + 3 \cdot 1 + 1) \mod 5 = 0\) \\
            \hline
            2   & \(6^2 \mod 5 = 1\) & \((2^2 + 3 \cdot 2 + 1) \mod 5 = 1\) \\
            \hline
            3   & \(7^2 \mod 5 = 4\) & \((3^2 + 3 \cdot 3 + 1) \mod 5 = 4\) \\
            \hline
            4   & \(8^2 \mod 5 = 4\) & \((4^2 + 3 \cdot 4 + 1) \mod 5 = 4\) \\
            \hline
        \end{tabular}
        \arrayrulecolor{black}
    \end{center}

    The table shows that \(f(x) = g(x)\) for every $x$ in \(J_5\). Thus, by definition of equality of functions, $f = g$.
\end{proof}

\subsection{Exercise 14}
Define functions $H$ and $K$ from $\R$ to $\R$ by the following formulas: For every \(x \in \R, H(x) = \floor{x}
+ 1\) and \(K(x) = \ceil{x}\). Does $H = K$? Explain.

\begin{proof}
    No, because \(H(2) = \floor{2} + 1 = 3 \neq 2 = \ceil{2} = K(2)\).
\end{proof}

\subsection{Exercise 15}
Let $F$ and $G$ be functions from the set of all real numbers to itself. Define new functions
\(F \cdot G: \R \to \R\) and \(G \cdot F: \R \to \R\) as follows: For every \(x \in \R\),
\((F \cdot G)(x) = F(x) \cdot G(x)\), \((G \cdot F)(x) = G(x) \cdot F(x)\). Does $F \cdot G = G \cdot F$? Explain.

\begin{proof}
    \begin{tabular}{rcll}
        \((F \cdot G)(x)\) & = & \(F(x) \cdot G(x)\) & {\cy by definition of $F \cdot G$}        \\
                           & = & \(G(x) \cdot F(x)\) & {\cy by commutative law for real numbers} \\
                           & = & \((G \cdot F)(x)\)  & {\cy by definition of $G \cdot F$}
    \end{tabular}

    for every real number $x$. Therefore $F\cdot G$ and $G \cdot F$ are equal.
\end{proof}

\subsection{Exercise 16}
Let $F$ and $G$ be functions from the set of all real numbers to itself. Define new functions
\(F - G: \R \to \R\) and \(G - F: \R \to \R\) as follows: For every \(x \in \R\), \((F - G)(x) = F(x) - G(x)\),
\((G - F)(x) = G(x) - F(x)\). Does $F - G = G - F$? Explain.

\begin{proof}
    \underline{Counterexample:} Let \(F(x) = 2x, G(x) = 3x\).
    Then
    \[
        (F-G)(1) = F(1) - G(1) = 2 - 3 = -1 \neq 1 = 3 - 2 = G(1) - F(1) = (G-F)(1).
    \]
    Therefore $F - G$ does not equal $G - F$.
\end{proof}

\subsection{Exercise 17}
Use the definition of logarithm to fill in the blanks below.

\subsubsection{(a)}
\(\log_{2} 8 = 3\) because \fbl.

\begin{proof}
    \(2^3 = 8\)
\end{proof}

\subsubsection{(b)}
\(\log_{5}(\frac{1}{25}) = -2\) because \fbl.

\begin{proof}
    \(\dps 5^{-2} = \frac{1}{25}\)
\end{proof}

\subsubsection{(c)}
\(\log_{4} 4 = 1\) because \fbl.

\begin{proof}
    \(4^1 = 4\)
\end{proof}

\subsubsection{(d)}
\(\log_{3}(3^n) = n\) because \fbl.

\begin{proof}
    \(3^n = 3^n\)
\end{proof}

\subsubsection{(e)}
\(\log_{4} 1 = 0\) because \fbl.

\begin{proof}
    \(4^0 = 1\)
\end{proof}

\subsection{Exercise 18}
Find exact values for each of the following quantities without using a calculator.

\subsubsection{(a)}
\(\log_{3} 81\)

\begin{proof}
    4
\end{proof}

\subsubsection{(b)}
\(\log_{2} 1024\)

\begin{proof}
    10
\end{proof}

\subsubsection{(c)}
\(\log_{3} \frac{1}{27}\)

\begin{proof}
    $-3$
\end{proof}

\subsubsection{(d)}
\(\log_{2} 1\)

\begin{proof}
    0
\end{proof}

\subsubsection{(e)}
\(\log_{10}(\frac{1}{10})\)

\begin{proof}
    $-1$
\end{proof}

\subsubsection{(f)}
\(\log_{3} 3\)

\begin{proof}
    1
\end{proof}

\subsubsection{(g)}
\(\log_{2} 2^k\)

\begin{proof}
    $k$
\end{proof}

\subsection{Exercise 19}
Use the definition of logarithm to prove that for any positive real number $b$ with \(b \neq 1, \log_b b = 1\).

\begin{proof}
    Let $b$ be any positive real number with \(b \neq 1\). Since \(b^1 = b\), then \(\log_b b = 1\) by definition of logarithm.
\end{proof}

\subsection{Exercise 20}
Use the definition of logarithm to prove that for any positive real number $b$ with \(b \neq 1, \log_b 1 = 0\).

\begin{proof}
    Let $b$ be any positive real number with \(b \neq 1\). Since \(b^0 = 1\), then \(\log_b 1 = 0\) by definition of logarithm.
\end{proof}

\subsection{Exercise 21}
If $b$ is any positive real number with \(b \neq 1\) and $x$ is any real number, \(b^{-x}\) is defined as follows:
\(\dps b^{-x} = \frac{1}{b^x}\). Use this definition and the definition of logarithm to prove that \(\dps \log_b
\left(\frac{1}{u}\right) = -\log_b(u)\) for all positive real numbers $u$ and $b$, with \(b \neq 1\).

\begin{proof}
    Suppose $b$ and $u$ are any positive real numbers with \(b \neq 1\). {\it [We must show that \(\log_b (\frac{1}{u}) =
                -\log_b(u)\).]} Let \(v = \log_b (\frac{1}{u})\). By definition of logarithm, \(b^v = \frac{1}{u}\). Multiplying
    both sides by $u$ and dividing by $b^v$ gives \(u=b^{-v}\), and thus, by definition of logarithm, \(-v = \log_b(u)\).
    When both sides of this equation are multiplied by $-1$, the result is \(v = -\log_b(u)\). Therefore, \(\log_b
    (\frac{1}{u}) = - \log_b(u)\) because both expressions equal $v$. {\it [This is what was to be shown.]}
\end{proof}

\subsection{Exercise 22}
Use the unique factorization for the integers theorem (Section 4.4) and the definition of logarithm to prove that
$\log_3(7)$ is irrational.

\begin{proof}
    1. Argue by contradiction and assume \(r = \log_3(7)\) is rational.

    2. By 1 and definition of rational, \(r = a / b\) for some integers $a,b$ where $b \neq 0$.

    3. We may assume \(b > 0\). (If \(b < 0\) then \(a/b = (-a)/(-b)\) therefore we can replace $a/b$ with $-a/(-b)$
    where \(-b > 0\).)

    4. By 1, 2 and definition of logarithm, \(7 = 3^{a/b}\).

    5. By 4, taking the $b$th powers of both sides, we get \(7^b = 3^a\).

    6. Since \(b\) is a positive integer, \(7^b\) is a positive integer. Therefore \(3^a\) is the same positive integer.

    7. By 6, we have two different prime factorizations of the same positive integer. By the uniqueness part of the prime
    factorization theorem, this is only possible if the positive integer is equal to 1.

    8. By 7, \(7^b = 3^a = 1\) so \(a = b = 0\). This is a contradiction since \(b > 0\).

    9. So our supposition in 1 is false by 8, thus \(\log_3(7)\) is irrational.
\end{proof}

\subsection{Exercise 23}
If $b$ and $y$ are positive real numbers such that \(\log_b y = 3\), what is \(\log_{1/b}(y)\)? Explain.

\begin{proof}
    By definition of logarithm with base $b$, \(b^3 = y\). So
    \[
        y = b^3 = \frac{1}{\frac{1}{b^3}} = \frac{1}{\left(\frac{1}{b}\right)^3} = \left(\frac{1}{b}\right)^{-3}
    \]
    So by definition of logarithm with base $1/b$, \(\log_{1/b}(y) = -3\).
\end{proof}

\subsection{Exercise 24}
If $b$ and $y$ are positive real numbers such that \(\log_b y = 2\), what is \(\log_{b^2}(y)\)? Explain.

\begin{proof}
    By definition of logarithm with base $b$, \(b^2 = y\). So
    by definition of logarithm with base $b^2$,
    \(\log_{b^2}(y) = 1\) because \((b^2)^1 = y\).
\end{proof}

\subsection{Exercise 25}
Let \(A = \{2, 3, 5\}\) and \(B = \{x, y\}\). Let $p_1$ and $p_2$ be the projections of \(A \times B\) onto the first
and second coordinates. That is, for each pair \((a, b) \in A \times B, p_1(a, b) = a\) and \(p_2(a, b) = b\).

\subsubsection{(a)}
Find \(p_1(2, y)\) and \(p_1(5, x)\). What is the range of $p_1$?

\begin{proof}
    \(p_1(2, y) = 2, p_1(5, x) = 5\), range of \(p_1 = \{2, 3, 5\}\)
\end{proof}

\subsubsection{(b)}
Find \(p_2(2, y)\) and \(p_2(5, x)\). What is the range of $p_2$?

\begin{proof}
    \(p_2(2, y) = y, p_2(5, x) = x\), range of \(p_2 = \{x, y\}\)
\end{proof}

\subsection{Exercise 26}
Observe that mod and div can be defined as functions from \(\Z^{nonneg} \times \Z^+\) to $\Z$. For each ordered pair
\((n, d)\) consisting of a nonnegative integer $n$ and a positive integer $d$, let

\(mod(n, d) = n \mod d\) (the nonnegative remainder obtained when $n$ is divided by $d$).

\(div(n, d) = n \text{ div } d\) (the integer quotient obtained when $n$ is divided by $d$).

Find each of the following:

\subsubsection{(a)}
\(mod(67, 10)\) and \(div(67, 10)\)

\begin{proof}
    \(mod(67, 10) = , div(67, 10) = \)
\end{proof}

\subsubsection{(b)}
$mod(59, 8)$ and $div(59, 8)$

\begin{proof}
    \(mod(67, 10) = 7, div(67, 10) = 6\)
\end{proof}

\subsubsection{(c)}
$mod(30, 5)$ and $div(30, 5)$

\begin{proof}
    \(mod(30, 5) = 0, div(30, 5) = 6\)
\end{proof}

\subsection{Exercise 27}
Let $S$ be the set of all strings of $a$’s and $b$’s.

\subsubsection{(a)}
Define \(f: S \to \Z\) as follows: For each string $s$ in $S$
\[
    f(s) =
    \left\{
    \begin{tabular}{lr}
        the number of $b$'s to the left of the left most $a$ in $s$ & if $s$ contains some $a$'s \\
        0                                                           & if $s$ contains no $a$'s
    \end{tabular}
    \right.
\]
Find \(f(aba), f(bbab)\), and \(f(b)\). What is the range of $f$?

\begin{proof}
    \(f(aba) = 0\) {\it [because there are no $b$'s to the left
                of the leftmost $a$ in $aba$]}

    \(f(bbab) = 2\) {\it [because there are two $b$'s to the left of the leftmost $a$ in $bbab$]}

    \(f(b) = 0\) {\it [because the string $b$ contains no $a$'s]}

    range of \(f = \Z^{nonneg}\)
\end{proof}

\subsubsection{(b)}
Define \(g: S \to S\) as follows: For each string $s$ in $S$, \(g(s) =\) the string obtained by writing the
characters of $s$ in reverse order. Find \(g(aba), g(bbab)\), and \(g(b)\). What is the range of $g$?

\begin{proof}
    \(g(aba) = aba, g(bbab) = babb, g(b) = b\), range of \(g = S\)
\end{proof}

\subsection{Exercise 28}
Consider the coding and decoding functions $E$ and $D$ defined in Example 7.1.9.

\subsubsection{(a)}
Find \(E(0110)\) and \(D(111111000111)\).

\begin{proof}
    \(E(0110) = 000111111000\) and \(D(111111000111) = 1101\)
\end{proof}

\subsubsection{(b)}
Find \(E(1010)\) and \(D(000000111111)\).

\begin{proof}
    \(E(1010) = 111000111000\) and \(D(000000111111) = 0011\)
\end{proof}

\subsection{Exercise 29}
Consider the Hamming distance function defined in Example 7.1.10.

\subsubsection{(a)}
Find \(H(10101, 00011)\).

\begin{proof}
    \(H(10101, 00011)\) = 3
\end{proof}

\subsubsection{(b)}
Find \(H(00110, 10111)\).

\begin{proof}
    \(H(00110, 10111)\) = 2
\end{proof}

\subsection{Exercise 30}
Draw arrow diagrams for the Boolean functions defined by the following input/output tables.

\subsubsection{(a)}
\begin{center}
    \arrayrulecolor{cyan}
    \begin{tabular}{|cc|c|}
        \hline
        \multicolumn{2}{|c|}{\cy Input} & {\cy Output}       \\
        \hline
        $P$                             & $Q$          & $R$ \\
        \hline
        1                               & 1            & 0   \\
        \hline
        1                               & 0            & 1   \\
        \hline
        0                               & 1            & 0   \\
        \hline
        0                               & 0            & 1   \\
        \hline
    \end{tabular}
    \arrayrulecolor{black}
\end{center}

\begin{proof}
    \begin{figure}[ht!]
        \centering
        \includegraphics[scale=0.6]{../images/7.1.30.a.png}
    \end{figure}
\end{proof}

\subsubsection{(b)}
\begin{center}
    \arrayrulecolor{cyan}
    \begin{tabular}{|ccc|c|}
        \hline
        \multicolumn{3}{|c|}{\cy Input} & {\cy Output}             \\
        \hline
        $P$                             & $Q$          & $R$ & $S$ \\
        \hline
        1                               & 1            & 1   & 1   \\
        \hline
        1                               & 1            & 0   & 0   \\
        \hline
        1                               & 0            & 1   & 1   \\
        \hline
        1                               & 0            & 0   & 1   \\
        \hline
        0                               & 1            & 1   & 0   \\
        \hline
        0                               & 1            & 0   & 0   \\
        \hline
        0                               & 0            & 1   & 0   \\
        \hline
        0                               & 0            & 0   & 1   \\
        \hline
    \end{tabular}
    \arrayrulecolor{black}
\end{center}

\begin{proof}
    \begin{figure}[ht!]
        \centering
        \includegraphics[scale=0.4]{../images/7.1.30.b.png}
    \end{figure}
\end{proof}

\subsection{Exercise 31}
Fill in the following table to show the values of all possible two-place Boolean functions.

\begin{proof}
    \begin{center}
        \arrayrulecolor{cyan}
        \begin{tabular}{|c|c|c|c|c|c|c|c|c|c|c|c|c|c|c|c|c|}
            \hline
            {\bf Input}     & \(\bm{f_{1}}\)  & \(\bm{f_{2}}\)  &
            \(\bm{f_{3}}\)  & \(\bm{f_{4}}\)  & \(\bm{f_{5}}\)  &
            \(\bm{f_{6}}\)  & \(\bm{f_{7}}\)  & \(\bm{f_{8}}\)  &
            \(\bm{f_{9}}\)  & \(\bm{f_{10}}\) & \(\bm{f_{11}}\) &
            \(\bm{f_{12}}\) & \(\bm{f_{13}}\) & \(\bm{f_{14}}\) &
            \(\bm{f_{15}}\) & \(\bm{f_{16}}\)                                                                           \\
            \hline
            1 1             & 0               & 0               & 0 & 0 & 0 & 0 & 0 & 0 & 1 & 1 & 1 & 1 & 1 & 1 & 1 & 1 \\
            \hline
            1 0             & 0               & 0               & 0 & 0 & 1 & 1 & 1 & 1 & 0 & 0 & 0 & 0 & 1 & 1 & 1 & 1 \\
            \hline
            0 1             & 0               & 0               & 1 & 1 & 0 & 0 & 1 & 1 & 0 & 0 & 1 & 1 & 0 & 0 & 1 & 1 \\
            \hline
            0 0             & 0               & 1               & 0 & 1 & 0 & 1 & 0 & 1 & 0 & 1 & 0 & 1 & 0 & 1 & 0 & 1 \\
            \hline
        \end{tabular}
        \arrayrulecolor{black} % change it back!
    \end{center}
\end{proof}

\subsection{Exercise 32}
Consider the three-place Boolean function $f$ defined by the following rule: For each triple \((x_1, x_2, x_3)\) of
0’s and 1’s,
\[
    f(x_1, x_2, x_3) = (4x_1 + 3x_2 + 2x_3) \mod 2.
\]
\subsubsection{(a)}
Find $f(1, 1, 1)$ and $f(0, 0, 1)$.

\begin{proof}
    \(f(1, 1, 1) = (4 \cdot 1 + 3 \cdot 1 + 2 \cdot 1) \mod 2 = 9 \mod 2 = 1\)

    \(f(0, 0, 1) = (4 \cdot 0 + 3 \cdot 0 + 2 \cdot 1) \mod 2 = 2 \mod 2 = 0\)
\end{proof}

\subsubsection{(b)}
Describe $f$ using an input/output table.

\begin{proof}
    \begin{center}
        \arrayrulecolor{cyan}
        \begin{tabular}{|ccc|c|}
            \hline
            \multicolumn{3}{|c|}{\cy Input} & {\cy Output}                              \\
            \hline
            $x_1$                           & $x_2$        & $x_3$ & $f(x_1, x_2, x_3)$ \\
            \hline
            1                               & 1            & 1     & 1                  \\
            \hline
            1                               & 1            & 0     & 1                  \\
            \hline
            1                               & 0            & 1     & 0                  \\
            \hline
            1                               & 0            & 0     & 0                  \\
            \hline
            0                               & 1            & 1     & 1                  \\
            \hline
            0                               & 1            & 0     & 1                  \\
            \hline
            0                               & 0            & 1     & 0                  \\
            \hline
            0                               & 0            & 0     & 0                  \\
            \hline
        \end{tabular}
        \arrayrulecolor{black}
    \end{center}
\end{proof}

\subsection{Exercise 33}
Student $A$ tries to define a function \(g: \Q \to \Z\) by
the rule
\[
    g \left(\frac{m}{n}\right) = m-n \text{ for all integers } m \text{ and } n \text{ with } n \neq 0.
\]
Student $B$ claims that $g$ is not well defined. Justify student $B$'s claim.

\begin{proof}
    If $g$ were well defined, then \(g(1/2) = g(2/4)\) because \(1/2 = 2/4\). However, \(g(1/2) = 1 - 2 = -1\) and
    \(g(2/4) = 2 - 4 = -2\). Since \(-1 \neq -2, g(1/2) \neq g(2/4)\). Thus $g$ is not well defined.
\end{proof}

\subsection{Exercise 34}
Student $C$ tries to define a function \(h: \Q \to \Q\) by
the rule
\[
    h \left(\frac{m}{n}\right) = \frac{m^2}{n} \text{ for all integers } m \text{ and } n \text{ with } n \neq 0.
\]
Student $D$ claims that $h$ is not well defined. Justify student $D$'s claim.

\begin{proof}
    \[
        h(2) = h\left(\frac{4}{2}\right) = \frac{4^2}{2} = 8 \neq 4 = \frac{2^2}{1} = h\left(\frac{2}{1}\right) = h(2).
    \]
\end{proof}

\subsection{Exercise 35}
Let \(U = \{1, 2, 3, 4\}\). Student $A$ tries to define a function \(R: U \to \Z\) as follows: For each \(x \in U\),
\(R(x)\) is the integer $y$ so that \((xy) \mod 5 = 1\). Student $B$ claims that $R$ is not well defined. Who is
right: student $A$ or student $B$? Justify your answer.

\begin{proof}
    Student $B$ is correct. If $R$ were well defined, then $R(3)$ would have a uniquely determined value. However, on
    the one hand, \(R(3) = 2\) because \((3 \cdot 2) \mod 5 = 1\), and, on the other hand, \(R(3) = 7\) because
    \((3 \cdot 7) \mod 5 = 1\). Hence \(R(3)\) does not have a uniquely determined value, and so $R$ is not well defined.
\end{proof}

\subsection{Exercise 36}
Let \(V = \{1, 2, 3\}\). Student $C$ tries to define a function \(S: V \to V\) as follows: For each \(x \in V\),
\(S(x)\) is the integer $y$ in $V$ so that \((xy) \mod 4 = 1\). Student $D$ claims that $S$ is not well defined. Who
is right: student $C$ or student $D$? Justify your answer.

\begin{proof}
    Student $D$ is right, because $S(2)$ is not defined. \(2 \cdot 1 \mod 4 = 2, 2 \cdot 2 \mod 4 = 0\), and
    \(2 \cdot 3 \mod 4 = 2\). So when $x = 2$ there is no $y$ in $V$ such that \(xy \mod 4 = 1\).
\end{proof}

\subsection{Exercise 37}
On certain computers the integer data type goes from -2,147,483,648 to 2,147,483,647. Let $S$ be the set of
all integers from -2,147,483,648 through 2,147,483,647. Try to define a function \(f: S \to S\) by the rule
\(f(n) = n^2\) for each $n$ in $S$. Is $f$ well defined? Explain.

\begin{proof}
    No, \(2,147,483,647 = 2^{31} - 1\) so for values of $n$ greater than, say, $2^{16}$, \(f(n) = n^2\) will be greater
    than $2^{32}$ which falls outside of $S$.

    Computers handle this by using 2's complement and looping the overshoot around back to \(-2^{31}\) and onward toward
    the positive values again. Here is an example from Scala:

    \begin{minted}{scala}
$ scala
// Welcome to Scala 3.3.0 (17.0.7, Java OpenJDK 64-Bit Server VM).
// Type in expressions for evaluation. Or try :help.
scala> def f(n: Int): Int = n*n
def f(n: Int): Int
scala> f(2147483647)
val res0: Int = 1
scala>
\end{minted}
\end{proof}

\subsection{Exercise 38}
Let \(X = \{a,b,c\}\) and \(Y = \{r,s,t,u,v,w\}\). Define \(f: X \to Y\) as follows: \(f(a) = b, f(b) = v, f(c) = t\).

\subsubsection{(a)}
Draw an arrow diagram for $f$.

\begin{proof}
    \begin{figure}[ht!]
        \centering
        \includegraphics[scale=0.5]{../images/7.1.38.a.png}
    \end{figure}
\end{proof}

\subsubsection{(b)}
Let \(A = \{a, b\}, C = \{t\}, D = \{u, v\}, E = \{r, s\}\).

Find \(f(A), f(X), f^{-1}(C), f^{-1}(D), f^{-1}(E), f^{-1}(Y)\).

\begin{proof}
    \(f(A) = \{v\}, f(X) = \{t, v\}, f^{-1}(C) = \{c\}, f^{-1}(D) = \{a, b\}, f^{-1}(E) = \es\),

    \(f^{-1}(Y) = \{a,b,c\}\)
\end{proof}

\subsection{Exercise 39}
Let \(X = \{1, 2, 3, 4\}\) and \(Y = \{a, b, c, d, e\}\). Define \(g: X \to Y\) as follows:
\(g(1) = a, g(2) = a, g(3) = a, g(4) = d\).

\subsubsection{(a)}
Draw an arrow diagram for $g$.

\begin{proof}
    \begin{figure}[ht!]
        \centering
        \includegraphics[scale=0.3]{../images/7.1.39.a.png}
    \end{figure}
\end{proof}

\subsubsection{(b)}
Let \(A = \{2, 3\}, C = \{a\}\), and \(D = \{b, c\}\). Find \(g(A), g(X), g^{-1}(C), g^{-1}(D)\), and \(g^{-1}(Y)\).

\begin{proof}
    \(g(A) = \{a\}, g(X) = \{a, d\}, g^{-1}(C) = \{1, 2, 3\}, g^{-1}(D) = \es, g^{-1}(Y) = \{1, 2, 3, 4\}\).
\end{proof}

\subsection{Exercise 40}
Let $X$ and $Y$ be sets, let $A$ and $B$ be any subsets of $X$, and let $F$ be a function from $X$ to $Y$. Fill in the
blanks in the following proof that \(F(A) \cup F (B) \subseteq F(A \cup B)\).

    {\bf Proof:} Let $y$ be any element in \(F(A) \cup F(B)\). {\it [We must show that $y$ is in \(F(A \cup B)\).]} By
definition of union, {\cy (i) \fbl.}

    {\bf Case 1, \(\bm{y \in F(A)}\):} In this case, by definition of $F(A)$, \(y = F(x)\) for {\cy (ii) \fbl}
\(x \in A\). Since \(A \subseteq A \cup B\), it follows from the definition of union that \(x \in\) {\cy (iii)
        \fbl.} Hence, \(y = F(x)\) for some \(x \in A \cup B\), and thus, by definition of \(F(A \cup B), y \in\) {\cy (iv) \fbl.}

    {\bf Case 2, \(\bm{y \in F(B)}\):} In this case, by definition of $F(B)$, {\cy (v) \fbl} for some $x \in B$.
Since \(B \subseteq A \cup B\) it follows from the definition of union that {\cy (vi) \fbl.}
Thus \(y \in F(A \cup B)\).

Therefore, regardless of whether \(y \in F(A)\) or \(y \in F(B)\), we have that \(y \in F(A \cup B)\) {\it [as was to be shown].}

\begin{proof}
    (i) \(y \in F(A)\) or \(y \in F(B)\) (ii) some (iii) \(A \cup B\) (iv) \(F(A \cup B)\) (v) \(y = F(x)\) (vi) \(x \in A \cup B\)
\end{proof}

{\bf \cy In $41-49$ let $X$ and $Y$ be sets, let $A$ and $B$ be any subsets of $X$, and let $C$ and $D$ be any
subsets of $Y$. Determine which of the properties are true for every function $F$ from $X$ to $Y$ and which are false
for at least one function $F$ from $X$ to $Y$. Justify your answers.}

\subsection{Exercise 41}
If \(A \subseteq B\) then \(F(A) \subseteq F(B)\).

\begin{proof}
    Let $F$ be a function from $X$ to $Y$, and suppose \(A \subseteq X, B \subseteq X\), and \(A \subseteq B\).
    Let \(y \in F(A)\). {\it [We must show that \(y \in F(B)\).]} By definition of image of a set, \(y = F(x)\) for
    some \(x \in A\). Thus since \(A \subseteq B, x \in B\), and so \(y = F(x)\) for some \(x \in B\).
    Hence \(y \in F(B)\) {\it [as was to be shown].}
\end{proof}

\subsection{Exercise 42}
\(F(A \cap B) \subseteq F(A) \cap F(B)\)

\begin{proof}
    1. Assume \(y \in F(A \cap B)\). {\it [We want to show \(y \in F(A) \cap F(B)\).]}

    2. By 1 and definition of \(F(A \cap B)\), \(y = F(x)\) for some \(x \in A \cap B\).

    3. By 2 and definition of intersection, \(x \in A\) and \(x \in B\).

    4. By 3 and definition of $F(A)$ and $F(B)$, \(y = F(x)\) is in $F(A)$ and in $F(B)$.

    5. By 4 and definition of intersection, \(y \in F(A) \cap F(B)\).

    6. By 1, 5 and definition of subset, \(F(A \cap B) \subseteq F(A) \cap F(B)\).
\end{proof}

\subsection{Exercise 43}
\(F(A) \cap F(B) \subseteq F(A \cap B)\)

\begin{proof}
    \begin{figure}[ht!]
        \centering
        \includegraphics[scale=0.5]{../images/7.1.43.png}
    \end{figure}
    \underline{Counterexample:} Let \(X = \{1, 2, 3\}\), let \(Y = \{a, b\}\), and define a function \(F: X \to Y\)
    by the arrow diagram shown above.

    Let \(A = \{1, 2\}\) and \(B = \{1, 3\}\). Then \(F(A) = \{a, b\} = F(B)\), and so \(F(A) \cap F(B) = \{a, b\}\). But
    \(F(A \cap B) = F(\{1\}) = \{a\} \neq \{a, b\}\). And so \(F(A) \cap F(B) \nsubseteq F(A \cap B)\). (This is just one
    of many possible counterexamples.)
\end{proof}

\subsection{Exercise 44}
For all subsets $A$ and $B$ of $X$, \(F(A - B) = F(A) - F(B)\).

\begin{proof}
    \underline{Counterexample:} Let \(X = \{1, 2\}, Y = \{a\}, A = \{1\}, B = \{2\}\), define \(F: X \to Y\) by
    \(F(1) = F(2) = a\). Then \(A - B = \{1\}, F(A) = \{a\}, F(B)=\{a\}\). So \(F(A-B) = \{a\} \neq \es = F(A) - F(B)\).
\end{proof}

\subsection{Exercise 45}
For all subsets $C$ and $D$ of $Y$, if \(C \subseteq D\) then \(F^{-1}(C) \subseteq F^{-1}(D)\).

\begin{proof}
    Let $F$ be a function from a set $X$ to a set $Y$, and suppose \(C \subseteq Y, D \subseteq Y\), and
    \(C \subseteq D\). {\it [We must show that \(F^{-1}(C) \subseteq F^{-1}(D)\).]} Suppose \(x \in F^{-1}(C)\). Then
    \(F(x) \in C\). Since \(C \subseteq D, F(x) \in D\) also. Hence, by definition of inverse image, \(x \in F^{-1}(D)\).
        {\it [So \(F^{-1}(C) \subseteq F^{-1}(D)\).]}
\end{proof}

\subsection{Exercise 46}
For all subsets $C$ and $D$ of $Y$, \(F^{-1}(C \cup D) = F^{-1}(C) \cup F^{-1}(D)\).

\begin{proof}
    1. Assume \(x \in F^{-1}(C \cup D)\) and let $y = F(x)$. {\it [Want to show \(x \in F^{-1}(C) \cup F^{-1}(D)\).]}

    2. By 1 and definition of \(F^{-1}(C \cup D)\), \(y \in C \cup D\).

    3. By 2 and definition of union, \(y \in C\) or \(y \in D\).

    4. {\bf Case 1 (\(\bm{y \in C}\)):} By definition of \(F^{-1}(C)\), \(x \in F^{-1}(C)\).

    By definition of union, \(x \in F^{-1}(C) \cup F^{-1}(D)\).

    5. {\bf Case 2 (\(\bm{y \in D}\)):} By definition of \(F^{-1}(D)\), \(x \in F^{-1}(D)\).

    By definition of union, \(x \in F^{-1}(C) \cup F^{-1}(D)\).

    6. By 4 and 5, \(x \in F^{-1}(C) \cup F^{-1}(D)\).

    7. By 1, 6 and definition of subset, \(F^{-1}(C \cup D) \subseteq F^{-1}(C) \cup F^{-1}(D)\).

    The proof of the reverse direction \(F^{-1}(C) \cup F^{-1}(D) \subseteq F^{-1}(C \cup D)\) is similar.
\end{proof}

\subsection{Exercise 47}
For all subsets $C$ and $D$ of $Y$, \(F^{-1}(C \cap D) = F^{-1}(C) \cap F^{-1}(D)\).

\begin{proof}
    True, the proof is extremely similar to exercise 46.
\end{proof}

\subsection{Exercise 48}
For all subsets $C$ and $D$ of $Y$, \(F^{-1}(C-D) = F^{-1}(C) - F^{-1}(D)\).

\begin{proof}
    1. Assume \(x \in F^{-1}(C-D)\) and let $y = F(x)$. {\it [Want to show \(x \in F^{-1}(C) - F^{-1}(D)\).]}

    2. By 1 and definition of \(F^{-1}(C-D), y \in C-D\).

    3. By 2 and definition of difference, \(y \in C\) and \(y \notin D\).

    4. By 3 and definition of \(F^{-1}(C)\), \(x \in F^{-1}(C)\). Similarly, since $y = F(x)$ and $y \notin D$,
    \(x \notin F^{-1}(D)\).

    5. By 4 and definition of difference, \(x \in F^{-1}(C) - F^{-1}(D)\).

    6. By 1, 5 and definition of subset, \(F^{-1}(C-D) \subseteq F^{-1}(C) - F^{-1}(D)\).

        {\it Now the reverse part:}

    7. Assume \(x \in F^{-1}(C) - F^{-1}(D)\) and let $y = F(x)$. {\it [Want to show \(x \in F^{-1}(C-D)\).]}

    8. By 7 and definition of difference, \(x \in F^{-1}(C)\) and \(x \notin F^{-1}(D)\).

    9. By 8 and definition of \(F^{-1}(C), y \in C\). Similarly \(y \notin D\).

    10. By 9 and definition of difference \(y \in C-D\).

    11. Since $y = F(x)$, by 10 and definition of \(F^{-1}(C-D)\), \(x \in F^{-1}(C-D)\).

    12. By 7, 11 and definition of subset, \(F^{-1}(C) - F^{-1}(D) \subseteq F^{-1}(C-D)\).

        {\it Conclusion:}

    13. By 6, 12 and definition of set equality \(F^{-1}(C) - F^{-1}(D) = F^{-1}(C-D)\).
\end{proof}

\subsection{Exercise 49}
\(F(F^{-1}(C)) \subseteq C\).

\begin{proof}
    1. Assume \(y \in F(F^{-1}(C))\). {\it [Want to show \(y \in C\).]}

    2. By 1 and definition of \(F(F^{-1}(C))\), there exists some \(x \in F^{-1}(C)\) such that \(y = F(x)\).

    3. By 2 and definition of \(F^{-1}(C)\), \(F(x) \in C\). So $y \in C$ because $y = F(x)$.

    4. By 1, 3 and definition of subset, \(F(F^{-1}(C)) \subseteq C\).
\end{proof}

\subsection{Exercise 50}
Given a set $S$ and a subset $A$, the characteristic function of $A$, denoted $\chi_A$, is the function defined
from $S$ to $\Z$ with the property that for each $u \in S$,
\[
    \chi_A(u) =
    \left\{
    \begin{tabular}{lr}
        \(1\) & if \(u \in A\)    \\
        \(0\) & if \(u \notin A\)
    \end{tabular}
    \right.
\]
Show that each of the following holds for all subsets $A$ and $B$ of $S$ and every $u \in S$.

\subsubsection{(a)}
\(\chi_{A \cap B}(u) = \chi_A(u) \cdot \chi_B(u)\)

\begin{proof}
    Assume $A, B$ are any subsets of $S$ and $u$ is any element in $S$. There are 4 cases:

    {\bf Case 1 (\(\bm{u \in A, u \in B}\)):} Then \(\chi_A(u) = 1 \) and \(\chi_B(u) = 1\).

    By definition of intersection \(u \in A \cap B\). Thus
    \(\chi_{A \cap B}(u) = 1\) also. Since $1 = 1 \cdot 1$,
    \(\chi_{A \cap B}(u) = \chi_A(u) \cdot \chi_B(u)\).

        {\bf Case 2 (\(\bm{u \in A, u \notin B}\)):} Then \(\chi_A(u) = 1 \) and \(\chi_B(u) = 0\).

    By definition of intersection \(u \notin A \cap B\). Thus
    \(\chi_{A \cap B}(u) = 0\) also. Since $0 = 1 \cdot 0$,
    \(\chi_{A \cap B}(u) = \chi_A(u) \cdot \chi_B(u)\).

        {\bf Case 3 (\(\bm{u \notin A, u \in B}\)):} Then \(\chi_A(u) = 0 \) and \(\chi_B(u) = 1\).

    By definition of intersection \(u \notin A \cap B\). Thus
    \(\chi_{A \cap B}(u) = 0\) also. Since $0 = 0 \cdot 1$,
    \(\chi_{A \cap B}(u) = \chi_A(u) \cdot \chi_B(u)\).

        {\bf Case 4 (\(\bm{u \notin A, u \notin B}\)):} Then \(\chi_A(u) = 0 \) and \(\chi_B(u) = 0\).

    By definition of intersection \(u \notin A \cap B\). Thus
    \(\chi_{A \cap B}(u) = 0\) also. Since $0 = 0 \cdot 0$,
    \(\chi_{A \cap B}(u) = \chi_A(u) \cdot \chi_B(u)\).
\end{proof}

\subsubsection{(b)}
\(\chi_{A \cup B}(u) = \chi_A(u) + \chi_B(u) - \chi_A(u) \cdot \chi_B(u)\)

\begin{proof}
    Assume $A, B$ are any subsets of $S$ and $u$ is any element in $S$. There are 4 cases:

    {\bf Case 1 (\(\bm{u \in A, u \in B}\)):} Then \(\chi_A(u) = 1 \) and \(\chi_B(u) = 1\).

    By definition of union \(u \in A \cup B\). Thus
    \(\chi_{A \cup B}(u) = 1\) also.
    Since $1 = 1 + 1 - (1 \cdot 1)$, \(\chi_{A \cup B}(u) = \chi_A(u) + \chi_B(u) - \chi_A(u) \cdot \chi_B(u)\).

        {\bf Case 2 (\(\bm{u \in A, u \notin B}\)):} Then \(\chi_A(u) = 1\) and \(\chi_B(u) = 0\).

    By definition of union \(u \in A \cup B\). Thus
    \(\chi_{A \cup B}(u) = 1\) also.
    Since $1 = 1 + 0 - (1 \cdot 0)$, \(\chi_{A \cup B}(u) = \chi_A(u) + \chi_B(u) - \chi_A(u) \cdot \chi_B(u)\).

        {\bf Case 3 (\(\bm{u \notin A, u \in B}\)):} Then \(\chi_A(u) = 0\) and \(\chi_B(u) = 1\).

    By definition of union \(u \in A \cup B\). Thus
    \(\chi_{A \cup B}(u) = 1\) also.
    Since $1 = 0 + 1 - (0 \cdot 1)$, \(\chi_{A \cup B}(u) = \chi_A(u) + \chi_B(u) - \chi_A(u) \cdot \chi_B(u)\).

        {\bf Case 4 (\(\bm{u \notin A, u \notin B}\)):} Then \(\chi_A(u) = 0 \) and \(\chi_B(u) = 0\).

    By definition of union \(u \notin A \cup B\). Thus
    \(\chi_{A \cup B}(u) = 0\) also.
    Since $0 = 0 + 0 - (0 \cdot 0)$, \(\chi_{A \cup B}(u) = \chi_A(u) + \chi_B(u) - \chi_A(u) \cdot \chi_B(u)\).
\end{proof}

{\bf \cy Each of exercises $51-53$ refers to the Euler phi function, denoted $\phi$, which is defined as follows: For
each integer \(n \geq 1\), $\phi(n)$ is the number of positive integers less than or equal to $n$ that have no
common factors with $n$ except $\pm 1$. For example, \(\phi(10) = 4\) because there are four positive integers
less than or equal to 10 that have no common factors with 10 except $\pm 1$, namely, 1, 3, 7, and 9.}

\subsection{Exercise 51}
Find each of the following:

\subsubsection{(a)}
\(\phi(15)\)

\begin{proof}
    \(\phi(15) = 8\) {\it [because $1, 2, 4, 7, 8, 11, 13$, and $14$ have no common factors with $15$ other than $\pm 1$]}
\end{proof}

\subsubsection{(b)}
\(\phi(2)\)

\begin{proof}
    \(\phi(2) = 1\) {\it [because the only positive integer less than or equal to $2$ having no common factors with $2$
                other than $\pm 1$ is $1$]}
\end{proof}

\subsubsection{(c)}
\(\phi(5)\)

\begin{proof}
    \(\phi(5) = 4\) {\it [because $1, 2, 3$, and $4$ have no common factors with $5$ other than $\pm 1$]}
\end{proof}

\subsubsection{(d)}
\(\phi(12)\)

\begin{proof}
    \(\phi(12) = 4\) (1, 5, 7, 11)
\end{proof}

\subsubsection{(e)}
\(\phi(11)\)

\begin{proof}
    \(\phi(11) = 10\) (1, 2, 3, 4, 5, 6, 7, 8, 9, 10)
\end{proof}

\subsubsection{(f)}
\(\phi(1)\)

\begin{proof}
    \(\phi(1) = 1\)
\end{proof}

\subsection{Exercise 52}
Prove that if $p$ is a prime number and $n$ is an integer with \(n \geq 1\), then \(\phi(p^n) = p^n - p^{n-1}\).

\begin{proof}
    Let $p$ be any prime number and $n$ any integer with \(n \geq 1\). There are \(p^{n-1}\) positive integers less than
    or equal to \(p^n\) that have a common factor other than $\pm 1$ with \(p^n\), namely, \(p, 2p, 3p, \ldots,
    (p^{n-1})p\). Hence, there are \(p^n - p^{n-1}\) positive integers less than or equal to \(p^n\) that do not have a
    common factor with \(p^n\) except for $\pm 1$.
\end{proof}

\subsection{Exercise 53}
Prove that there are infinitely many integers $n$ for which \(\phi(n)\) is a perfect square.

\begin{proof}
    By exercise 52, for any integer $n$ with \(n \geq 1, \phi(2^n) = 2^n - 2^{n-1} = 2^{n-1}\).

    So, for all integers $k$ with $k \geq 1$, we have that \(\phi(2^{2k+1}) = 2^{2k} = (2^k)^2\) is a perfect square.
\end{proof}

\section{Exercise Set 7.2}

\subsection{Exercise 1}
The definition of one-to-one is stated in two ways:
\[
    \forall x_1, x_2 \in X, \text{ if } F(x_1) = F(x_2) \text{ then } x_1 = x_2
\]
and
\[
    \forall x_1, x_2 \in X, \text{ if } x_1 \neq x_2 \text{ then } F(x_1) \neq F(x_2).
\]
Why are these two statements logically equivalent?

\begin{proof}
    The second statement is the contrapositive of the first.
\end{proof}

\subsection{Exercise 2}
Fill in each blank with the word most or least.

\subsubsection{(a)}
A function $F$ is one-to-one if, and only if, each element in the co-domain of $F$ is the image of at \fbl one element
in the domain of $F$.

\begin{proof}
    most
\end{proof}

\subsubsection{(b)}
A function $F$ is onto if, and only if, each element in the co-domain of $F$ is the image of at \fbl one element in the
domain of $F$.

\begin{proof}
    least
\end{proof}

\subsection{Exercise 3}
When asked to state the definition of one-to-one, a student replies, “A function $f$ is one-to-one if, and only if,
every element of $X$ is sent by $f$ to exactly one element of $Y$.” Give a counterexample to show that the student’s
reply is incorrect.

\begin{proof}
    One counterexample is given and explained below. Give a different counterexample and accompany it with an explanation.

    \underline{Counterexample:} Consider \(X = \{a,b\}, Y = \{u,v\}\) and the function $f$ defined by\(f(a) = f(b) =u\).
    Observe that $a$ is sent to exactly one element of $Y$, namely, $u$, and $b$ is also sent to exactly one element of
    $Y$, namely, $u$ also. So it is true that every element of $X$ is sent to exactly one element of $Y$. But $f$ is not
    one-to-one because $f(a) = f(b)$ whereas $a \neq b$. {\it [Note that to say, “Every element of $X$ is sent to exactly
                one element of $Y$” is just another way of saying that in the arrow diagram for the function there is only one arrow
                coming out of each element of $X$. But this statement is part of the definition of any function, not just of a
                one-to-one function.]}
\end{proof}

\subsection{Exercise 4}
Let \(f: X \to Y\) be a function. True or false? A sufficient condition for $f$ to be one-to-one is that for
every element $y$ in $Y$, there is at most one $x$ in $X$
with \(f(x) = y\). Explain your answer.

\begin{proof}
    True. Assume \(x_1, x_2 \in X\) and assume \(f(x_1) = f(x_2)\). {\it [We want to show \(x_1 = x_2\)].} Let
    \(y = f(x_1) = f(x_2)\). By assumption there is at most one $x$ in $X$ such that $y = f(x)$. Thus \(x = x_1 = x_2\),
    {\it [as was to be shown.]}
\end{proof}

\subsection{Exercise 5}
All but two of the following statements are correct ways to express the fact that a function $f$ is onto.
Find the two that are incorrect.

\subsubsection{(a)}
$f$ is onto $\iff$ every element in its co-domain is the image of some element in its domain.

\begin{proof}
    Correct.
\end{proof}

\subsubsection{(b)}
$f$ is onto $\iff$ every element in its domain has a corresponding image in its co-domain.

\begin{proof}
    Incorrect.
\end{proof}

\subsubsection{(c)}
$f$ is onto \(\iff  \forall y \in Y, \exists x \in X\) such that \(f(x) = y\).

\begin{proof}
    Correct.
\end{proof}

\subsubsection{(d)}
$f$ is onto \(\iff \forall x \in X, \exists y \in Y\) such that \(f(x) = y\).

\begin{proof}
    Incorrect.
\end{proof}

\subsubsection{(e)}
$f$ is onto $\iff$ the range of $f$ is the same as the co-domain of $f$.

\begin{proof}
    Correct.
\end{proof}

\subsection{Exercise 6}
Let \(X = \{1, 5, 9\}, Y = \{3, 4, 7\}\).

\subsubsection{(a)}
Define \(f: X \to Y\) by specifying that \(f(1) = 4, f(5) = 7, f(9) = 4\). Is $f$ one-to-one? Is $f$ onto?
Explain your answers.

\begin{proof}
    Not 1-1: \(f(1) = f(9)\) but \(1 \neq 9\). Not onto: no $x \in X$ such that $f(x) = 3$.
\end{proof}

\subsubsection{(b)}
Define \(g: X \to Y\) by specifying that \(g(1) = 7, g(5) = 3, g(9) = 4\). Is $g$ one-to-one? Is $g$ onto?
Explain your answers.

\begin{proof}
    $g$ is 1-1, because \(g(1) \neq f(5), g(1) \neq g(9)\) and \(g(1) \neq g(9)\).

    $g$ is onto because each element of $Y$ is the image of some $x \in X$: \(3 = g(5), 4 = g(9), 7 = g(1)\).
\end{proof}

\subsection{Exercise 7}
Let \(X = \{a, b, c, d\}\) and \(Y = \{e, f, g\}\). Define
functions $F$ and $G$ by the arrow diagrams below.

\begin{figure}[ht!]
    \centering
    \includegraphics[scale=0.5]{../images/7.2.7.png}
\end{figure}

\subsubsection{(a)}
Is $F$ one-to-one? Why or why not? Is it onto? Why or why not?

\begin{proof}
    $F$ is not 1-1 because $F(c) = F(d)$ but $c \neq d$.

    $F$ is onto, because each element in $Y$ is the image of some element in $X$.
\end{proof}

\subsubsection{(b)}
Is $G$ one-to-one? Why or why not? Is it onto? Why or why not?

\begin{proof}
    $G$ is 1-1 because $G(a) \neq G(b), G(a) \neq G(c)$ and $G(b) \neq G(c)$.

    $G$ is not onto, because there is no $x \in X$ such that $G(x) = g$.
\end{proof}

\subsection{Exercise 8}
Let \(X = \{a, b, c\}\) and \(Y = \{d, e, f, g\}\). Define
functions $H$ and $K$ by the arrow diagrams below.

\begin{figure}[ht!]
    \centering
    \includegraphics[scale=0.5]{../images/7.2.8.png}
\end{figure}

\subsubsection{(a)}
Is $H$ one-to-one? Why or why not? Is it onto? Why or why not?

\begin{proof}
    $H$ is not 1-1 because \(H(b) = H(c)\) but \(b \neq c\).

    $H$ is not onto, because there is no $x \in X$ such that $H(x) = g$.
\end{proof}

\subsubsection{(b)}
Is $K$ one-to-one? Why or why not? Is it onto? Why or why not?

\begin{proof}
    $K$ is 1-1 because \(K(a) \neq K(b), K(a) \neq K(c)\) and \(K(b) \neq K(c)\).

    $K$ is not onto, because there is no $x \in X$ such that $K(x) = g$.
\end{proof}

\subsection{Exercise 9}
Let \(X = \{1, 2, 3\}, Y = \{1, 2, 3, 4\}\), and \(Z = \{1, 2\}\).

\subsubsection{(a)}
Define a function \(f: X \to Y\) that is one-to-one but not onto.

\begin{proof}
    One example of many is the following:
    \begin{figure}[ht!]
        \centering
        \includegraphics[scale=0.5]{../images/7.2.9.png}
    \end{figure}
\end{proof}

\subsubsection{(b)}
Define a function \(g: X \to Z\) that is onto but not one-to-one.

\begin{proof}
    Let \(g(1) = 1, g(2) = 2, g(3) = 2\).
\end{proof}

\subsubsection{(c)}
Define a function \(h: X \to X\) that is neither one-to-one nor onto.

\begin{proof}
    Let \(h(1) = 1, h(2) = 1, h(3) = 1, h(4) = 1\).
\end{proof}

\subsubsection{(d)}
Define a function \(k: X \to X\) that is one-to-one and onto but is not the identity function on $X$.

\begin{proof}
    Let \(k(1) = 2, k(2) = 3, k(3) = 4, k(4) = 1\).
\end{proof}

\subsection{Exercise 10}
\subsubsection{(a)}
Define \(f: \Z \to \Z\) by the rule \(f(n) = 2n\), for every integer $n$.

(i) Is $f$ one-to-one? Prove or give a counterexample.

(ii) Is $f$ onto? Prove or give a counterexample.

\begin{proof}
    (i) $f$ is one-to-one. Suppose \(f(n_1) = f(n_2)\) for some integers $n_1$ and $n_2$. {\it [We must show that \(n_1 =
                n_2\).]} By definition of \(f, 2n_1 = 2n_2\), and dividing both sides by 2 gives $n_1 = n_2$ {\it [as was to be shown].}

    (ii) $f$ is not onto. \underline{Counterexample:} Consider \(1 \in \Z\). We claim that \(1 \neq f(n)\), for any
    integer $n$, because if there were an integer $n$ such that \(1 = f(n)\), then, by definition of \(f, 1 = 2n\).
    Dividing both sides by 2 would give \(n = 1/2\). But $1/2$ is not an integer. Hence \(1 \neq f(n)\) for any integer
    $n$, and so $f$ is not onto.
\end{proof}

\subsubsection{(b)}
Let $2\Z$ denote the set of all even integers. That is,
\[
    2\Z = \{n \in \Z \,|\,n=2k, \text{ for some integer } k\}.
\]
Define \(h: \Z \to 2\Z\) by the rule \(h(n) = 2n\), for each integer $n$. Is $h$ onto? Prove or give a counterexample.

\begin{proof}
    $h$ is onto. Suppose \(m \in 2\Z\). {\it [We must show that there exists an integer such that $h$ of that integer
                equals $m$.]} Since \(m \in 2\Z, m = 2k\) for some integer $k$. Then \(h(k) = 2k = m\). Hence there exists an integer
    (namely, $k$) such that \(h(k) = m\) {\it [as to be shown].}
\end{proof}

\subsection{Exercise 11}
\subsubsection{(a)}
Define \(g: \Z \to \Z\) by the rule \(g(n) = 4n - 5\), for every integer $n$.

(i) Is $g$ one-to-one? Prove or give a counterexample.

(ii) Is $g$ onto? Prove or give a counterexample.

\begin{proof}
    $g$ is 1-1: assume \(n_1, n_2 \in \Z\) and \(g(n_1) = g(n_2)\). {\it [We want to show \(n_1 = n_2\)].} By $g$'s
    definition \(4n_1-5=4n_2-5\). Adding 5 to both sides and dividing both sides by 4 we get \(n_1 = n_2\).

    $g$ is not onto: there is no $n \in \Z$ such that $g(n) = 0$. Argue by contradiction and assume \(g(n) = 0\) for some
    \(n \in \Z\). Then \(4n-5 = 0\) so \(n = 5/4\) which is not an integer, a contradiction.
\end{proof}

\subsubsection{(b)}
Define \(G: \R \to \R\) by the rule \(G(x) = 4x - 5\), for every real number $x$. Is $G$ onto? Prove or give a counterexample.

\begin{proof}
    $G$ is onto. Suppose $y$ is any element of $\R$. {\it [We must show that there is an element $x$ in $\R$ such that
                \(G(x) = y\). What would $x$ be if it exists? Scratch work shows that $x$ would have to equal \((y + 5)/4\). The proof
                must then show that $x$ has the necessary properties.]} Let \(x = (y + 5)/4\). Then (1) \(x \in \R\), and (2) \(G(x) =
    G((y + 5)/4) = 4[(y + 5)/4] - 5 = (y + 5) - 5 = y\) {\it [as was to be shown].}
\end{proof}

\subsection{Exercise 12}
\subsubsection{(a)}
Define \(F: \Z \to \Z\) by the rule \(F(n) = 2 - 3n\), for each integer $n$.

(i) Is $F$ one-to-one? Prove or give a counterexample.

(ii) Is $F$ onto? Prove or give a counterexample.

\begin{proof}
    $F$ is 1-1: assume \(n_1, n_2 \in \Z\) and \(F(n_1) = F(n_2)\). {\it [We want to show \(n_1 = n_2\).]} By $F$'s
    definition, \(2 - 3n_1 = 2-3n_2\). Subtracting 2 from both sides and dividing by $-3$ we get \(n_1 = n_2\).

    $F$ is not onto: there is no \(n \in \Z\) such that \(F(n) = 0\). Because otherwise \(2 - 3n = 0\) for some integer
    $n$, but then \(n = 2/3\) which is not an integer, a contradiction.
\end{proof}

\subsubsection{(b)}
Define \(G: \R \to \R\) by the rule \(G(x) = 2 - 3x\), for each real number $x$. Is $G$ onto? Prove or give a counterexample.

\begin{proof}
    $G$ is 1-1 just like $F$ above. Same proof applies.

    $G$ is also onto: for every $y \in \R$ there exists an $x \in \R$ such that $G(x) = y$: let \(x = (y - 2) / (-3)\).
    Then \(\dps G(x) = G((y - 2) / (-3)) = 2 - 3 \cdot \frac{y-2}{-3} = 2 + (y-2) = y\).
\end{proof}

\subsection{Exercise 13}
\subsubsection{(a)}
Define \(H: \R \to \R\) by the rule \(H(x) = x^2\), for each real number $x$.

(i) Is $H$ one-to-one? Prove or give a counterexample.

(ii) Is $H$ onto? Prove or give a counterexample.

\begin{proof}
    (i) $H$ is not one-to-one. \underline{Counterexample:} \(H(1) = 1 = H(-1)\) but \(1 \neq -1\).

    (ii) H is not onto. \underline{Counterexample:} \(H(x) \neq -1\) for any real number $x$ because \(H(x) = x^2\) and no
    real numbers have negative squares.
\end{proof}

\subsubsection{(b)}
Define \(K: \R^{nonneg} \to \R^{nonneg}\) by the rule \(K(x) = x^2\), for each nonnegative real number $x$.
Is $K$ onto? Prove or give a counterexample.

\begin{proof}
    $K$ is onto. Given any \(y \in \R^{nonneg}\) let \(x = \sqrt{y}\). Then \(K(x) = K(\sqrt{y}) = (\sqrt{y})^2 = y\).
\end{proof}

\subsection{Exercise 14}
Explain the mistake in the following “proof.”

{\bf Theorem:} The function \(f: \Z \to \Z\) defined by the formula \(f(n) = 4n + 3\), for each integer $n$, is one-to-one.

{\bf “Proof:} Suppose any integer $n$ is given. Then by definition of $f$, there is only one possible value for
$f(n)$, namely, \(4n + 3\). Hence $f$ is one-to-one.”

\begin{proof}
    The “proof” claims that $f$ is one-to-one because for each integer $n$ there is only one possible value for $f(n)$.
    But to say that for each integer $n$ there is only one possible value for $f(n)$ is just another way of saying
    that $f$ satisfies one of the conditions necessary for it to be a function. To show that $f$ is one-to-one, one must
    show that any integer $n$ has a different function value from that of the integer $m$ whenever \(n \neq m\).
\end{proof}

{\bf \cy In each of $15-18$ a function $f$ is defined on a set of real numbers. Determine whether or not $f$ is one-
to-one and justify your answer.}

\subsection{Exercise 15}
\(\dps f(x) = \frac{x+1}{x}\), for each real number \(x \neq 0\)

\begin{proof}
    $f$ is 1-1: Suppose \(f(x_1) = f(x_2)\) where $x_1$ and $x_2$ are nonzero real numbers. {\it [We must show that
                \(x_1 = x_2\).]} By definition of $f$,
    \[
        \frac{x_1 + 1}{x_1} = \frac{x_2 + 1}{x_2}
    \]
    Cross-multiplying gives \(x_1x_2 + x_2 = x_1x_2 + x_1\) and subtracting $x_1x_2$ gives \(x_1 = x_2\).
\end{proof}

\subsection{Exercise 16}
\(\dps f(x) = \frac{x}{x^2+1}\), for each real number \(x\)

\begin{proof}
    $f$ is not one-to-one. \underline{Counterexample:} Note that

    \begin{center}
        \begin{tabular}{ccccccc}
            \(\dps\frac{x_1}{x_1^2+1}\) & = & \(\dps\frac{x_2}{x_2^2+1}\) & \(\implies\) & \(x_1x_2^2 + x_1\)      & =  & \(x_2x_1^2 + x_2\) \\
                                        &   &                             & \(\implies\) & \(x_1x_2^2 - x_2x_1^2\) & =  & \(x_2-x_1\)        \\
                                        &   &                             & \(\implies\) & \(x_1x_2(x_2-x_1)\)     & =  & \(x_2-x_1\)        \\
                                        &   &                             & \(\implies\) & \(x_1 = x_2\)           & or & \(x_1x_2 = 1\).
        \end{tabular}
    \end{center}

    Thus take any $x_1$ and $x_2$ with $x_1 \neq x_2$ but $x_1x_2 = 1$. For instance, take
    \(x_1 = 2\) and \(x_2 = 1/2\). Then \(f(x_1) = f(2) = 2/5\) and \(f(x_2) = f(1/2) = 2/5\), but $2 \neq 1/2$.
\end{proof}

\subsection{Exercise 17}
\(\dps f(x) = \frac{3x-1}{x}\), for each real number \(x \neq 0\)

\begin{proof}
    $f$ is 1-1: Assume \(x_1 \neq 0 \neq x_2\) and assume \(\frac{3x_1-1}{x_1} = \frac{3x_2-1}{x_2}\).
        {\it [We want to show \(x_1 = x_2\).]} Cross-multiplying gives \((3x_1-1)x_2 = (3x_2-1)x_1\). So
    \(3x_1x_2 - x_2 = 3x_1x_2 - x_1\). Canceling $3x_1x_2$ gives \(-x_2 = -x_1\), so \(x_1 = x_2\).
\end{proof}

\subsection{Exercise 18}
\(\dps f(x) = \frac{x+1}{x-1}\), for each real number \(x \neq 1\)

\begin{proof}
    $f$ is 1-1: Assume \(x_1 \neq 1 \neq x_2\) and assume \(\frac{x_1+1}{x_1-1} = \frac{x_2+1}{x_2-1}\).
        {\it [We want to show \(x_1 = x_2\).]} Cross-multiplying gives \((x_1+1)(x_2-1) = (x_2+1)(x_1-1)\). So
    \(x_1x_2 - x_1 + x_2 - 1 = x_1x_2 + x_1 - x_2 - 1\). Canceling $x_1x_2 - 1$ gives \(-x_1 + x_2 = x_1 - x_2\), so
    \(2x_2 = 2x_1\) and dividing by 2 gives \(x_2 = x_1\).
\end{proof}

\subsection{Exercise 19}
Referring to Example 7.2.3, assume that records with the following ID numbers are to be placed in sequence into
Table 7.2.1. Find the position into which each record is placed.

\begin{figure}[ht!]
    \centering
    \includegraphics[scale=0.3]{../images/7.2.1.png}
\end{figure}

\subsubsection{(a)}
417302072

\begin{proof}
    When 417302072 is divided by 11, the remainder is 0. So, \(417302072 \mod 11 = H(417302072) = 0\).
    Since position 0 is unoccupied, the record is placed there.
\end{proof}

\subsubsection{(b)}
364981703

\begin{proof}
    When 364981703 is divided by 11, the remainder is 9. So, \(364981703 \mod 11 = H(364981703) = 9\).
    Since position 9 is unoccupied, the record is placed there.
\end{proof}

\subsubsection{(c)}
283090787

\begin{proof}
    When 283090787 is divided by 11, the remainder is 1. So, \(283090787 \mod 11 = H(283090787) = 1\). Since position 1
    is occupied, the record is placed in position 3.
\end{proof}

\subsection{Exercise 20}
Define Floor\(: \R \to \Z\) by the formula Floor\((x) = \floor{x}\), for every real number $x$.

\subsubsection{(a)}
Is Floor one-to-one? Prove or give a counterexample.

\begin{proof}
    Floor is not one-to-one. \underline{Counterexample:} Floor(0) = 0 = Floor(1/2) but $0 \neq 1/2$.
\end{proof}

\subsubsection{(b)}
Is Floor onto? Prove or give a counterexample.

\begin{proof}
    Floor is onto. Suppose $m \in \Z$. {\it [We must show that there exists a real number $y$ such that Floor$(y) = m$.]}
    Let $y = m$. Then Floor$(y)$ = Floor$(m) = m$ since $m$ is an integer. (Actually, Floor takes the value $m$ for all
    real numbers in the interval \(m \leq x < m + 1\).) Hence there exists a real number $y$ such that Floor$(y) = m$
    {\it [as was to be shown].}
\end{proof}

\subsection{Exercise 21}
Let $S$ be the set of all strings of 0’s and 1’s, and define \(L: S \to \Z^{nonneg}\) by \(L(s) =\) the length of
$s$, for every string $s$ in $S$.

\subsubsection{(a)}
Is $L$ one-to-one? Prove or give a counterexample.

\begin{proof}
    $L$ is not one-to-one. \underline{Counterexample:} \(L(0) = L(1) = 1\) but $1 \neq 0$.
\end{proof}

\subsubsection{(b)}
Is $L$ onto? Prove or give a counterexample.

\begin{proof}
    $L$ is onto. Suppose $n$ is a nonnegative integer. {\it [We must show that there exists a string $s$ in $S$ such that
                \(L(s) = n\).]} Let
    \[
        s =
        \left\{
        \begin{tabular}{lr}
            \(\lambda\) (the null string) & if \(n = 0\) \\
            \(00\ldots0\) (with $n$ 0's)  & if \(n > 0\)
        \end{tabular}
        \right.
    \]
    Then \(L(s) =\) the length of $s = n$ {\it [as was to be shown].}
\end{proof}

\subsection{Exercise 22}
Let $S$ be the set of all strings of 0’s and 1’s, and define \(D: S \to \Z\) as follows: for every string $s$ in
$S$, \(D(s) =\) the number of 1's in $s$ minus the number of 0's in $s$.

\subsubsection{(a)}
Is $D$ one-to-one? Prove or give a counterexample.

\begin{proof}
    No. \underline{Counterexample:} \(D(10) = 0 = D(01)\) but \(10 \neq 01\).
\end{proof}

\subsubsection{(b)}
Is $D$ onto? Prove or give a counterexample.

\begin{proof}
    Yes. Given any $n \in \Z$, if $n < 0$ then let $s$ be the string of $-n$ consecutive 0's. Then $D(s) = n$.
    Similarly if $n > 0$ then let $s$ be the string of $n$ consecutive 1's. Then $D(s) = n$.
    If $n = 0$ then $D(\lambda) = n$.
\end{proof}

\subsection{Exercise 23}
Define \(F: \ps(\{a, b, c\}) \to \Z\) as follows: For every
$A$ in \(\ps(\{a, b, c\})\), \(F(A) =\) the number of
elements in $A$.

\subsubsection{(a)}
Is $F$ one-to-one? Prove or give a counterexample.

\begin{proof}
    $F$ is not one-to-one. \underline{Counterexample:} Let \(A = \{a\}, B = \{b\}\). Then \(F(A) = F(B) = 1\) but \(A \neq B\).
\end{proof}

\subsubsection{(b)}
Is $F$ onto? Prove or give a counterexample.

\begin{proof}
    No. \underline{Counterexample:} There are no subsets of $A$ with a negative number of, say, $-1$ elements.
\end{proof}

\subsection{Exercise 24}
Let $S$ be the set of all strings of $a$’s and $b$’s, and define \(N: S \to \Z\) by \(N(s) =\) the number of $a$’s in
$s$, for each \(s \in S\).

\subsubsection{(a)}
Is $N$ one-to-one? Prove or give a counterexample.

\begin{proof}
    No. \underline{Counterexample:} $N(a) = 1 = N(ab)$ but $a \neq ab$.
\end{proof}

\subsubsection{(b)}
Is $N$ onto? Prove or give a counterexample.

\begin{proof}
    No. \underline{Counterexample:} There are no strings with $-1$ $a$'s in it.
\end{proof}

\subsection{Exercise 25}
Let $S$ be the set of all strings in $a$’s and $b$’s, and
define \(C: S \to S\) by \(C(s) = as\), for each $s \in S$.
($C$ is called concatenation by $a$ on the left.)

\subsubsection{(a)}
Is $C$ one-to-one? Prove or give a counterexample.

\begin{proof}
    Yes. Assume \(C(s_1) = C(s_2)\). So \(as_1 = as_2\). Then $s_1$ and $s_2$ are the same string.
\end{proof}

\subsubsection{(b)}
Is $C$ onto? Prove or give a counterexample.

\begin{proof}
    No. \underline{Counterexample:} there is no string $s$ such that \(C(s) = b\) because $b$ has no $a$'s in it.
\end{proof}

\subsection{Exercise 26}
Define \(S: \Z^+ \to \Z^+\) by the rule: For each integer $n$, $S(n) =$ the sum of the positive divisors of $n$.

\subsubsection{(a)}
Is $S$ one-to-one? Prove or give a counterexample.

\begin{proof}
    No. \underline{Counterexample:} \(S(6) = 1 + 2 + 3 + 6 = 12\) and \(S(11) = 1 + 11 = 12\) but \(6 \neq 11\).
\end{proof}

\subsubsection{(b)}
Is $S$ onto? Prove or give a counterexample.

\begin{proof}
    No. \underline{Counterexample:} In order for there to be a positive integer $n$ such that $S(n) = 5$, $n$ would have
    to be less than 5. But \(S(1) = 1, S(2) = 3, S(3) = 4\), and \(S(4) = 7\). Hence there is no positive integer $n$
    such that \(S(n) = 5\).
\end{proof}

\subsection{Exercise 27}
Let $D$ be the set of all finite subsets of positive integers, and define \(T: \Z^+ \to D\) by the following
rule: For every integer $n$, $T(n) =$ the set of all of the positive divisors of $n$.

\subsubsection{(a)}
Is $T$ one-to-one? Prove or give a counterexample.

\begin{proof}
    Yes. Assume \(T(n_1) = T(n_2)\). {\it [We want to show \(n_1 = n_2\)].} Argue by contradiction and assume
    \(n_1 \neq n_2\). Then either \(n_1 < n_2\) or \(n_1 > n_2\). In the first case, $n_2$ is a positive divisor of
    $n_2$ so \(n_2 \in T(n_2)\). But since \(T(n_1) = T(n_2), n_2 \in T(n_1)\) too. So $n_2$ is a positive divisor of
    $n_1$, contradiction. The other case is similar.
\end{proof}

\subsubsection{(b)}
Is $T$ onto? Prove or give a counterexample.

\begin{proof}
    No. \underline{Counterexample:} There is no \(n \in \Z^+\) such that \(T(n) = \{1, 2, 3\}\), any such $n$ would also
    be divisible by 6, so \(T(n)\) would have to include 6 too.
\end{proof}

\subsection{Exercise 28}
Define \(G: \R \times \R \to \R \times \R\) as follows: \(G(x, y) = (2y, -x)\) for every \((x, y) \in \R \times \R\).

\subsubsection{(a)}
Is $G$ one-to-one? Prove or give a counterexample.

\begin{proof}
    Yes. Suppose \((x_1, y_1)\) and \((x_2, y_2)\) are any elements of \(\R \times \R\) such that \(G(x_1, y_1) =
    G(x_ 2, y_2)\). {\it [We must show that \((x_1, y_1) = (x_2, y_2)\).]} Then, by definition of $G$, \((2y_1, -x_1)
    = (2y_2, -x_2)\), and, by definition of ordered pair, \(2y_1 = 2y_2\) and \(-x_1 = -x_2\). Dividing both sides of
    the equation on the left by 2 and both sides of the equation on the right by $-1$ gives that \(y_1 = y_2\) and
    \(x_1 = x_2\), and so, by definition of ordered pair, \((x_1, y_1) = (x_2, y_2)\) {\it [as was to be shown].}
\end{proof}

\subsubsection{(b)}
Is $G$ onto? Prove or give a counterexample.

\begin{proof}
    Yes. Suppose \((u, v)\) is any element of \(\R \times \R\). {\it [We must show that there is an element \((x, y)\) in
                \(\R \times \R\) such that \(G(x, y) = (u, v)\).]} Let \((x, y) = (-v, u/2)\). Then (1) \((x, y) \in \R \times \R\)
    and (2) \(G(x, y) = (2y, -x) = (2(u/2), -(-v)) = (u, v)\) {\it [as was to be shown].}
\end{proof}

\subsection{Exercise 29}
Define \(H: \R \times \R \to \R \times \R\) as follows: \(H(x, y) = (x+1, 2-y)\) for every \((x, y) \in \R \times \R\).

\subsubsection{(a)}
Is $H$ one-to-one? Prove or give a counterexample.

\begin{proof}
    Yes. Assume \(H(x_1, y_1) = H(x_2, y_2)\). {\it [We want to show that \(x_1 = x_2\) and \(y_1 = y_2)\).]}
    We have \((x_1+1, 2-y_1) = (x_2+1, 2-y_2)\). By definition of a tuple, \(x_1+1 = x_2+1\) and \(2-y_1 = 2-y_2\).
    Solving, we get \(x_1 = x_2\) and \(y_1 = y_2)\).
\end{proof}

\subsubsection{(b)}
Is $H$ onto? Prove or give a counterexample.

\begin{proof}
    Yes. Assume \((x,y) \in \R \times \R\) is any pair of real numbers. {\it [We want to show there exists a pair
                \((a, b) \in \R \times \R\) such that \(H(a,b) = (x,y)\).]} Let \(a = x-1, b = 2-y\). Then \(H(a, b) = (a+1, 2-b) =
    (x-1+1, 2-(2-y)) = (x,y)\), {\it [as was to be shown.]}
\end{proof}

\subsection{Exercise 30}
Define \(J: \Q \times \Q \to \R\) as follows: \(J(r, s) = r + \sqrt{2}s\) for every \((r, s) \in \Q \times \Q\).

\subsubsection{(a)}
Is $J$ one-to-one? Prove or give a counterexample.

\begin{proof}
    Yes. Assume \(J(r_1, s_1) = J(r_2, s_2)\). {\it [We want to show that \(r_1 = r_2\) and \(s_1 = s_2)\).]}
    We have \(r_1 + \sqrt{2}s_1 = r_2 + \sqrt{2}s_2\). Moving terms around we get (*) \(r_1-r_2=\sqrt{2}(s_2-s_1)\).
    For the moment assume \(s_1 \neq s_2\). Then \(s_2 - s_1 \neq 0\). Dividing, we get
    \[
        \frac{r_1 - r_2}{s_2 - s_1} = \sqrt{2}.
    \]
    This is a contradiction, since the right hand side is an irrational number, and the left hand side is a rational
    number (being a ratio of differences of rational numbers). Thus our supposition was false and \(s_1 = s_2\). Going
    back to the equation (*), this gives \(r_1-r_2=\sqrt{2} \cdot 0 = 0\) and thus \(r_1 = r_2\), as was to be shown.
\end{proof}

\subsubsection{(b)}
Is $J$ onto? Prove or give a counterexample.

\begin{proof}
    No. \underline{Counterexample:} There are no rationals $(r, s)$ such that \(J(r,s) = \sqrt{3}\). Argue by contradiction
    and assume \(J(r,s) = \sqrt{3}\) for some rationals $r,s$. Then \(r + \sqrt{2}s = \sqrt{3}\). Squaring both sides, we
    get
    \[
        (r + \sqrt{2}s)^2 = \sqrt{3}^2 \implies r^2 + 2rs\sqrt{2} + 2s^2 = 3 \implies \sqrt{2} = \frac{3 - r^2 - 2s^2}{2rs}.
    \]
    This is a contradiction, since the left hand side is irrational, and the right side is rational (being a ratio
    of differences of products of rationals). Thus our supposition was false, and $J$ is not onto.
\end{proof}

\subsection{Exercise 31}
Define \(F: \Z^+ \times \Z^+ \to \Z^+\) and \(G: \Z^+ \times \Z^+ \to \Z^+\) as follows: for each \((n, m) \in
\Z^+ \times \Z^+, F(n, m) = 3^n5^m\) and \(G(n, m) = 3^n6^m\).

\subsubsection{(a)}
Is $F$ one-to-one? Prove or give a counterexample.

\begin{proof}
    Yes. Assume \(F(n_1, m_1) = F(n_2, m_2)\). Then \(3^{n_1}5^{m_1} = 3^{n_2}5^{n_2}\). Call this positive integer $z$.
    So we have two different prime factorizations of the same integer $z$. By the uniqueness part of the factorization
    theorem, the exponents are unique, therefore \(n_1 = n_2\) and \(m_1 = m_2\).
\end{proof}

\subsubsection{(b)}
Is $G$ one-to-one? Prove or give a counterexample.

\begin{proof}
    Yes. We can rewrite \(G(n, m) = 3^n6^m = 3^n (2 \cdot 3)^m = 3^n \cdot 2^m \cdot 3^m = 2^m \cdot 3^{m+n}\). Now
    assume \(G(n_1, m_1) = G(n_2, m_2)\). Then \(2^{m_1} \cdot 3^{m_1+n_1} = 2^{m_2} \cdot 3^{m_2+n_2}\). Similar to part
    (a), by the uniqueness of prime factorizations, the exponents must be equal, so \(m_1 = m_2\) and
    \(m_1 + n_1 = m_2 + n_2\). Using \(m_1 = m_2\) in the second equation we can cancel the $m$s to get \(n_1=n_2\).
\end{proof}

\subsection{Exercise 32}
\subsubsection{(a)}
Is \(\log_8 27 = \log_2 3\)? Why or why not?

\begin{proof}
    1. Let \(x = \log_8 27, y = \log_2 3\).

    2. By 1 and definition of logarithm, \(8^x = 27\) and \(2^y = 3\).

    3. By 2 and laws of exponents, \((2^3)^x = 2^{3x} = (2^x)^3\) and \(27 = 3^3\). So \((2^x)^3 = 3^3\).

    4. By 3, taking the cube root of both sides, we get \(2^x = 3\).

    5. By 2 and 4, \(2^x = 2^y\). Applying \(\log_2\) to both sides, we get $x = y$.
\end{proof}

\subsubsection{(b)}
Is \(\log_{16} 9 = \log_4 3\)? Why or why not?

\begin{proof}
    1. Let \(x = \log_{16} 9, y = \log_4 3\).

    2. By 1 and definition of logarithm, \(16^x = 9\) and \(4^y = 3\).

    3. By 2 and laws of exponents, \((2^4)^x = 2^{4x} = (2^{2x})^2\) and \(9 = 3^2\). So \((2^{2x})^2 = 3^2\).
    Similarly \(2^{2y} = 3\).

    4. By 3, taking the square root of both sides, we get \(2^{2x} = 3\).

    5. By 2 and 4, \(2^{2x} = 2^{2y}\). Applying \(\log_2\) to both sides, we get $2x = 2y$ so $x = y$.
\end{proof}

{\bf \cy The properties of logarithm established in $33-35$ are used in Sections 11.4 and 11.5.}

\subsection{Exercise 33}
Prove that for all positive real numbers $b$, $x$, and $y$ with \(b \neq 1\),
\[
    \log_b\left(\frac{x}{y}\right) = \log_b x - \log_b y.
\]
\begin{proof}
    Suppose that $b$, $x$, and $y$ are any positive real numbers such that \(b \neq 1\). Let \(u = \log_b(x)\) and
    \(v = \log_b(y)\). By definition of logarithm, \(b^u = x\) and \(b_v = y\). By substitution, \(\frac{x}{y} =
    \frac{b^u}{b^v} = b^{u-v}\) {\it [by (7.2.3) and the fact that \(b^{-v} = \frac{1}{b^v}\)].} Translating
    \(\frac{x}{y} = b^{u-v}\) into logarithmic form gives \(\log_b \frac{x}{y} = u-v\), and so, by substitution,
    \(\log_b \frac{x}{y} = \log_b(x) - \log_b(y)\), {\it [as was to be shown].}
\end{proof}

\subsection{Exercise 34}
Prove that for all positive real numbers $b$, $x$, and $y$ with \(b \neq 1\),
\[
    \log_b(xy) = \log_b x + \log_b y.
\]
\begin{proof}
    1. Let \(z = \log_b(xy), m = \log_b x, n = \log_b y\).

    2. By 1 and definition of log, \(b^z = xy\).

    3. By 1 and definition of log, \(b^m = x\).

    4. By 1 and definition of log, \(b^n = y\).

    5. By 2, 3, 4, and law of exponents, \(b^z = xy = b^m \cdot b^n = b^{m+n}\).

    6. By 5, applying $\log_b$ to both sides, we get \(\log_b(b^z) = \log_b(b^{m+n})\), which gives $z = m+n$.

    7. By 1 and 6, \(\log_b(xy) = \log_b x + \log_b y\).
\end{proof}

\subsection{Exercise 35}
Prove that for all real numbers $a$, $b$ and $x$ with $b$ and $x$ positive and \(b \neq 1\),
\[
    \log_b (x^a) = a \log_b x.
\]
\begin{proof}
    1. Let \(z = \log_b(x^a), m = \log_b x\).

    2. By 1 and definition of log, \(b^z = x^a\).

    3. By 1 and definition of log, \(b^m = x\).

    4. By 3 and law of exponents, \((b^m)^a = x^a\) so \(b^{am} = x^a\).

    5. By 2 and 4, \(b^z = x^a = b^{am}\).

    6. By 5, applying $\log_b$ to both sides, we get \(\log_b(b^z) = \log_b(b^{am})\), which gives $z = am$.

    7. By 1 and 6, \(\log_b(x^a) = a \log_b x\).
\end{proof}

{\bf \cy Exercises 36 and 37 use the following definition: If \(f: \R \to \R\) and \(g: \R \to \R\) are functions,
then the function \((f + g): \R \to \R\) is defined by the formula \((f + g)(x) = f(x) + g(x)\) for every real number $x$.}

\subsection{Exercise 36}
If \(f: \R \to \R\) and \(g: \R \to \R\) are both one-to-one, is \(f + g\) also one-to-one? Justify your answer.

\begin{proof}
    No. \underline{Counterexample:} Define \(f: \R \to \R\) and \(g: \R \to \R\) as follows: \(f(x) = x\) and \(g(x) = -x\)
    for every real number $x$. Then $f$ and $g$ are both one-to-one {\it [because for all real numbers $x_1$ and $x_2$,
        if \(f(x_1) = f(x_2)\) then \(x_1 = x_2\), and if \(g(x_1) = g(x_2)\) then \(-x_1 = -x_2\), so \(x_1 = x_2\) in this
        case as well].} But \(f + g\) is not one-to-one {\it [because $f + g$ satisfies the equation \((f + g)(x) = x +
        (-x) = 0\) for every real number $x$, and so, for instance, \((f + g)(1) = (f + g)(2)\) but $1 \neq 2$].}
\end{proof}

\subsection{Exercise 37}
If \(f: \R \to \R\) and \(g: \R \to \R\) are both onto, is \(f + g\) also onto? Justify your answer.

\begin{proof}
    No. Same counterexample as exercise 36 works. \(f(x) = x\) and \(g(x) = -x\) are both onto, but \((f+g)(x) = 0\) is not.
\end{proof}

{\bf \cy Exercises 38 and 39 use the following definition: If \(f: \R \to \R\) is a function and $c$ is a nonzero real
number, then the function \((c \cdot f): \R \to \R\) is defined by the formula \((c \cdot f)(x) = c \cdot (f(x))\)
for every real number $x$.}

\subsection{Exercise 38}
Let \(f: \R \to \R\) be a function and $c$ a nonzero real number. If $f$ is one-to-one, is \(c \cdot f\) also one-to-
one? Justify your answer.

\begin{proof}
    Yes. Let $f$ be a one-to-one function from $\R$ to $\R$, and let $c$ be any nonzero real number. Suppose
    \((c \cdot f)(x_1) = (c \cdot f)(x_2)\). {\it [We must show that \(x_1 = x_2\).]} It follows by definition of
    \((c \cdot f)\) that \(c \cdot (f(x1)) = c \cdot (f(x_2))\). Since \(c \neq 0\), we may divide both sides of the
    equation by $c$ to obtain \(f(x_1) = f(x_2)\). And since $f$ is one-to-one, this implies that \(x_1 = x_2\),
    {\it [as was to be shown].}
\end{proof}

\subsection{Exercise 39}
Let \(f: \R \to \R\) be a function and $c$ a nonzero real number. If $f$ is onto, is \(c \cdot f\) also onto?
Justify your answer.

\begin{proof}
    Yes. Suppose \(y \in \R\). {\it [We want to show there exists \(x \in \R\) such that \((c \cdot f)(x) = y\)].}
    Since \(c \neq 0\) and $f$ is onto, there exists \(z \in \R\) such that \(f(z) = y/c\). Let $x = z$. Then
    \((c \cdot f)(x) = c \cdot f(x) = c \cdot f(z) = c \cdot (y/c) = y\), {\it [as was to be shown]}.
\end{proof}

\subsection{Exercise 40}
Suppose \(F: X \to Y\) is one-to-one.

\subsubsection{(a)}
Prove that for every subset \(A \subseteq X, F^{-1}(F(A)) = A\).

\begin{proof}
    Assume \(A \subseteq X\).

    1. Assume \(x \in F^{-1}(F(A))\). {\it [Want to show \(x \in A\).]}

    2. By 1 and definition of inverse image \(F^{-1}(F(A)) = \{t \in X \, | \, F(t) \in F(A)\}\) applied to $t = x$, we have \(x \in X\) and \(F(x) \in F(A)\).

    3. By 2 and definition of $F(A)$, there exists \(r \in A\) such that \(F(r) = F(x)\).

    4. By 3 and since $F$ is 1-1, $r = x$, hence \(x \in A\).

    5. By 1, 4 and definition of subset, \(F^{-1}(F(A)) \subseteq A\).

        {\it Now the reverse direction:}

    6. Assume \(x \in A\). {\it [Want to show \(x \in F^{-1}(F(A))\).]}

    7. By 6 and definition of $F(A)$, we have \(F(x) \in F(A)\).

    8. By 6 and 7, \(x \in X\) and \(F(x) \in F(A)\). So $x$ satisfies the definition of being a member of the inverse
    image \(F^{-1}(F(A)) = \{t \in X \, | \, F(t) \in F(A)\}\) (with $t = x$). Therefore we have \(x \in F^{-1}(F(A))\).

    9. By 6, 8 and definition of subset, \(A \subseteq F^{-1}(F(A))\).

        {\it Conclusion:}

    10. By 5, 9 and definition of set equality, \(F^{-1}(F(A)) = A\).
\end{proof}

\subsubsection{(b)}
Prove that for all subsets $A_1$ and $A_2$ in $X$, \(F(A_1 \cap A_2) = F(A_1) \cap F(A_2)\).

\begin{proof}
    Assume \(A_1 \subseteq X, A_2 \subseteq X\).

    1. Assume \(y \in F(A_1 \cap A_2)\). {\it [Want to show \(y \in F(A_1) \cap F(A_2)\).]}

    2. By 1 and definition of \(F(A_1 \cap A_2)\), there exists \(x \in A_1 \cap A_2\) such that \(y = F(x)\).

    3. By 2 and definition of intersection, \(x \in A_1\) and \(x \in A_2\).

    4. By 3, and since $y = F(x)$, and by definitions of $F(A_1)$ and $F(A_2)$, \(y \in F(A_1)\) and \(y \in F(A_2)\).

    5. By 4 and definition of intersection, \(y \in F(A_1) \cap F(A_2)\).

    6. By 1, 5 and definition of subset, \(F(A_1 \cap A_2) \subseteq F(A_1) \cap F(A_2)\).

        {\it Now the reverse direction:}

    7. Assume \(y \in F(A_1) \cap F(A_2)\). {\it [Want to show \(y \in F(A_1 \cap A_2)\).]}

    8. By 7 and definition of intersection, \(y \in F(A_1)\) and \(y \in F(A_2)\).

    9. By 8 and definitions of \(F(A_1)\) and \(F(A_2)\), there exist \(x_1 \in A_1\) and \(x_2 \in A_2\) such that \(y = F(x_1)\) and \(y = F(x_2)\).

    10. By 9 and since $F$ is 1-1, \(x_1 = x_2\).

    11. By 9, 10 and definition of intersection, \(x_1 \in A_1 \cap A_2\).

    12. By 11 and since \(y = F(x_1), y \in F(A_1 \cap A_2)\).

    13. By 7, 12 and definition of subset, \(F(A_1) \cap F(A_2) \subseteq F(A_1 \cap A_2)\).

        {\it Conclusion:}

    14. By 6, 13 and definition of set equality, \(F(A_1 \cap A_2) = F(A_1) \cap F(A_2)\).
\end{proof}

\subsection{Exercise 41}
Suppose \(F: X \to Y\) is onto. Prove that for every subset \(B \subseteq Y, F(F^{-1}(B)) = B\).

\begin{proof}
    Assume \(B \subseteq Y\).

    1. Assume \(y \in F(F^{-1}(B))\). {\it [Want to show \(y \in B\).]}

    2. By 1 and definition of \(F(F^{-1}(B))\), there exists \(x \in F^{-1}(B)\) such that \(F(x) = y\).

    3. By 2 and definition of inverse image \(F^{-1}(B) = \{t \in X \, | \, F(t) \in B\}\), we have \(F(x) \in B\).

    4. By 2 and 3, \(y = F(x) \in B\). So \(y \in B\).

    5. By 1, 4 and definition of subset, \(F(F^{-1}(B)) \subseteq B\).

        {\it Now the reverse direction:}

    6. Assume \(y \in B\). {\it [Want to show \(y \in F(F^{-1}(B))\).]}

    7. By 6 and since $F$ is onto, there exists \(x \in X\) such that \(F(x) = y\).

    8. By 6 and 7, \(x \in X\) and \(F(x) \in B\). So $x$ satisfies the definition of being a member of the inverse
    image \(F^{-1}(B) = \{t \in X \, | \, F(t) \in B\}\) (with $t = x$). So \(x \in F^{-1}(B)\).

    9. By 7 and 8, \(y = F(x) \in F(F^{-1}(B))\).

    10. By 6, 9 and definition of subset, \(B \subseteq F(F^{-1}(B))\).

        {\it Conclusion:}

    11. By 5, 10 and definition of set equality, \(F(F^{-1}(B)) = B\).
\end{proof}

{\bf \cy Let \(X = \{a, b, c, d, e\}\) and \(Y = \{s, t, u, v, w\}\). In each of 42 and 43 a one-to-one correspondence
\(F: X \to Y\) is defined by an arrow diagram. In each case draw an arrow diagram for \(F^{-1}\).}

\subsection{Exercise 42}
\begin{figure}[ht!]
    \centering
    \includegraphics[scale=0.8]{../images/7.2.42.png}
\end{figure}

\begin{proof}
    \begin{figure}[ht!]
        \centering
        \includegraphics[scale=0.5]{../images/7.2.42.sol.png}
    \end{figure}
\end{proof}

\subsection{Exercise 43}
\begin{figure}[ht!]
    \centering
    \includegraphics[scale=0.5]{../images/7.2.43.png}
\end{figure}

\begin{proof}
    \begin{figure}[ht!]
        \centering
        \includegraphics[scale=0.4]{../images/7.2.43.sol.png}
    \end{figure}
\end{proof}

{\bf \cy In $44-55$ indicate which of the functions in the referenced exercise are one-to-one correspondences. For
each function that is a one-to-one correspondence, find the inverse function.}

\subsection{Exercise 44}
Exercise 10a
\begin{proof}
    The function is not a one-to-one correspondence because it is not onto.
\end{proof}

\subsection{Exercise 45}
Exercise 10b
\begin{proof}
    The answer to exercise 10(b) shows that $h$ is onto. To show that $h$ is one-to-one, suppose \(h(n_1) = h(n_2)\).
    By definition of $h$, this implies that \(2n_1 = 2n_2\). Dividing both sides by 2 gives \(n_1 = n_2\). Hence $h$ is
    one-to-one, and so $h$ is a one-to-one correspondence. Given any even integer $m$, if \(m = h(n)\), then by
    definition of \(h, m = 2n\), and so \(n = m/2\). Thus \(h^{-1}(m) = m/2\) for every \(m \in 2\Z\).
\end{proof}

\subsection{Exercise 46}
Exercise 11a
\begin{proof}
    The function $g$ is not a one-to-one correspondence because it is not onto. For instance, if $m = 2$, it is impossible
    to find an integer $n$ such that \(g(n) = m\). (This is because if \(g(n) = m\), then \(4n - 5 = 2\), which implies
    that \(n = 7/4\). Thus the only number $n$ with the property that \(g(n) = m\) is 7/4. But 7/4 is not an integer.)
\end{proof}

\subsection{Exercise 47}
Exercise 11b
\begin{proof}
    The answer to exercise 11b shows that $G$ is onto. In addition, $G$ is one-to-one. To prove this, suppose
    \(G(x_1) = G(x_2)\) for some $x_1$ and $x_2$ in $\R$. {\it [We must show that \(x_1 = x_2\).]} By definition of $G$,
    \(4x_1 - 5 = 4x_2 - 5\). Add 5 to both sides of this equation and divide both sides by 4 to obtain \(x_1 = x_2\)
    {\it [as was to be shown]}. We claim that \(G^{-1}(y) = (y + 5)/4\) for each $y$ in $\R$. By definition of inverse
    function, this is true if, and only if, \(G((y + 5)/4) = y\). But \(G((y+5)/4) = 4((y + 5)/4)-5=(y + 5) - 5 = y\),
    and so it is the case that \(G^{-1}(y) = (y + 5)/4\) for each $y$ in $\R$.
\end{proof}

\subsection{Exercise 48}
Exercise 12a
\begin{proof}
    The function $F$ is not a one-to-one correspondence because it is not onto.
\end{proof}

\subsection{Exercise 49}
Exercise 12b
\begin{proof}
    The function $G$ is a one-to-one correspondence because it is both one-to-one and onto. As shown in the solution to
    exercise 12b, the inverse function is \(G^{-1}(y) = \frac{y-2}{-3}\).
\end{proof}

\subsection{Exercise 50}
Exercise 21
\begin{proof}
    The function $L$ is not a one-to-one correspondence because it is not 1-1.
\end{proof}

\subsection{Exercise 51}
Exercise 22
\begin{proof}
    The function $D$ is not a one-to-one correspondence because it is not 1-1.
\end{proof}

\subsection{Exercise 52}
Exercise 15 with the co-domain taken to be the set of all real numbers not equal to 1

\begin{proof}
    The answer to exercise 15 shows that $f$ is one-to-one, and if the co-domain is taken to be the set of all real numbers
    not equal to 1, then $f$ is also onto. The reason is that given any real number \(y \neq 1\), if we take
    \(x = \frac{1}{y-1}\), then $x$ is a real number and
    \[
        f(x) = f\left(\frac{1}{y-1}\right) = \frac{\frac{1}{y-1} + 1}{\frac{1}{y-1}} = \frac{1+(y-1)}{1} = y.
    \]
    Thus \(f^{-1}(y) = \frac{1}{y-1}\) for each real number \(y \neq 1\).
\end{proof}

\subsection{Exercise 53}
Exercise 16 with the co-domain taken to be the set of all real numbers

\begin{proof}
    The answer to exercise 16 shows that $f$ is not one-to-one. Therefore, it is not a one-to-one correspondence.
\end{proof}

\subsection{Exercise 54}
Exercise 17 with the co-domain taken to be the set of all real numbers not equal to 3

\begin{proof}
    The answer to exercise 17 shows that $f$ is 1-1. If the co-domain excludes 3, then it also becomes onto. Assume $y$ is
    any real number with \(y \neq 3\). Then letting \(y = \frac{3x-1}{x}\) and solving, we get \(x = \frac{1}{3-y}\)
    which is defined since \(y \neq 3\). Thus $f$ is a one-to-one correspondence and \(f^{-1}(y) = \frac{1}{3-y}\).
\end{proof}

\subsection{Exercise 55}
Exercise 18 with the co-domain taken to be the set of all real numbers not equal to 1

\begin{proof}
    The answer to exercise 18 shows that $f$ is 1-1. If the co-domain excludes 1, then it also becomes onto. Assume $y$ is
    any real number with \(y \neq 1\). Then letting \(y = \frac{x+1}{x-1}\) and solving, we get \(x=\frac{y+1}{y-1}\)
    which is defined since \(y \neq 1\). Thus $f$ is a one-to-one correspondence and \(f^{-1}(y) = \frac{y+1}{y-1}\).
\end{proof}

\subsection{Exercise 56}
In Example 7.2.8 a one-to-one correspondence was defined from the power set of $\{a, b\}$ to the set of all strings
of 0’s and 1’s that have length 2. Thus the elements of these two sets can be matched up exactly, and so the two
sets have the same number of elements.

\subsubsection{(a)}
Let \(X = \{x_1, x_2 , \ldots, x_n\}\) be a set with $n$ elements. Use Example 7.2.8 as a model to define a
one-to-one correspondence from \(\ps(X)\), the set of all subsets of $X$, to the set of all strings of 0’s and 1’s
that have length $n$.

\begin{proof}
    Let $S_n$ be the set of all strings of 0's and 1's that have length $n$.

    Define \(f : \ps(X) \to S_n\) as follows. For any subset $A$ of $X$, define $f(A)$ to be the string where, the $n$th
    character in the string is 0 if \(x_i \notin A\), and 1 if \(x_i \in A\).

    For example, if $n=5$ and \(A = \{x_2, x_4\}\) then \(f(A) = 01010\).
\end{proof}

\subsubsection{(b)}
In Section 9.2 we show that there are $2^n$ strings of 0’s and 1’s that have length $n$. What does this allow you to
conclude about the number of elements of $\ps(X)$? (This provides an alternative proof of Theorem 6.3.1.)

\begin{proof}
    Since \(f: \ps(X) \to S_n\) is a one-to-one correspondence, \(\ps(X)\) has the same number of elements as \(S_n\).
    Since \(S_n\) has $2^n$ elements, $\ps(X)$ also has $2^n$ elements.
\end{proof}

\subsection{Exercise 57}
Write a computer algorithm to check whether a function from one finite set to another is one-to-one. Assume the
existence of an independent algorithm to compute values of the function.

\begin{proof}
    Let a function $F$ be given and suppose the domain of $F$ is represented as a one-dimensional array
    \(a[1], a[2], \ldots, a[n]\).

    \begin{tabbing}
        \(answer \coloneqq\) ``one-to-one'' \\
        \(i \coloneqq 1\) \\
        {\bf while} \= (\(i \leq n-1\) and \(answer = \) ``one-to-one'') \\
        \> \(j \coloneqq i + 1\) \\
        \> {\bf while} \= (\(j \leq n\) and \(answer =\) ''one-to-one'') \\
        \>             \> {\bf if} (\(F(a[i]) = F(a[j])\) and \(a[i] \neq a[j]\)) \\
        \>             \> {\bf then} \(answer \coloneqq \) ``not one-to-one'' \\
        \>             \> \(j \coloneqq j + 1\) \\
        \> {\bf end while} \\
        \(i \coloneqq i + 1\) \\
        {\bf end while} \\
        {\bf return} $answer$
    \end{tabbing}
\end{proof}

\subsection{Exercise 58}
Write a computer algorithm to check whether a function from one finite set to another is onto. Assume the existence of
an independent algorithm to compute values of the function.

\begin{proof}
    Let a function $F$ be given and suppose the domain and the co-domain of $F$ are represented as one-dimensional arrays
    \(a[1], a[2], \ldots, a[n]\) and \(b[1], b[2], \ldots, b[m]\).

    \begin{tabbing}
        \(answer \coloneqq\) ``onto'' \\
        \(i \coloneqq 1\) \\
        {\bf while} \= (\(i\leq m\) and \(answer = \) ``onto'') \\
        \> \(found \coloneqq\) ``false'' \\
        \> \(j \coloneqq 1\) \\
        \> {\bf while} \= (\(j \leq n\) and \(found =\) ``false'') \\
        \>             \> {\bf if} (\(b[i]=F(a[j])\))\\
        \>             \> {\bf then} \(found \coloneqq \) ``true'' \\
        \>             \> \(j \coloneqq j + 1\) \\
        \> {\bf end while} \\
        \> {\bf if} \(found = \) false \\
        \> {\bf then} \(answer = \) ``not onto'' \\
        \(i \coloneqq i + 1\) \\
        {\bf end while} \\
        {\bf return} $answer$
    \end{tabbing}
\end{proof}

\section{Exercise Set 7.3}

 {\bf \cy In each of 1 and 2, functions $f$ and $g$ are defined by arrow diagrams. Find \(g \circ f\) and
  \(f \circ g\) and determine whether \(g \circ f\) equals \(f \circ g\).}

\subsection{Exercise 1}
\begin{figure}[ht!]
    \centering
    \includegraphics[scale=0.5]{../images/7.3.1.png}
\end{figure}

\begin{proof}
    \(g \circ f\) is defined as follows:
    \begin{center}
        \begin{tabular}{ccccccc}
            \((g \circ f)(1)\) & = & \(g(f(1))\) & = & \(g(5)\) & = & 1 \\
            \((g \circ f)(3)\) & = & \(g(f(3))\) & = & \(g(3)\) & = & 5 \\
            \((g \circ f)(5)\) & = & \(g(f(5))\) & = & \(g(1)\) & = & 3 \\
        \end{tabular}
    \end{center}

    \(f \circ g\) is defined as follows:
    \begin{center}
        \begin{tabular}{ccccccc}
            \((f \circ g)(1)\) & = & \(f(g(1))\) & = & \(f(3)\) & = & 3 \\
            \((f \circ g)(3)\) & = & \(f(g(3))\) & = & \(f(5)\) & = & 1 \\
            \((f \circ g)(5)\) & = & \(f(g(5))\) & = & \(f(1)\) & = & 5 \\
        \end{tabular}
    \end{center}

    Then \(g \circ f \neq f \circ g\) because, for example, \((g \circ f)(1) \neq (f \circ g)(1)\).
\end{proof}

\subsection{Exercise 2}
\begin{figure}[ht!]
    \centering
    \includegraphics[scale=0.5]{../images/7.3.2.png}
\end{figure}

\begin{proof}
    \(g \circ f\) is defined as follows:

    \begin{center}
        \begin{tabular}{ccccccc}
            \((g \circ f)(1)\) & = & \(g(f(1))\) & = & \(g(3)\) & = & 1 \\
            \((g \circ f)(3)\) & = & \(g(f(3))\) & = & \(g(1)\) & = & 1 \\
            \((g \circ f)(5)\) & = & \(g(f(5))\) & = & \(g(5)\) & = & 1 \\
        \end{tabular}
    \end{center}

    \(f \circ g\) is defined as follows:

    \begin{center}
        \begin{tabular}{ccccccc}
            \((f \circ g)(1)\) & = & \(f(g(1))\) & = & \(f(1)\) & = & 3 \\
            \((f \circ g)(3)\) & = & \(f(g(3))\) & = & \(f(1)\) & = & 3 \\
            \((f \circ g)(5)\) & = & \(f(g(5))\) & = & \(f(1)\) & = & 3 \\
        \end{tabular}
    \end{center}

    Then \(g \circ f \neq f \circ g\) because, for example, \((g \circ f)(1) \neq (f \circ g)(1)\).
\end{proof}

{\bf \cy In 3 and 4, functions $F$ and $G$ are defined by formulas. Find \(G \circ F\) and \(F \circ G\) and
determine whether \(G \circ F\) equals \(F \circ G\).}

\subsection{Exercise 3}
\(F(x) = x^3\) and \(G(x) = x - 1\), for each real number $x$.

\begin{proof}
    \((G \circ F)(x) = G(F(x)) = G(x^3) = x^3 - 1\) for every real number $x$.

    \((F \circ G)(x) = F(G(x)) = F(x - 1) = (x - 1)^3\) for every real number $x$.

    \(G \circ F \neq F \circ G\) because, for instance, \((G \circ F)(2) = 2^3 - 1 = 7\),
    whereas \((F \circ G)(2) = (2 - 1)^3 = 1\).
\end{proof}

\subsection{Exercise 4}
\(F(x) = x^5\) and \(G(x) = x^{1/5}\), for each real number $x$.

\begin{proof}
    \((G \circ F)(x) = G(F(x)) = G(x^5) = (x^5)^{1/5} = x\) for every real number $x$.

    \((F \circ G)(x) = F(G(x)) = F(x^{1/5}) = (x^{1/5})^5 = x\) for every real number $x$.

    So \(G \circ F = F \circ G\).
\end{proof}

\subsection{Exercise 5}
Define \(f: \R \to \R\) by the rule \(f(x) = -x\) for every real number $x$. Find \((f \circ f)(x)\).

\begin{proof}
    \((f \circ f)(x) = f(f(x)) = f(-x) = -(-x) = x\).
\end{proof}

\subsection{Exercise 6}
Define \(F: \Z \to \Z\) and \(G: \Z \to \Z\) by the rules \(F(a) = 7a\) and \(G(a) = a \mod 5\) for each integer $a$.
Find \((G \circ F)(0), (G \circ F)(1), (G \circ F)(2), (G \circ F)(3)\), and \((G \circ F)(4)\).

\begin{proof}
    \((G \circ F)(0) = G(F(0)) = G(7 \cdot 0) = 0 \mod 5 = 0\),

    \((G \circ F)(1) = G(F(1)) = G(7 \cdot 1) = 7 \mod 5 = 2\),

    \((G \circ F)(2) = G(F(2)) = G(7 \cdot 2) = 14 \mod 5 = 4\),

    \((G \circ F)(3) = G(F(3)) = G(7 \cdot 3) = 21 \mod 5 = 1\),

    \((G \circ F)(4) = G(F(4)) = G(7 \cdot 4) = 28 \mod 5 = 3\).
\end{proof}

\subsection{Exercise 7}
Define \(L: \Z \to \Z\) and \(M: \Z \to \Z\) by the rules \(L(a) = a^2\) and \(M(a) = a \mod 5\) for each integer $a$.

\subsubsection{(a)}
Find \((L \circ M)(12), (M \circ L)(12), (L \circ M)(9)\), and \((M \circ L)(9)\).

\begin{proof}
    \((L \circ M)(12) = L(M(12)) = L(12 \mod 5) = L(2) = 2^2 = 4\).

    \((M \circ L)(12) = M(L(12)) = M(12^2) = M(144) = 144 \mod 5 = 4\).

    \((L \circ M)(9) = L(M(9)) = L(9 \mod 5) = L(4) = 4^2 = 16\).

    \((M \circ L)(9) = M(L(9)) = M(9^2) = M(81) = 81 \mod 5 = 1\).
\end{proof}

\subsubsection{(b)}
Is \(L \circ M = M \circ L\)?

\begin{proof}
    No, because \((L \circ M)(9) \neq (M \circ L)(9)\).
\end{proof}

\subsection{Exercise 8}
Let $S$ be the set of all strings in $a$’s and $b$’s and let \(L: S \to \Z\) be the length function: For all strings
\(s \in S, L(s) =\) the number of characters in $s$. Let \(T: \Z \to \{0, 1, 2\}\) be the $\mod 3$ function: For
every integer \(n, T(n) = n \mod 3\).

\subsubsection{(a)}
\((T \circ L)(abaa) = ?\)

\begin{proof}
    \((T \circ L)(abaa) = T(L(abaa)) = T(4) = 4 \mod 3 = 1\)
\end{proof}

\subsubsection{(b)}
\((T \circ L)(baaab) = ?\)

\begin{proof}
    \((T \circ L)(baaab) = T(L(baaab)) = T(5) = 5 \mod 3 = 2\)
\end{proof}

\subsubsection{(c)}
\((T \circ L)(aaa) = ?\)

\begin{proof}
    \((T \circ L)(aaa) = T(L(aaa)) = T(3) = 3 \mod 3 = 0\)
\end{proof}

\subsection{Exercise 9}
Define \(F: \R \to \R\) and \(G: \R \to \Z\) by the following formulas: \(F(x) = x^2y^3\) and
\(G(x) = \floor{x}\) for every \(x \in \R\).

\subsubsection{(a)}
\((G \circ F)(2) = ?\)

\begin{proof}
    \((G \circ F)(2) = G(F(2)) = G(2^2/3) = G(4/3) = \floor{4/3} = 1\)
\end{proof}

\subsubsection{(b)}
\((G \circ F)(-3) = ?\)

\begin{proof}
    \((G \circ F)(-3) = G(F(-3)) = G((-3)^2/3) = G(9/3) = \floor{3} = 3\)
\end{proof}

\subsubsection{(c)}
\((G \circ F)(5) = ?\)

\begin{proof}
    \((G \circ F)(5) = G(F(5)) = G(5^2/3) = G(25/3) = \floor{25/3} = 8\)
\end{proof}

\subsection{Exercise 10}
Define \(F: \Z \to \Z\) and \(G: \Z \to \Z\) by the following formulas: \(F(n) = 2n\) and \(G(n)=\floor{n/2}\)
for every integer $n$.

\subsubsection{(a)}
Find \((G \circ F)(8)\), \((F \circ G)(8)\), \((G \circ F)(3)\) and \((F \circ G)(3)\).

\begin{proof}
    \((G \circ F)(8) = G(F(8)) = G(16) = \floor{16/2} = 8\)

    \((F \circ G)(8) = F(G(8)) = F(\floor{8/2}) = 2 \cdot 4 = 8\)

    \((G \circ F)(3) = G(F(3)) = G(6) = \floor{6/2} = 3\)

    \((F \circ G)(3) = F(G(3)) = F(\floor{3/2}) = 2 \cdot 1 = 2\)
\end{proof}

\subsubsection{(b)}
Is \(G \circ F = F \circ G\)? Explain.

\begin{proof}
    No, because \((G \circ F)(3) \neq (F \circ G)(3)\).
\end{proof}

\subsection{Exercise 11}
Define \(F: \R \to \R\) and \(G: \R \to \R\) by the rules
\(F(x) = 3x\) and \(G(x) = \floor{x/3}\) for every real number $x$.

\subsubsection{(a)}
Find \((G \circ F)(6)\), \((F \circ G)(6)\), \((G \circ F)(1)\) and \((F \circ G)(1)\).

\begin{proof}
    \((G \circ F)(6) = G(F(6)) = G(18) = \floor{18/3} = 6\)

    \((F \circ G)(6) = F(G(6)) = F(\floor{6/3}) = 3 \cdot 2 = 6\)

    \((G \circ F)(1) = G(F(1)) = G(3) = \floor{3/3} = 1\)

    \((F \circ G)(1) = F(G(1)) = F(\floor{1/3}) = 3 \cdot 0 = 0\)
\end{proof}

\subsubsection{(b)}
Is \(G \circ F = F \circ G\)? Explain.

\begin{proof}
    No, because \((G \circ F)(1) \neq (F \circ G)(1)\).
\end{proof}

{\bf \cy The functions of each pair in $12-14$ are inverse to each other. For each pair, check that both compositions
give the identity function.}

\subsection{Exercise 12}
\(F: \R \to \R\) and \(F^{-1}: \R \to \R\) are defined by
\[
    F(x) = 3x+2 \text{ and } F^{-1}(y) = \frac{y-2}{3}, \text{ for every } x, y \in \R.
\]
\begin{proof}
    \((F^{-1} \circ F)(x) = F^{-1}(F(x)) = F^{-1}(3x+2) = \frac{(3x+2)-2}{3} = x = I_{\R}(x)\) for every \(x \in\R\).
    Hence \(F^{-1} \circ F = I_{\R}\) by definition of equality of functions.

    \((F \circ F^{-1})(y) = F(F^{-1}(y)) = F(\frac{y-2}{3}) = 3 \cdot \frac{y-2}{3} + 2 = (y-2) + 2 = y = I_{\R}(y)\) for
    every \(y \in\R\). Hence \(F \circ F^{-1} = I_{\R}\) by definition of equality of functions.
\end{proof}

\subsection{Exercise 13}
\(G: \R^+ \to \R^+\) and \(G^{-1}: \R^+ \to \R^+\) are defined by
\[
    G(x) = x^2 \text{ and } G^{-1}(y) = \sqrt{y}, \text{ for every } x, y \in \R.
\]
\begin{proof}
    \((G^{-1} \circ G)(x) = G^{-1}(G(x)) = G^{-1}(x^2) = \sqrt{x^2} = x = I_{\R}(x)\) for every \(x \in \R\).
    Hence \(G^{-1} \circ G = I_{\R}\) by definition of equality of functions.

    \((G \circ G^{-1})(y) = G(G^{-1}(y)) = G(\sqrt{y}) = (\sqrt{y})^2 = y = I_{\R}(y)\) for every \(y \in \R\).
    Hence \(G \circ G^{-1} = I_{\R}\) by definition of equality of functions.
\end{proof}

\subsection{Exercise 14}
$H$ and $H^{-1}$ are both defined from $\R - \{1\}$ to $\R - \{1\}$ by the formula
\[
    H(x) = H^{-1}(x) = \frac{x+1}{x-1}, \text{ for every } x \in \R - \{1\}.
\]
\begin{proof}
    \[
        (H^{-1} \circ H)(x) = H^{-1}(H(x)) = H^{-1}\left(\frac{x+1}{x-1}\right)=\frac{\frac{x+1}{x-1}+1}{\frac{x+1}{x-1}-1} =
        \frac{\frac{2x}{x-1}}{\frac{2}{x-1}} = x = I_{\R}(x)
    \]
    for every \(x \in \R\). Hence \(H^{-1} \circ H = I_{\R}\) by definition of equality of functions. The calculation for
    \((H \circ H^{-1})(x)\) is the same as above. Hence \(H \circ H^{-1} = I_{\R}\) by definition of equality of functions.
\end{proof}

\subsection{Exercise 15}
Explain how it follows from the definition of logarithm that

\subsubsection{(a)}
\(\log_b(b^x) = x\), for every real number $x$.

\begin{proof}
    By definition of logarithm with base $b$, for each real number $x$, \(\log_b(b^x)\) is the exponent to which $b$
    must be raised to obtain $b^x$. But this exponent is just $x$. So \(\log_b(b^x) = x\).
\end{proof}

\subsubsection{(b)}
\(b^{\log_b x} = x\), for every positive real number $x$.

\begin{proof}
    By definition of logarithm with base $b$, for each real number $x$, \(\log_b(x)\) is the exponent to which $b$
    must be raised to obtain $x$. So \(b^{\log_b(x)} = x\).
\end{proof}

\subsection{Exercise 16}
Prove Theorem 7.3.1(b): If $f$ is any function from a set $X$ to a set $Y$, then \(I_Y \circ f = f\), where $I_Y$ is
the identity function on $Y$.

\begin{proof}
    Suppose $f$ is any function from a set $X$ to a set $Y$. {\it [We want to show \((I_Y \circ f)(x) = f(x)\)]} for
    every \(x \in X\). Indeed \((I_Y \circ f)(x) = I_Y(f(x)) = f(x)\) by definition of $I_Y$, {\it [as was to be shown.]}
\end{proof}

\subsection{Exercise 17}
Prove Theorem 7.3.2(b): If \(f: X \to Y\) is a one-to-one and onto function with inverse function \(f^{-1}:Y \to X\),
then \(f \circ f^{-1} = I_Y\), where \(I_Y\) is the identity function on $Y$.

\begin{proof}
    {\it [We want to show that for every \(y \in Y, I_Y(y) = (f \circ f^{-1})(y)\)].} Assume \(y \in Y\). Since $f$ is 1-1
    and onto, there is a unique \(x \in X\) such that \(f(x) = y\), and therefore, \(f^{-1}(y) = x\). Then
    \[
        (f \circ f^{-1})(y) = f(f^{-1}(y)) = f(x) = y = I_Y(y),
    \]
    {\it [as was to be shown.]}
\end{proof}

\subsection{Exercise 18}
Suppose $Y$ and $Z$ are sets and \(g: Y \to Z\) is a one-to-one function. This means that if $g$ takes the same
value on any two elements of $Y$, then those elements are equal. Thus, for example, if $a$ and $b$ are elements of
$Y$ and \(g(a) = g(b)\), then it can be inferred that $a = b$. What can be inferred in the following situations?

\subsubsection{(a)}
$s_k$ and $s_m$ are elements of $Y$ and \(g(s_k) = g(s_m)\).

\begin{proof}
    We can infer that \(s_k = s_m\).
\end{proof}

\subsubsection{(b)}
$z/2$ and $t/2$ are elements of $Y$ and \(g(z/2) = g(t/2)\).

\begin{proof}
    We can infer that \(z/2 = t/2\).
\end{proof}

\subsubsection{(c)}
$f(x_1)$ and $f(x_2)$ are elements of $Y$ and \(g(f(x_1)) = g(f(x_2))\).

\begin{proof}
    We can infer that \(f(x_1) = f(x_2)\).
\end{proof}

\subsection{Exercise 19}
If \(f: X \to Y\) and \(g: Y \to Z\) are functions and \(g \circ f\) is one-to-one, must $g$ be one-to-one?
Prove or give a counterexample.

\begin{proof}
    The answer is no. \underline{Counterexample:} Define $f$ and $g$ by the arrow diagrams below.

    \begin{figure}[ht!]
        \centering
        \includegraphics[scale=0.4]{../images/7.3.19.png}
    \end{figure}

    Then \(g \circ f\) is one-to-one but $g$ is not one-to-one.
    (This is one counterexample among many. Can you construct a different one?)
\end{proof}

\subsection{Exercise 20}
If \(f: X \to Y\) and \(g: Y \to Z\) are functions and \(g \circ f\) is onto, must $f$ be onto?
Prove or give a counterexample.

\begin{proof}
    The answer is no. \underline{Counterexample:} Define $f$ and $g$ by the arrow diagrams below.

    \begin{figure}[ht!]
        \centering
        \includegraphics[scale=0.4]{../images/7.3.20.png}
    \end{figure}

    Then \(g \circ f\) is onto, because \((g \circ f)(x) = z\). But $f$ is not onto, because there is no \(e \in X\) such
    that \(f(e) = y_2\).
\end{proof}

\subsection{Exercise 21}
If \(f: X \to Y\) and \(g: Y \to Z\) are functions and \(g \circ f\) is one-to-one, must $f$ be one-to-one?
Prove or give a counterexample.

\begin{proof}
    Yes. Suppose \(f: X \to Y\) and \(g: Y \to Z\) are functions and \(g \circ f\) is one-to-one. Assume
    \(x_1, x_2 \in X\) and \(f(x_1) = f(x_2)\). {\it [We want to show \(x_1 = x_2\).]} Since \(f(x_1) = f(x_2)\), apply
    $g$ to both sides to get \(g(f(x_1)) = g(f(x_2))\), so by definition of \(g \circ f\), \((g \circ f)(x_1) =
    (g \circ f)(x_2)\). Since \(g \circ f\) is 1-1, \(x_1 = x_2\), {\it [as was to be shown.]}

\end{proof}

\subsection{Exercise 22}
If \(f: X \to Y\) and \(g: Y \to Z\) are functions and \(g \circ f\) is onto, must $g$ be onto?
Prove or give a counterexample.

\begin{proof}
    Suppose \(f: X \to Y\) and \(g: Y \to Z\) are functions and \(g \circ f\) is onto. Assume \(z \in Z\). {\it [We want
                to show there exists \(y \in Y\) such that \(g(y) = z\).]} Since \(g \circ f\) is onto, there exists \(x \in X\) such
    that \((g \circ f)(x) = z\). So \(g(f(x)) = z\). Let $y = f(x)\). Then \(y \in Y\) and \(g(y) = z\),
    {\it [as was to be shown.]}
\end{proof}

\subsection{Exercise 23}
Let \(f: W \to X, g: X \to Y\), and \(h: Y \to Z\) be functions.
Must \(h \circ (g \circ f) = (h \circ g) \circ f\)?
Prove or give a counterexample.

\begin{proof}
    True. {\it [We want to show \((h \circ (g \circ f))(w) = ((h \circ g) \circ f)(w)\) for all \(w \in W\).]} Assume
    \(w \in W\). Then by definition of \(h \circ (g \circ f)\)
    \[
        (h \circ (g \circ f))(w) = h((g \circ f)(w)) = h(g(f(w)))
    \]
    and by definition of \((h \circ g) \circ f\)
    \[
        ((h \circ g) \circ f)(w) = (h \circ g)(f(w)) = h(g(f(w)))
    \]
    thus \((h \circ (g \circ f))(w) = ((h \circ g) \circ f)(w)\) {\it [as was to be shown.]}
\end{proof}

\subsection{Exercise 24}
True or False? Given any set $X$ and given any functions \(f: X \to X, g: X \to X\), and \(h: X \to X\), if $h$ is
one-to-one and \(h \circ f = h \circ g\), then $f = g$. Justify your answer.

\begin{proof}
    True. Suppose $X$ is any set and $f$, $g$, and $h$ are functions from $X$ to $X$ such that $h$ is one-to-one and
    \(h \circ f = h \circ g\). {\it [We must show that for every $x$ in $X$, \(f(x) = g(x)\).]} Suppose $x$ is any
    element in $X$. Because \(h \circ f = h \circ g\), we have that \((h \circ f)(x) = (h \circ g)(x)\) by definition of
    equality of functions. Then, by definition of composition of functions, \(h(f(x)) = h(g(x))\). And since $h$ is one-
    to-one, this implies that \(f(x) = g(x)\) {\it [as was to be shown].}
\end{proof}

\subsection{Exercise 25}
True or False? Given any set $X$ and given any functions \(f: X \to X, g: X \to X\), and \(h: X \to X\), if $h$ is
one-to-one and \(f \circ h = g \circ h\), then $f = g$. Justify your answer.

\begin{proof}
    Interestingly, if $X$ is a finite set then this statement is true, because if $h: X \to X$ is 1-1, and $X$ is finite,
    then $h$ is also onto, but this is not necessarily true if $X$ is infinite.

    False. \underline{Counterexample:} Let \(X = \Z\), and for all \(n \in \Z, h(n) = 2n\),
    \begin{center}
        \(
        f(n) =
        \left\{
        \begin{tabular}{ll}
            \(n\) & if $n$ is even \\
            \(5\) & if $n$ is odd
        \end{tabular}
        \right.
        \), and
        \(
        g(n) =
        \left\{
        \begin{tabular}{ll}
            \(n\) & if $n$ is even \\
            \(3\) & if $n$ is odd
        \end{tabular}
        \right.
        \).
    \end{center}
    Then $h$ is one-to-one: if \(h(n_1) = h(n_2)\) then \(2n_1 = 2n_2\) therefore \(n_1 = n_2\).

    Also \(f \circ h = g \circ h\): for all \(n \in \Z\) we have \((f \circ h)(n) = f(h(n)) = f(2n) = 2n\) (since $2n$
    is even), and similarly \((g \circ h)(n) = g(h(n)) = g(2n) = 2n\), so \((f \circ h)(n) = (g \circ h)(n)\).

    But $f \neq g$, because \(f(1) = 5 \neq 3 = g(1)\).
\end{proof}

{\bf \cy In 26 and 27 find \((g \circ f)^{-1}, g^{-1}, f^{-1}\), and \(f^{-1} \circ g^{-1}\), and state how
\((g \circ f)^{-1}\) and \(f^{-1} \circ g^{-1}\) are related.}

\subsection{Exercise 26}
Let \(X = \{a, b, c\}, Y = \{x, y, z\}\), and \(Z = \{u, v, w\}\). Define \(f: X \to Y\) and \(g: Y \to Z\) by the
arrow diagrams below.

\begin{figure}[ht!]
    \centering
    \includegraphics[scale=0.5]{../images/7.3.26.png}
\end{figure}

\begin{proof}
    \begin{figure}[ht!]
        \centering
        \includegraphics[scale=0.4]{../images/7.3.26.sol.png}
    \end{figure}

    The functions \((g \circ f)^{-1}\) and \(f^{-1} \circ g^{-1}\) are equal.
\end{proof}

\subsection{Exercise 27}
Define \(f: \R \to \R\) and \(g: \R \to \R\) by the formulas \(f(x) = x + 3\) and \(g(x) = -x\) for each \(x \in R\).

\begin{proof}
    \((g \circ f)(x) = g(f(x)) = g(x+3) = -x-3, (g \circ f)^{-1}(y) = -y-3\)

    \(g^{-1}(y) = -y, f^{-1}(y) = y - 3\)

    \((f^{-1} \circ g^{-1})(y) = f^{-1}(g^{-1}(y)) = f^{-1}(-y) = -y-3\)

    The functions \((g \circ f)^{-1}\) and \(f^{-1} \circ g^{-1}\) are equal.
\end{proof}

\subsection{Exercise 28}
Prove or give a counterexample: If \(f: X \to Y\) and \(g: Y \to X\) are functions such that \(g \circ f = I_X\) and
\(f \circ g = I_Y\), then $f$ and $g$ are both one-to-one and onto and \(g = f^{-1}\).

\begin{proof}
    {\bf $\bm{f}$ is 1-1:} Assume \(f(x_1) = f(x_2)\) for some \(x_1, x_2 \in X\). {\it [Want to show \(x_1 = x_2\).]}
    Since \(f(x_1) = f(x_2)\), we have \(g(f(x_1))=g(f(x_2))\). So \((g \circ f)(x_1) = (g \circ f)(x_2)\).
    So \(I_X(x_1) = I_X(x_2)\), therefore \(x_1 = x_2\). {\it [as was to be shown.]}

    In a parallel way, we can prove $g$ is 1-1.

    {\bf $\bm{f}$ is onto:} Assume \(y \in Y\). {\it [Want to show there exists \(x \in X\) such that \(y = f(x)\).]}
    Let \(x = g(y)\). Then \(f(x) = f(g(y)) = (f \circ g)(y) = I_Y(y) = y\), {\it [as was to be shown.]}

    In a parallel way, we can prove $g$ is onto.
\end{proof}

\subsection{Exercise 29}
Suppose \(f: X \to Y\) and \(g: Y \to Z\) are both one-to-one and onto. Prove that \((g \circ f)^{-1}\) exists and
that \((g \circ f)^{-1} = f^{-1} \circ g^{-1}\).

\begin{proof}
    By Theorem 7.3.3 \(g \circ f\) is 1-1. By Theorem 7.3.4 \(g \circ f\) is onto. By Theorem 7.2.2 \((g \circ f)^{-1}\)
    exists and has the inverse function property: for all \(x \in X, z \in Z, (g \circ f)^{-1}(z) = x\) if and only if
    \((g \circ f)(x) = z\).

        {\it [We want to show that for all \(z \in Z, (g \circ f)^{-1}(z) = (f^{-1} \circ g^{-1})(z)\).]} Assume $z\in Z$.
    Let \(x = (g \circ f)^{-1}(z)\). By the inverse function property of \((g \circ f)^{-1}\), \((g \circ f)(x) = z\).
    So \(g(f(x)) = z\). By the inverse function property of $g^{-1}$, \(g^{-1}(z) = f(x)\). Then by the inverse
    function property of $f^{-1}$, \(f^{-1}(g^{-1}(z)) = x\).
    So \((g \circ f)^{-1}(z) = (f^{-1} \circ g^{-1})(z)\), {\it [as was to be shown.]}
\end{proof}

\subsection{Exercise 30}
Let \(f: X \to Y\) and \(g: Y \to Z\). Is the following property true or false? For every subset $C$ in $Z$,
\((g \circ f)^{-1}(C) = f^{-1}(g^{-1}(C))\). Justify your answer.

\begin{proof}
    1. By definition of inverse image, \(g^{-1}(C) = \{y \in Y \, | \, g(y) \in C\}\).

    2. By 1 and definition of inverse image,
    \[
        f^{-1}(g^{-1}(C)) = \{x \in X \, | \, f(x) \in g^{-1}(C)\} = \{x \in X \, | \, g(f(x)) \in C\}.
    \]
    3. By definition of inverse image and $\circ$,
    \[
        (g \circ f)^{-1}(C) = \{x \in X \, | \, (g \circ f)(x) \in C\} = \{x \in X \, | \, g(f(x)) \in C\}.
    \]
    4. By 2 and 3, \(f^{-1}(g^{-1}(C)) = (g \circ f)^{-1}(C)\).
\end{proof}

\section{Exercise Set 7.4}

\subsection{Exercise 1}
When asked what it means to say that set $A$ has the same cardinality as set $B$, a student replies, “$A$ and $B$ are
one-to-one and onto.” What should the student have replied? Why?

\begin{proof}
    The student should have replied that for $A$ to have the same cardinality as $B$ means that there is a function from
    $A$ to $B$ that is one-to-one and onto. A set cannot have the property of being onto or one-to-one another set; only
    a function can have these properties.
\end{proof}

\subsection{Exercise 2}
Show that “there are as many squares as there are numbers” by exhibiting a one-to-one correspondence from the positive
integers, $\Z^+$, to the set $S$ of all squares of positive integers: \(S = \{n \in \Z^+ \, | \, n = k^2,
\text{ for some positive integer } k\}\).

\begin{proof}
    Define a function \(f: \Z^+ \to S\) as follows: For every positive integer \(k, f(k) = k^2\).

        {\it $f$ is one-to-one:} {\it [We must show that for all $k_1$ and \(k_2 \in \Z^+\), if \(f(k_1) = f(k_2)\) then
                \(k_1 = k_2\).]} Suppose $k_1$ and $k_2$ are positive integers and \(f(k_1) = f(k_2)\). By definition of \(f, (k_1)^2 = (k_2)^2\), so \(k_1 = \pm k_2\). But $k_1$ and $k_2$ are positive. Hence \(k_1 = k_2\).

        {\it $f$ is onto:} {\it [We must show that for each \(n \in S\), there exists \(k \in \Z^+\) such that \(n = f(k)\).]}
    Suppose \(n \in S\). By definition of \(S, n = k^2\) for some positive integer $k$. Then by definition of \(f, n = f(k)\).

    Since there is a one-to-one, onto function (namely, $f$) from $\Z^+$ to $S$, the two sets have the same cardinality.
\end{proof}

\subsection{Exercise 3}
Let \(3\Z = \{n \in \Z \,|\, n = 3k, \text{ for some integer } k\}\). Prove that $\Z$ and $3\Z$ have the same
cardinality.

\begin{proof}
    Define \(f: \Z \to 3\Z\) by the rule \(f(n) = 3n\) for each integer $n$. The function $f$ is one-to-one because for any
    integers $n_1$ and $n_2$, if \(f(n_1) = f(n_2)\) then \(3n_1 = 3n_2\) and so \(n_1 = n_2\). Also $f$ is onto
    because if $m$ is any element in $3\Z$, then \(m = 3k\) for some integer $k$. Then \(f(k) = 3k = m\) by definition of
    $f$. Thus, since there is a function \(f: \Z \to 3\Z\) that is one-to-one and onto, $\Z$ has the same cardinality as $3\Z$.
\end{proof}

\subsection{Exercise 4}
Let {\bf O} be the set of all odd integers. Prove that {\bf O} has the same cardinality as $2\Z$, the set of all even
integers.

\begin{proof}
    Define \(f: {\bf O} \to 2\Z\) with \(f(n) = n-1\). (Notice that $f$ is well-defined because, subtracting 1 from an odd
    integer always gives an even integer.) $f$ is 1-1 because, if \(n_1 - 1 = n_2 - 1\) then \(n_1 = n_2\). $f$ is onto
    because, given any even integer \(2n \in 2\Z\) there is an odd integer \(o = 2n+1\) such that
    \(f(o) = f(2n+1) = 2n+1-1 = 2n\).
\end{proof}

\subsection{Exercise 5}
Let $25\Z$ be the set of all integers that are multiples of 25. Prove that $25\Z$ has the same cardinality as $2\Z$,
the set of all even integers.

\begin{proof}
    Define \(f: 25\Z \to 2\Z\) as follows: given any \(t \in 25\Z, t = 25n\) for some integer $n$. Then define \(f(t) =
    2n\). In other words, \(f(t) = \frac{2}{25} \cdot t\). (Notice $f$ is well-defined because \(\frac{2}{25} \cdot
    t\) is always an integer for all \(t \in 25\Z\)). $f$ is 1-1 because, if \(\frac{2}{25}t_1 = \frac{2}{25}t_2\) then
    by canceling $\frac{2}{25}$ we get \(t_1 = t_2\). $f$ is onto because, given any even integer \(2n \in 2\Z\) there
    exists \(t \in 25\Z\), namely \(t = \frac{25}{2} \cdot (2n) = 25n\) such that \(f(t) = f(25n) = \frac{2}{25}(25n) = 2n\).
\end{proof}

\subsection{Exercise 6}
Use the functions $I$ and $J$ defined in the paragraph following Example 7.4.1 to show that even though there is a
one-to-one correspondence, $H$, from $2\Z$ to $\Z$, there is also a function from $2\Z$ to $Z$ that is one-to-one but
not onto and a function from $Z$ to $2\Z$ that is onto but not one-to-one. In other words, show that $I$ is one-to-one
but not onto, and show that $J$ is onto but not one-to-one.

\begin{proof}
    Recall \(I: 2\Z \to \Z, I(n) = n\) and \(J: \Z \to 2\Z, J(n) = 2\floor{n/2}\) for each \(n \in \Z\).

        {\it I is 1-1:} Assume \(I(n_1) = I(n_2)\). Then by definition of \(I, n_1 = n_2\).

        {\it I is not onto:} Let \(n = 3\). There is no \(m \in 2\Z\) such that \(I(m) = n\). Why? Argue by contradiction
    and assume otherwise. Then \(I(m) = m = n\) so \(m = 3\). But \(3 \notin 2\Z\), contradiction.

        {\it J is not 1-1:} because \(J(2) = 2\floor{2/2} = 2 \cdot 1 = 2\) and \(J(3) = 2\floor{3/2} = 2 \cdot 1 = 2\). So
    \(J(2) = J(3)\) but \(2 \neq 3\).

        {\it J is onto:} Let \(m \in 2\Z\) be any even integer. So \(m = 2n\) for some integer $n$. Then there exists an
    integer $k \in \Z$, namely $k = m$, such that \(J(k) = J(m) = J(2n) = 2\floor{(2n)/2} = 2\floor{n} = 2n = m\).
\end{proof}

\subsection{Exercise 7}
\subsubsection{(a)}
Check that the formula for $F$ given at the end of Example 7.4.2 produces the correct values for \(n = 1, 2, 3, 4\).

\begin{proof}
    Recall that the formula is:
    \[
        F(n) =
        \left\{
        \begin{tabular}{ll}
            \(\frac{n}{2}\)    & if $n$ is positive and even \\
            \(-\frac{n-1}{2}\) & if $n$ is positive and odd
        \end{tabular}
        \right.
    \]
    and it is supposed to give us the values \(F(1) = 0, F(2) = 1, F(3) = -1, F(4) = 2\).

    Let's check: 1 is odd, so \(F(1) = -\frac{1-1}{2} = 0\). 2 is even, so \(F(2) = \frac{2}{2} = 1\). 3 is odd, so
    \(F(3) = -\frac{3-1}{2} = -1\). 4 is even, so \(F(4) = \frac{4}{2} = 2\).
\end{proof}

\subsubsection{(b)}
Use the floor function to write a formula for $F$ as a single algebraic expression for each positive integer $n$.

\begin{proof}
    For each positive integer \(\dps n, F(n) = (-1)^n \floor{ \frac{n}{2}}\).
\end{proof}

\subsection{Exercise 8}
Use the result of exercise 3 to prove that $3\Z$ is countable.

\begin{proof}
    It was shown in Example 7.4.2 that $\Z$ is countably infinite, which means that $\Z^+$ has the same cardinality
    as $\Z$. By exercise 3, $\Z$ has the same cardinality as $3\Z$. It follows by the transitive property of cardinality
    (Theorem 7.4.1 (c)) that $\Z^+$ has the same cardinality as $3\Z$. Thus $3\Z$ is countably infinite {\it [by definition
                of countably infinite]}, and hence $3\Z$ is countable {\it [by definition of countable].}
\end{proof}

\subsection{Exercise 9}
Show that the set of all nonnegative integers is countable by exhibiting a one-to-one correspondence between $\Z^+$
and \(\Z^{nonneg}\).

\begin{proof}
    Define \(f: \Z^+ \to \Z^{nonneg}\) by \(f(n) = n-1\). $f$ is 1-1 because if \(n_1 - 1 = n_2 - 1\) then \(n_1 = n_2\).
    $f$ is onto because, for every \(n \in \Z^{nonneg}\) there is \(m \in \Z^+\), namely \(m = n+1\) such that
    \(f(m) = f(n+1) = (n+1)-1 = n\).
\end{proof}

{\bf \cy In $10-14$ $S$ denotes the set of real numbers strictly between 0 and 1. That is, \(S = \{x \in \R \, | \,
0 < x < 1\}\).}

\subsection{Exercise 10}
Let \(U = \{x \in \R \,|\, 0 < x < 2\}\). Prove that $S$ and $U$ have the same cardinality.

\begin{proof}
    Define \(f: S \to U\) by the rule \(f(x) = 2x\) for each real number $x$ in $S$. Then $f$ is one-to-one by the same
    argument as in exercise 10a of Section 7.2 with $\R$ in place of $\Z$. Furthermore, $f$ is onto because if $y$ is
    any element in $U$, then \(0 < y < 2\) and so \(0 < y/2 < 1\). Consequently, \(y/2 \in S\) and \(f(y/2)=2(y/2) = y\).
    Hence $f$ is a one-to-one correspondence, and so $S$ and $U$ have the same cardinality.
\end{proof}

\subsection{Exercise 11}
Let \(V = \{x \in \R \,|\, 2 < x < 5\}\). Prove that $S$ and $V$ have the same cardinality.

\begin{proof}
    Define \(h: S \to V\) as follows: \(h(x) = 3x + 2\), for every \(x \in S\). $h$ is 1-1 because if \(3x_1 + 2 = 3x_2
    + 2\) then \(3x_1 = 3x_2\) and \(x_1 = x_2\). $h$ is onto because, given any \(y \in V\), let \(x = (y-2)/3\). Since
    \(2 < y < 5\) we have \((2-2)/3 < x < (5-2)/3\), or \(0 < x < 1\) so \(x \in S\), and \(h(x) = h((y-2)/3) = 3(y-2)/3+2
    = (y-2)+2 = y\). Hence $h$ is a one-to-one correspondence, and so $S$ and $V$ have the same cardinality.
\end{proof}

\subsection{Exercise 12}
Let $a$ and $b$ be real numbers with $a < b$, and suppose that \(W = \{x \in \R \,|\, a < x < b\}\). Prove that $S$
and $W$ have the same cardinality.

\begin{proof}
    Define \(f: S \to W\) as follows: \(f(x) = (b-a)x + a\), for every \(x \in S\). $f$ is 1-1 because if \((b-a)x_1 + a
    = (b-a)x_2 + a\) then \((b-a)x_1 = (b-a)x_2\) and since \(b-a>0\) we can divide by \(b-a\) to get \(x_1 = x_2\). $f$
    is onto because, given any \(y \in W\), let \(x = (y-a)/(b-a)\). Since \(a < y < b\) we have \((a-a)/(b-a) < x < (b-
    a)/(b-a)\), or \(0 < x < 1\); so \(x \in S\), and \(f(x) = f((y-a)/(b-a))=(b-a)\cdot\frac{y-a}{b-a}+a = (y-a)+a = y\).
    Hence $f$ is a one-to-one correspondence, and so $S$ and $W$ have the same cardinality.
\end{proof}

\subsection{Exercise 13}
Draw the graph of the function $f$ defined by the following formula: For each real number $x$ with \(0 < x < 1\),
\(f(x) = \tan(\pi x - \frac{\pi}{2})\). Use the graph to explain why $S$ and $R$ have the same cardinality.

\begin{proof}
    It is clear from the graph that $f$ is one-to-one (since it is increasing) and that the image of $f$ is all of $\R$
    (since the lines $x = 0$ and $x = 1$ are vertical asymptotes). Thus $S$ and $\R$ have the same cardinality.

    \begin{figure}[ht!]
        \centering
        \includegraphics[scale=0.4]{../images/7.4.13.png}
    \end{figure}
\end{proof}

\subsection{Exercise 14}
Define a function $g$ from the set of real numbers to $S$ by the following formula: For each real number $x$,
\[
    g(x) = \frac{1}{2} \cdot \frac{x}{1 + |x|} + \frac{1}{2}.
\]
Prove that $g$ is a one-to-one correspondence. (It is possible to prove this statement either with calculus or
without it.) What conclusion can you draw from this fact?

\begin{proof}
    {\bf $g$ is onto:} Given \(y \in S\) (so \(0 < y < 1\)) let
    \[
        x =
        \left\{
        \begin{tabular}{ll}
            \(\frac{1}{2} \cdot \frac{1}{-y} + 1\)  & if \(0 < y \leq 1/2\) \\
            \(\frac{1}{2} \cdot \frac{1}{1-y} - 1\) & if \(1/2 < y < 1\)
        \end{tabular}
        \right.
    \]
    We claim that \(g(x) = y\).

        {\bf Case 1 (\(\bm{0 < y \leq 1/2})\):} Notice that \(-y \geq -1/2\), so \(\frac{1}{-y} \leq -2\), so \(\frac{1}{2}
    \cdot \frac{1}{-y} \leq -1\), so \(\frac{1}{2} \cdot \frac{1}{-y} + 1 \leq 0\). Thus \(\left|\frac{1}{2} \cdot
    \frac{1}{-y} + 1 \right| = \frac{1}{2} \cdot \frac{1}{y} - 1\). Then $g(x)$
    \begin{center}
        \begin{tabular}{llllll}
            =                                                                                       & \(\dps g\left(\frac{1}{2} \cdot \frac{1}{-y} + 1\right)\)                                                       & =                        & \(\dps \frac{1}{2} \cdot \frac{\frac{1}{2} \cdot
            \frac{1}{-y} + 1}{1 + \left|\frac{1}{2} \cdot \frac{1}{-y} + 1 \right|} + \frac{1}{2}\) & =                                                                                                               & \(\dps \frac{1}{2} \cdot
            \frac{\frac{1}{2} \cdot \frac{1}{-y} + 1}{1 + \frac{1}{2} \cdot \frac{1}{y} - 1} + \frac{1}{2}\)                                                                                                                                                                                        \\
            =                                                                                       & \(\dps \frac{1}{2} \cdot \frac{\frac{1}{2}\cdot\frac{1}{-y} + 1}{\frac{1}{2} \cdot \frac{1}{y}} + \frac{1}{2}\) &
            =                                                                                       & \(\dps y \cdot \left(\frac{1}{2}\cdot\frac{1}{-y} + 1\right) + \frac{1}{2}\)                                    & =                        & \(\dps -\frac{1}{2} + y +
            \frac{1}{2} = y\)
        \end{tabular}
    \end{center}

    {\bf Case 2 (\(\bm{1/2 < y < 1})\):} Notice that \(-1/2 > -y\), so \(1/2 > 1-y\), so \(2 < \frac{1}{1-y}\), so \(1 <
    \frac{1}{2} \cdot \frac{1}{1-y}\), so \(0 < \frac{1}{2} \cdot \frac{1}{1-y} - 1\). Thus \(\left|\frac{1}{2} \cdot
    \frac{1}{1-y} - 1 \right| = \frac{1}{2} \cdot \frac{1}{1-y} - 1\). Then $g(x)$
    \begin{center}
        \begin{tabular}{llllll}
            =                   & \(\dps g\left(\frac{1}{2} \cdot \frac{1}{1-y} - 1 \right)\)                       & =                                          & \(\dps \frac{1}{2} \cdot \frac{\frac{1}{2} \cdot \frac{1}{1-y} - 1}{1 + \left|\frac{1}{2} \cdot \frac{1}{1-y} - 1 \right|} + \frac{1}{2}\) & = & \(\dps \frac{1}{2} \cdot \frac{\frac{1}{2} \cdot \frac{1}{1-y} - 1}{1 + \frac{1}{2} \cdot \frac{1}{1-y} - 1} + \frac{1}{2}\) \\
            =                   & \(\dps \frac{1}{2} \cdot \frac{\frac{2y-1}{2-2y}}{\frac{1}{2-2y}} + \frac{1}{2}\) & =                                          & \(\dps \frac{2y-1}
            {2} + \frac{1}{2}\) & =                                                                                 & \(\dps y - \frac{1}{2} + \frac{1}{2} = y\)
        \end{tabular}
    \end{center}

    {\bf $g$ is 1-1:} Assume \(x_1, x_2 \in \R\) and \(g(x_1) = g(x_2)\). {\it [We want to show \(x_1 = x_2\)].} So
    \[
        \frac{1}{2} \cdot \frac{x_1}{1 + |x_1|} + \frac{1}{2} = \frac{1}{2} \cdot \frac{x_2}{1 + |x_2|} + \frac{1}{2}
    \]
    Subtracting 1/2 and multiplying by 2 we get \(\dps \frac{x_1}{1 + |x_1|} = \frac{x_2}{1 + |x_2|}\). Cross
    multiplying we get \(x_1(1+|x_2|) = x_2(1+|x_1|)\), so (*) \(x_1 + x_1|x_2| = x_2 + x_2|x_1|\). There are 4 cases:

    {\bf Case 1 (\(\bm{x_1 \geq 0, x_2 \geq 0}\)):} In this case \(|x_1| = x_1\) and \(|x_2| = x_2\). So (*) becomes
    \(x_1 + x_1x_2 = x_2 + x_2x_1\), after canceling we get \(x_1 = x_2\).

        {\bf Case 2 (\(\bm{x_1 \geq 0, x_2 \leq 0}\)):} In this case \(|x_1| = x_1\) and \(|x_2| = -x_2\). So (*) becomes
    \(x_1 - x_1x_2 = x_2 + x_2x_1\), solving for $x_1$ we get \(x_1 - 2x_1x_2 = x_2 \implies x_1(1-2x_2) = x_2\) so
    \(\dps x_1 = \frac{x_2}{1-2x_2}\). The left hand side is \(\geq 0\) while the right hand side is \(\leq 0\), which
    forces \(x_1 = x_2 = 0\). So \(x_1 = x_2\).

        {\bf Case 3 (\(\bm{x_1 \leq 0, x_2 \geq 0}\)):} In this case \(|x_1| = -x_1\) and \(|x_2| = x_2\). So (*) becomes
    \(x_1 + x_1x_2 = x_2 - x_2x_1\), solving for $x_2$ we get \(x_1 = x_2 - 2x_1x_2 \implies x_1 = x_2(1-2x_1)\) so
    \(\dps x_2 = \frac{x_1}{1-2x_1}\). The left hand side is \(\geq 0\) while the right hand side is \(\leq 0\), which
    forces \(x_1 = x_2 = 0\). So \(x_1 = x_2\).

        {\bf Case 4 (\(\bm{x_1 \leq 0, x_2 \leq 0}\)):} In this case \(|x_1| = -x_1\) and \(|x_2| = -x_2\). So (*) becomes
    \(x_1 - x_1x_2 = x_2 - x_2x_1\), after canceling we get \(x_1 = x_2\).

        {\bf Conclusion:} $S$ has the same cardinality as the set of real numbers.
\end{proof}

\subsection{Exercise 15}
Show that the set of all bit strings (strings of 0’s and 1’s) is countable.

\begin{proof}
    We can describe a one-to-one correspondence between this set and $\Z^+$ as follows: first consider the bit strings
    of length 0, namely the empty string. Map it to 1. Then consider the bit strings of length 1, namely 0 and 1. Map
    them to 2 and 3. Then consider the bit strings of length 2, namely 00, 01, 10 and 11. Map them to 4,5,6,7. And so on.
    Generally, for each integer \(n \geq 0\) there are \(2^n\) bit strings of length $n$, and we map them to the positive
    integers between \(2^n\) (inclusive) and \(2^{n+1}-1\) (inclusive). This mapping is a one-to-one correspondence by
    definition, since we are never mapping two different bit strings to the same positive integer (so it's 1-1) and we
    are not skipping over any positive integer in the range (so it's onto).
\end{proof}

\subsection{Exercise 16}
Show that $\Q$, the set of all rational numbers, is countable.

\begin{proof}
    In Example 7.4.4 it was shown that there is a one-to-one correspondence from $\Z^+$ to $\Q^+$. This implies that the
    positive rational numbers can be written as an infinite sequence: \(r_1, r_2, r_3, r_4, \ldots\). Now the set $\Q$
    of all rational numbers consists of the numbers in this sequence together with 0 and the negative rational numbers:
    \(-r_1, -r_2, -r_3, -r_4, \ldots\). Let \(r_0 = 0\). Then the elements of the set of all rational numbers can be
    “counted” as follows: \(r_0, r_1, -r_1, r_2, -r_2, r_3, -r_ 3, r_4, -r_4, \ldots\). In other words, we can define a
    one-to-one correspondence as follows: for each integer \(n \geq 1\),
    \[
        G(x) =
        \left\{
        \begin{tabular}{ll}
            \(r_{n/2}\)      & if \(n\) is even \\
            \(-r_{(n-1)/2}\) & if \(n\) is odd
        \end{tabular}
        \right.
    \]
    Therefore, $\Q$ is countably infinite and hence countable.
\end{proof}

\subsection{Exercise 17}
Show that $\Q$, the set of all rational numbers, is dense along the number line by showing that given any two
rational numbers $r_1$ and $r_2$ with \(r_1 < r_2\), there exists a rational number $x$ such that \(r_1 < x < r_2\).

\begin{proof}
    Assume $r_1$ and $r_2$ are any two rational numbers with \(r_1 < r_2\). Let \(x = (r_1 + r_2) / 2\). Then $x$ is a
    rational number, because $r_1 + r_2$ is rational (being the sum of two rationals) and thus $x$ is the ratio of two
    rational numbers. Now we want to show \(r_1 < x < r_2\).

    \(\bm{r_1 < x}:\) Since \(r_1 < r_2\), we have \(2r_1 < r_1 + r_2\). Dividing by 2 we get \(r_1 < (r_1 + r_2)/2 = x\).

    \(\bm{x < r_2}:\) Since \(r_1 < r_2\), we have \(r_1 + r_2 < 2r_2\). Dividing by 2 we get \(x = (r_1 + r_2)/2 < r_2\).
\end{proof}

\subsection{Exercise 18}
Must the average of two irrational numbers always be irrational? Prove or give a counterexample.

\begin{proof}
    No. \underline{Counterexample:} Both \(\pi\) and \(-\pi\) are irrational, but their average is 0, which is rational.
\end{proof}

\subsection{Exercise 19}
Show that the set of all irrational numbers is dense along the number line by showing that given any two real numbers, there is an irrational number in between.

    {\it Hint:} Suppose $r$ and $s$ are real numbers with \(0 < r < s\). Let $n$ be an integer such that \(\dps \frac
{\sqrt{2}}{s-r} < n\), and let \(\dps m = \floor {\frac{nr}{\sqrt{2}}} + 1\). Show that \(\dps m-1 \leq \frac{nr}
{\sqrt{2}} < m\), and use the fact that \(s = r + (s-r)\) to conclude that \(\dps r < \frac{\sqrt{2}m}{n} < s\).

\begin{proof}
    (following the Hint)

    1. Suppose $r$ and $s$ are real numbers with \(0 < r < s\).

    2. By 1, \(0 < s - r\), so \(\dps \frac{\sqrt{2}}{s-r}\) is a positive real number.

    3. By 2 and Archimedean property, there is a positive integer $n$ such that \(\dps \frac{\sqrt{2}}{s-r} < n\).

    4. Define \(\dps m = \floor{\frac{nr}{\sqrt{2}}} + 1\). Then $m$ is an integer by definition of floor and $+$.

    5. By definition of floor, \(\dps \floor{\frac{nr}{\sqrt{2}}} \leq \frac{nr}{\sqrt{2}} < \floor{\frac{nr}
        {\sqrt{2}}}+1\). So \(\dps m-1 \leq\frac{nr}{\sqrt{2}}<m\).

    6. By 5, \(\dps\frac{nr}{\sqrt{2}} < m\), so \(\dps r < \frac{\sqrt{2}m}{n}\).

    7. By 3, \(\dps \frac{\sqrt{2}}{n} < s-r\). By 5, \(\dps \frac{\sqrt{2}(m-1)}{n} \leq r\).

    8. Since \(s = r + (s-r)\), by 7, \(\dps \frac{\sqrt{2}(m-1)}{n} + \frac{\sqrt{2}}{n} < r + (s-r) = s\). So
    \(\dps \frac{\sqrt{2}m}{n} < s\).

    9. By 6 and 8, \(\dps r < \frac{\sqrt{2}m}{n} < s\).

    10. \(\dps \frac{\sqrt{2}m}{n}\) is irrational, since $m/n$ is rational and $\sqrt{2}$ is irrational, and the product
    of a rational and an irrational is irrational.

    11. By 9 and 10, there is an irrational number between $r$ and $s$, as needed.

        {\it Generalizing to any real numbers $r < s$:}

    So far we have assumed \(0 < r < s\). Now we need to generalize the argument to any two real numbers $r < s$.

        {\bf Case 1 ($\bm{r = 0}$):} Then $0 < s$, so let \(y = (r+s) / 2\) and notice $r = 0 < y < s$. So we can use the
    above argument with $y$ instead of $r$ to obtain an irrational number $x$ between $y$ and $s$: \(y < x < s\),
    which implies \(r < x < s\), as needed.

        {\bf Case 2 ($\bm{s = 0}$):} Then $r < 0$ so $0 < -r$, so we can use Case 1 with $s = 0$ and $-r$ instead of $r$ and
    $s$ to get an irrational $x$ with \(0 = s < x < -r\), which means that \(r < -x < s\) (and $-x$ is also irrational),
    as needed.

    Now for the rest ($r \neq 0$ and $s \neq 0$), there are 3 cases:

    {\bf Case 3 (\(\bm{0 < r < s}\)):} This is handled by the main argument above.

        {\bf Case 4 (\(\bm{r < s < 0}\)):} We have \(0 < -s < -r\), so we can use Case 1 to get an irrational $x$ such that
    \(-s < x < -r\), which implies \(r < -x < s\) (where $-x$ is also irrational), as needed.

        {\bf Case 5 (\(\bm{r < 0 < s}\)):} By Case 1 there is an irrational $x$ with \(0 < x < s\), so \(r < 0 < x < s\),
    thus \(r < x < s\) as needed.
\end{proof}

\subsection{Exercise 20}
Give two examples of functions from $\Z$ to $\Z$ that are one-to-one but not onto.

\begin{proof}

\end{proof}

\subsection{Exercise 21}
Give two examples of functions from $\Z$ to $\Z$ that are onto but not one-to-one.

\begin{proof}
    Define \(f: \Z \to \Z\) as follows:
    \[
        f(n) =
        \left\{
        \begin{tabular}{ll}
            \(n\)   & if \(n \leq 0\) \\
            \(0\)   & if \(n = 1\)    \\
            \(n-1\) & if \(2 \leq n\)
        \end{tabular}
        \right.
    \]
    Then $f$ is not 1-1 because \(f(0) = f(1) = 0\) but \(0 \neq 1\). But $f$ is onto, because the first case covers
    the range of all negative integers and 0, the second case covers 0, and the last case covers the range of all positive integers. Similarly
    \[
        g(n) =
        \left\{
        \begin{tabular}{ll}
            \(n\)   & if \(n \leq 0\) \\
            \(0\)   & if \(n = 1\)    \\
            \(0\)   & if \(n = 2\)    \\
            \(n-2\) & if \(3 \leq n\)
        \end{tabular}
        \right.
    \]
    is also onto but not 1-1.
\end{proof}

\subsection{Exercise 22}
Define a function \(g: \Z^+ \times \Z^+ \to \Z^+\) by the formula \(g(m, n) = 2^m3^n\) for all \((m, n) \in \Z^+
\times \Z^+\). Show that $g$ is one-to-one and use this result to prove that \(\Z^+ \times \Z^+\) is countable.

\begin{proof}
    Assume \(g(m_1, n_1) = g(m_2, n_2)\). Then \(2^{m_1}3^{n_1} = 2^{m_2}3^{n_2}\). So we have two prime factorizations of
    the same positive integer. By the uniqueness of prime factorizations, we have \(m_1 = m_2\) and \(n_1 = n_2\).
    Thus $g$ is 1-1. Then $g$ is a one-to-one correspondence between \(\Z^+ \times \Z^+\) and \(g(\Z^+ \times \Z^+)\).
    Since \(g(\Z^+ \times \Z^+) \subseteq \Z^+\) and $\Z^+$ is countable, by Theorem 7.4.3 \(g(\Z^+ \times \Z^+)\) is
    countable. Since $g$ is a one-to-one correspondence, \(\Z^+ \times \Z^+\) is countable.
\end{proof}

\subsection{Exercise 23}
\begin{figure}[ht!]
    \centering
    \includegraphics[scale=0.5]{../images/7.4.23.a.png}
\end{figure}

\subsubsection{(a)}
Explain how to use the following diagram to show that \(\Z^{nonneg} \times \Z^{nonneg}\) and \(\Z^{nonneg}\) have
the same cardinality.

\begin{proof}
    Define a function \(G: \Z^{nonneg} \to \Z^{nonneg} \times \Z^{nonneg}\) as follows: Let \(G(0) = (0, 0)\), and then
    follow the arrows in the diagram, letting each successive ordered pair of integers be the value of $G$ for the next successive integer. Thus, for instance,

    \(G(1) = (1, 0), G(2) = (0, 1), G(3) = (2, 0), G(4) = (1, 1), G(5) = (0, 2), G(6) = (3, 0)\),

    \(G(7) = (2, 1), G(8) = (1, 2)\), and so forth.
\end{proof}

\subsubsection{(b)}
Define a function \(H: \Z^{nonneg} \times \Z^{nonneg} \to \Z^{nonneg}\) by the formula
\[
    H(m, n) = n + \frac{(m+n)(m+n-1)}{2}
\]
for all nonnegative integers $m$ and $n$. Interpret the action of $H$ geometrically using the diagram of part (a).

\begin{proof}
    Observe that if the top ordered pair of any given diagonal is $(k, 0)$, the entire diagonal (moving from top to
    bottom) consists of \((k, 0), (k - 1, 1), (k - 2, 2), \ldots, (2, k - 2), (1, k - 1), (0, k)\). Thus for every
    ordered pair \((m, n)\) within any given diagonal, the value of \(m + n\) is constant, and as you move down the
    ordered pairs in the diagonal, starting at the top, the value of the second element of the pair keeps increasing by 1.
\end{proof}

\subsection{Exercise 24}
Prove that the function $H$ defined analytically in exercise 23b is a one-to-one correspondence.

\begin{proof}
    {\bf $\bm{H}$ is 1-1:} Assume \(H(m_1, n_1) = H(m_2, n_2)\). So
    \[
        n_1 + \frac{(m_1 + n_1)(m_1 + n_1 - 1)}{2} = n_2 + \frac{(m_2 + n_2)(m_2 + n_2 - 1)}{2}.
    \]
    Multiplying by 2 and moving everything to the left, we get
    \[
        2n_1 - 2n_2 + (m_1 + n_1)(m_1 + n_1 - 1) - (m_2 + n_2)(m_2 + n_2 - 1) = 0.
    \]
    ???

    {\bf $\bm{H}$ is onto:} Assume \(z \in \Z^{nonneg}\). The set
    \[
        S = \{w \in \Z^{nonneg} \, | \, z \leq \frac{(w+1)w}{2}\}
    \]
    is nonempty, since \(z \in S\). By the well-ordering principle $S$ has a least element, say $y$. Therefore
    \[
        \frac{y(y-1)}{2} < z \leq \frac{(y+1)y}{2}
    \]
    So there exists \(n \in \Z^{nonneg}\) such that \(\dps z = \frac{y(y-1)}{2} + n\). Let \(m = y - n\). Then \(m \in \Z^{nonneg}\) and
    \[
        z = n + \frac{y(y-1)}{2} = n + \frac{(m+n)(m+n-1)}{2}
    \]
    Therefore \(H(m, n) = z\), {\it [as was to be shown.]}
\end{proof}

\subsection{Exercise 25}
Prove that \(0.1999 \ldots = 0.2\).

\begin{proof}
    Let \(x = 0.1999 \ldots\). Then \(10x = 1.9999 \ldots\) and \(100x = 19.9999 \ldots\). So \(100x - 10x = 18\), or
    \(90x = 18\). Thus \(x = 18/90 = 1/5 = 0.2\).
\end{proof}

\subsection{Exercise 26}
Prove that any infinite set contains a countably infinite subset.

\begin{proof}
    Let $A$ be an infinite set. Construct a countably infinite subset \(a_1, a_2, a_3, \ldots\) of $A$, by letting $a_1$
    be any element of $A$, letting $a_2$ be any element of $A$ other than $a_1$, letting $a_3$ be any element of $A$ other
    than $a_1$ or $a_2$, and so forth. This process never stops (and hence \(a_1, a_2, a_3, \ldots\) is an infinite
    sequence) because $A$ is an infinite set. More formally,

    1. Let $a_1$ be any element of $A$.

    2. For each integer \(n \neq 2\), let an be any element of \(A = \{a_1, a_2, a_3, \ldots, a_{n-1}\}\). Such an element
    exists, for otherwise \(A - \{a_1, a_2, a_3, \ldots, a_{n-1}\}\) would be empty and $A$ would be finite.
\end{proof}

\subsection{Exercise 27}
Prove that if $A$ is any countably infinite set, $B$ is any set, and \(g: A \to B\) is onto, then $B$ is countable.

\begin{proof}
    Suppose $A$ is any countably infinite set, $B$ is any set, and \(g: A \to B\) is onto. Since $A$ is countably
    infinite, there is a one-to-one correspondence \(f: \Z^+ \to A\). Then, in particular, $f$ is onto, and so by
    Theorem 7.3.4, \(g \circ f\) is an onto function from \(\Z^+\) to $B$. Define a function \(h: B \to \Z^+\) as
    follows: Suppose $x$ is any element of $B$. Since \(g \circ f\) is onto, \(\{m \in \Z^+ \, | \, (g \circ f)(m) = x\}
    \neq \es\). Thus, by the well-ordering principle for the integers, this set has a least element. In other words,
    there is a least positive integer $n$ with \((g \circ f)(n) = x\). Let $h(x)$ be this integer. We claim that $h$ is
    one-to-one. Suppose \(h(x_1) = h(x_2) = n\). By definition of $h$, $n$ is the least positive integer with \((g \circ
    f)(n) = x_1\). Moreover, by definition of $h$, $n$ is the least positive integer with \((g \circ f)(n) = x_2\). Hence
    \(x_1 = (g \circ f)(n) = x_2\). Thus $h$ is a one-to-one correspondence between $B$ and a subset $S$ of positive
    integers (the range of $h$). Since any subset of a countable set is countable (Theorem 7.4.3), $S$ is
    countable, and so there is a one-to-one correspondence between $B$ and a countable set. It follows from the
    transitive property of cardinality that $B$ is countable.
\end{proof}

\subsection{Exercise 28}
Prove that a disjoint union of any finite set and any countably infinite set is countably infinite.

\begin{proof}
    1. Assume \(A = \{a_1, \ldots, a_n\}\) is a finite set, $B$ is a countably infinite set, and $A$ and $B$ are disjoint.

    2. By 1 and definition of countably infinite, there is a one-to-one correspondence \(f: \Z^+ \to B\).

    3. Using $f$ from 2, define \(g: \Z^+ \to (A \cup B)\) as follows: for each \(i \in \Z^+\), let
    \[
        g(i) =
        \left\{
        \begin{tabular}{ll}
            \(a_i\)      & if \(1 \leq i \leq n\) \\
            \(f(i - n)\) & if \(n+1 \leq i\).
        \end{tabular}
        \right.
    \]
    4. Then $g$ is 1-1:

    Assume \(g(i) = g(j)\). Since $A$ and $B$ are disjoint by 1, either both $g(i)$ and $g(j)$ are in $A$, or they are
    both in B.

    4.1. If they are both in $A$ then \(a_i = a_j\), which implies $i = j$.

    4.2. If they are both in $B$, then \(f(i-n) = f(j-n)\). Since $f$ is 1-1 by 2, \(i-n = j-n\), thus \(i = j\).

    So $g$ is 1-1.

    5. And $g$ is onto: assume \(x \in A \cup B\).

    5.1. If \(x \in A\) then \(x = a_i\) for some \(i \in \{1, \ldots, n\}\), so \(g(i) = a_i = x\), {\it [as needed].}

    5.2. If \(x \in B\) then since $f$ is onto by 2, there exists \(m \in \Z^+\) such that \(f(m)=x\). Let \(i =m+n\).
    Then \(g(i) = g(m+n) = f(m+n-n) = f(m) = x\), {\it [as needed].}

    6. So by 4 and 5 $g$ is a one-to-one correspondence between \(\Z^+\) and \(A \cup B\).
\end{proof}

\subsection{Exercise 29}
Prove that a union of any two countably infinite sets is countably infinite.

\begin{proof}
    Suppose $A$ and $B$ are any two countably infinite sets. Then there are one-to-one correspondences \(f: \Z^+ \to A\) and \(g: \Z^+ \to B\).

        {\bf Case 1 (\(\bm{A \cap B = \es}\)):} In this case define \(h: \Z^+ \to A \cup B\) as follows: For every integer \(n \geq 1\),
    \[
        h(n) =
        \left\{
        \begin{tabular}{ll}
            \(f(n/2)\)     & if $n$ is even \\
            \(g((n+1)/2)\) & if $n$ is odd.
        \end{tabular}
        \right.
    \]
    {\bf $h$ is one-to-one:} Assume \(h(n_1) = h(n_2)\).

    Since \(A \cap B = \es\), $n_1$ and $n_2$ are either both odd or both even.

    If $n_1$ and $n_2$ are both even, then \(f(n_1/2) =f(n_2/2)\). Since $f$ is 1-1, \(n_1/2 = n_2/2\) so \(n_1 = n_2\).

    If $n_1$ and $n_2$ are both odd, then \(g((n_1+1)/2) = g((n_2+1)/2)\). Since $g$ is 1-1, \((n_1+1)/2 = (n_2+1)/2\)
    so \(n_1 = n_2\).

        {\bf $h$ is onto:} Assume \(x \in A \cup B\).

    Since \(A \cap B = \es\), either \(x \in A\) or \(x \in B\).

    If \(x \in A\), since $f$ is onto, there is $n \in \Z^+$ such that \(f(n) = x\). Then \(h(2n) = f(2n/2) =f(n) = x\).

    If \(x \in B\), since $g$ is onto, there is $n \in \Z^+$ such that \(g(n) = x\).
    Then \(h(2n-1) = g((2n-1+1)/2) = g(n) = x\).

    So $h$ is a one-to-one correspondence between \(\Z^+\) and \(A \cup B\). So \(A \cup B\) is countably infinite.

        {\bf Case 2 (\(\bm{A \cap B \neq \es}\)):} In this case let \(C = B - A\). Then \(A \cup B = A \cup C\) and \(A \cap C
    = \es\).

    If $C$ is countably infinite, then by Case 1 \(A \cup C\) is countably infinite. If $C$ is finite, then by exercise
    28 \(A \cup C\) is countably infinite.

    Since \(A \cup B = A \cup C\), \(A \cup B\) is also countably infinite.
\end{proof}

\subsection{Exercise 30}
Use the result of exercise 29 to prove that the set of all irrational numbers is uncountable.

    {\it Hint:} Use proof by contradiction and the fact that the set of all real numbers is uncountable.

\begin{proof}
    Argue by contradiction and assume the set of all irrational numbers, let's call it $IR$, is countable.

    We know that the set of all rational numbers, $\Q$, is countable.

    By exercise 29, \(IR \cup \Q\) is countable. But \(IR \cup \Q = \R\), the set of all real numbers, which is
    uncountable, contradiction.

    Thus our supposition was false and $IR$ is uncountable.
\end{proof}

\subsection{Exercise 31}
Use the results of exercises 28 and 29 to prove that a union of any two countable sets is countable.

\begin{proof}
    Consider the following cases:

    (1) $A$ and $B$ are both finite: then \(A \cup B\) is also finite, therefore countable.

    (2) One of $A, B$ is finite and the other is countably infinite, and \(A \cap B = \es\): then by exercise 28
    \(A \cup B\) is countably infinite.

    (3) One of $A, B$ is finite and the other is countably infinite, and \(A \cap B \neq \es\):

    First assume $B$ is finite. Let \(C = B - A\). Notice \(A \cap C = \es\) and \(A \cup B = A \cup C\). Then $C$ is finite, so by exercise 28 \(A \cup C\) is countably infinite. Hence \(A \cup B\) is countably infinite.

    Now assume $A$ is finite. Let \(C = A - B\). Notice \(B \cap C = \es\) and \(A \cup B = B \cup C\). Then $C$ is finite, so by exercise 28 \(B \cup C\) is countably infinite. Hence \(A \cup B\) is countably infinite.

    (4) both $A$ and $B$ are countably infinite: then by exercise 29 \(A \cup B\) is countably infinite.
\end{proof}

\subsection{Exercise 32}
Prove that \(\Z \times \Z\), the Cartesian product of the set of integers with itself, is countably infinite.

\begin{proof}
    By Example 7.4.2, the function \(F:\Z^+ \to \Z\) defined by
    \[
        F(n) =
        \left\{
        \begin{tabular}{cl}
            \(\dps \frac{n}{2}\)    & if $n$ is a positive even integer \\
            \(\dps -\frac{n-1}{2}\) & if $n$ is a positive odd integer
        \end{tabular}
        \right.
    \]
    is a one-to-one correspondence. Define \(G: \Z^+ \times \Z^+ \to \Z \times \Z\) by \(G(m, n) = (F(m), F(n))\).

        {\bf $G$ is 1-1:} Assume \(G(m_1, n_1) = G(m_2, n_2)\). Then \((F(m_1), F(n_1)) = (F(m_2), F(n_2))\). By definition
    of an ordered pair, \(F(m_1) = F(m_2)\) and \(F(n_1) = F(n_2)\). Since $F$ is 1-1, \(m_1 = m_2\) and \(n_1 =n_2\).
    So by definition of ordered pair, \((m_1, n_1) = (m_2, n_2)\).

        {\bf $G$ is onto:} Assume \((x,y) \in \Z \times \Z\). Since $F$ is onto, there exist \(m, n \in \Z^+\) such that
    \(F(m) = x\) and \(F(n) = y\). Thus \(G(m,n) = (F(m),F(n)) = (x,y)\).

    So $G$ is a one-to-one correspondence, thus \(\Z^+ \times \Z^+\) has the same cardinality as \(\Z \times \Z\). By
    exercise 22, \(\Z^+ \times \Z^+\) has the same cardinality as \(\Z^+\). By the transitivity of cardinality,
    \(\Z \times \Z\) has the same cardinality as \(\Z^+\), {\it [as was to be shown.]}
\end{proof}

\subsection{Exercise 33}
Use the results of exercises 27, 31, and 32 to prove the following: If $R$ is the set of all solutions to all equations of the form \(x^2 + bx + c = 0\), where $b$ and $c$ are integers, then $R$ is countable.

\begin{proof}
    1. By exercise 32, \(\Z \times \Z\) is countable.

    2. There is a one-to-one correspondence $F$ between \(\Z \times \Z\) and the set of quadratic functions \(S = \{f \,
    | \, f(x) = x^2 + bx + c \text{ for some integers } b,c\}\), given by \(F(b,c) = f\) where \(f(x) = x^2 + bx + c\).
    Thus $S$ is countable.

    3. Consider the subset $S_{1,2}$ of $S$ consisting of only those quadratic functions that have 1 or 2 solutions
    (because some quadratics like \(f(x) = x^2+1\) have no real solutions). Then $S_{1,2}$ is countable by Theorem 7.4.3.

    4. We can split $S_{1,2}$ into two disjoint subsets $S_1$ and $S_2$, consisting of those quadratic functions that
    have exactly 1 or exactly 2 real solutions, respectively. Then both $S_1$ and $S_2$ are countable by Theorem 7.4.3.

    5. Let $R_1$ and $R_2$ be sets of real numbers defined as the sets of solutions to all the quadratic functions in
    $S_1$ and $S_2$, respectively.

    6. There is a function \(G: S_1 \to R_1\) given by, for each \(f \in S_1\), letting \(G(f) = \) the (one) real
    solution to $f(x) = 0$. $G$ is a one-to-one correspondence, because for any $r \in R_1$ there is only one $f \in S_1$
    which has $r$ as its only solution, namely \(f(x) = (x-r)^2\). So $R_1$ is countable.

    7. Each $f \in S_2$ has two distinct roots, so one root is smaller, the other is bigger. We can split $R_2$ into two
    (possibly overlapping) subsets \(R_2^{small}\) and \(R_2^{big}\), by defining \(R_2^{small}\) to be the set of
    the {\it smaller} solutions to quadratic functions in $S_2$, and \(R_2^{big}\) to be the set of the {\it bigger}
    solutions to quadratic functions in $S_2$. Then there are onto functions \(H: S_2 \to R_2^{small}\) and \(J: S_2 \to
    R_2^{big}\) defined by, for each $f \in S_2$, \(H(f) = \) the smaller root of $f$ and \(J(f) = \) the bigger root of
    $f$, respectively. Therefore by exercise 27, both \(R_2^{small}\) and \(R_2^{big}\) are countable.

    8. By exercise 31 \(R_2^{small} \cup R_2^{big}\) is countable. Notice that \(R_2 \subseteq R_2^{small} \cup
    R_2^{big}\), so by Theorem 7.4.3 \(R_2\) is countable.

    9. Finally, by exercise 31 \(R_1 \cup R_2\) is countable. Notice that \(R \subseteq R_1 \cup R_2\), so by Theorem
    7.4.3 \(R\) is countable.
\end{proof}

\subsection{Exercise 34}
Let \(\ps(S)\) be the set of all subsets of $S$, and let $T$ be the set of all functions from $S$ to \(\{0,1\}\).
Show that \(\ps(S)\) and $T$ have the same cardinality.

\begin{proof}
    Define a function \(f: \ps(S) \to T\) as follows: For each subset $A$ of $S$, let \(f(A)=\chi_A\), the characteristic
    function of $A$, where \(\chi_A: S \to \{0, 1\}\) is defined by the rule: for every \(x \in S\),
    \[
        \chi_A(x) =
        \left\{
        \begin{tabular}{ll}
            1 & if \(x \in A\)     \\
            0 & if \(x \notin A\).
        \end{tabular}
        \right.
    \]
    {\bf Show that $f$ is one-to-one:} Assume $A_1$ and $A_2$ are subsets of $S$ and \(\chi_{A_1} = \chi_{A_2}\). So for
    all \(x \in S\), \(\chi_{A_1}(x) = \chi_{A_2}(x)\). This means that if \(\chi_{A_1}(x) = \chi_{A_2}(x) = 1\) then
    \(x \in A_1\) and \(x \in A_2\), and if \(\chi_{A_1}(x) = \chi_{A_2}(x) = 0\) then \(x \notin A_1\) and \(x \notin
    A_2\). So \(x \in A_1\) if and only if \(x \in A_2\) for all \(x \in S\), therefore \(A_1 = A_2\).

        {\bf Show that $f$ is onto:} Assume \(g: S \to \{0, 1\}\) is any function. Define \(A = \{x \in S \, | \, g(x)=1\}\).
    Notice that \(A \subseteq S\). Moreover, notice that \(g(x) = 1\) for all \(x \in A\) and \(g(x) = 0\) for all
    \(x \notin A\). Therefore \(g(x) = \chi_A(x)\) for all \(x \in S\). Thus \(g = \chi_A\).

    So $f$ is a one-to-one correspondence between \(\ps(S)\) and $T$.
\end{proof}

\subsection{Exercise 35}
Let $S$ be a set and let \(\ps(S)\) be the set of all subsets of $S$. Show that $S$ is “smaller than” \(\ps(S)\)
in the sense that there is a one-to-one function from $S$ to \(\ps(S)\) but there is no onto function from $S$ to
\(\ps(S)\).

\begin{proof}
    Let \(g: S \to \ps(S)\) be defined by: \(g(s) = \{s\}\) for all \(s \in S\). Then $g$ is one-to-one: if \(g(r) = g(s)\)
    then \(\{r\} = \{s\}\), which implies $r = s$.

    Suppose there is an onto function \(f: S \to \ps(S)\). Let \(A = \{x\in S\,|\, x\notin f(x)\}\). Then \(A \in \ps(S)\)
    and since $f$ is onto, there exists \(z \in S\) such that \(A = f(z)\). Now if \(z \in A\), then \(z \in f(z)\)
    because \(A = f(z)\), but also \(z \notin f(z)\) by the definition of $A$, a contradiction. So \(z \notin A\). Then
    \(z \notin f(z)\) because \(A = f(z)\), but since now \(z \notin f(z)\), $z$ satisfies the definition of being an
    element of $A$, so \(z \in A\), a contradiction again. Thus our initial supposition was false, and there is no onto function \(f: S \to \ps(S)\).
\end{proof}

\subsection{Exercise 36}
The Schroeder–Bernstein theorem states the following: If $A$ and $B$ are any sets with the property that there is a
one-to-one function from $A$ to $B$ and a one-to-one function from $B$ to $A$, then $A$ and $B$ have the same
cardinality. Use this theorem to prove that there are as many functions from \(\Z^+\) to \(\{0, 1, 2, 3, 4, 5, 6, 7,
8, 9\}\) as there are functions from \(\Z^+\) to \(\{0, 1\}\).

\begin{proof}
    Let $R$ be the set of functions from \(\Z^+\) to \(\{0, 1, 2, 3, 4, 5, 6, 7, 8, 9\}\).

    Let $S$ be the set of functions from \(\Z^+\) to \(\{0, 1,\}\).

    Define \(F: S \to R\) as follows: given $g \in S$, define $f \in R$ by \(f(z) = g(z)\) for all \(z \in \Z^+\).

    $F$ is 1-1: assume \(f_1 = F(g_1), f_2 = F(g_2)\) and assume \(f_1 = f_2\). So by definition of \(f_1, f_2\),
    \(g_1(z) = g_2(z)\) for all \(z \in \Z^+\), which implies \(g_1 = g_2\), {\it [as needed.]}

    Given \(f \in R\), we can think of $f$ as an infinite sequence that represents the decimal expansion of a real
    number between 0 and 1. For example, if \(f(1) = 5, f(2) = 9, f(3) = 3, \ldots\) then $f$ corresponds to the real
    number \(f_{dec} = 0.593...\) which is an infinite series written in terms of negative powers of 10:
    \[
        f_{dec} = 5 \cdot 10^{-1} + 9 \cdot 10^{-2} + 3 \cdot 10^{-3} + \cdots
    \]
    For every such number between 0 and 1 with infinite decimal expansion, there is a corresponding infinite {\it binary}
    expansion, which converges to the same real number but uses 0's and 1's in its digits. For the example of $f_{dec}$
    above, the corresponding binary expansion would be: \(f_{bin} = 0.1001011111001110111...\):
    \[
        f_{bin} = 1 \cdot 2^{-1} + 0 \cdot 2^{-2} + 0 \cdot 2^{-3} + 1 \cdot 2^{-4} + \cdots
    \]
    This infinite binary sequence can be viewed as a function \(g:\Z^+ \to\{0,1\}\).

    Define \(G: R \to S\) as \(G(f) = f_{bin}\) (viewed as a function) for each $f \in S$. $G$ is 1-1 by the uniqueness
    of binary expansions, because both the decimal expansion and the binary expansion converge to the same real number.

    So there is a 1-1 function $R \to S$ and a 1-1 function $S \to R$. Then by the Schroeder-Bernstein theorem, $R$ and
    $S$ have the same cardinality.
\end{proof}

\subsection{Exercise 37}
Prove that if $A$ and $B$ are any countably infinite sets, then \(A \times B\) is countably infinite.

\begin{proof}
    Since $A$ and $B$ are countable, their elements can be listed as \(A: a_1, a_2, a_3, \ldots\) and \(B: b_1, b_2, b_3, \ldots\).

    We can represent the elements of \(A \times B\) in a grid:

    \begin{center}
        \begin{tabular}{cccc}
            \((a_1, b_1)\) & \((a_1, b_2)\) & \((a_1, b_3)\) & \(\ldots\) \\
            \((a_2, b_1)\) & \((a_2, b_2)\) & \((a_2, b_3)\) & \(\ldots\) \\
            \((a_3, b_1)\) & \((a_3, b_2)\) & \((a_3, b_3)\) & \(\ldots\) \\
            \(\vdots\)     & \(\vdots\)     & \(\vdots\)     &            \\
        \end{tabular}
    \end{center}

    Define \(F: A \times B \to \Z^+ \times \Z^+\) by \(F(a_i, b_j) = (i, j)\). It is easy to see that $F$ is a one-to-one
    correspondence. Since \(\Z^+ \times \Z^+\) is countable by exercise 22, \(A \times B\) is also countable.
\end{proof}

\subsection{Exercise 38}
Suppose \(A_1, A_2, A_3, \ldots\) is an infinite sequence of countable sets. Recall that
\[
    \bigcup_{i=1}^{\infty} A_i = \{x \, | \, x \in A_i \text{ for some positive integer } i\}.
\]
Prove that $A$ is countable. (In other words, prove that a countably infinite union of countable sets is countable.)

\begin{proof}
    First suppose \(A_1, A_2, A_3, \ldots\) are mutually disjoint.

    Since \(A_1, A_2, A_3, \ldots\) are all countable, their elements can be listed as:

    \begin{center}
        \begin{tabular}{ccccc}
            \(A_1:\)   & \(a_1^1\)  & \(a_1^2\)  & \(a_1^3\)  & \(\ldots\) \\
            \(A_2:\)   & \(a_2^1\)  & \(a_2^2\)  & \(a_2^3\)  & \(\ldots\) \\
            \(A_3:\)   & \(a_3^1\)  & \(a_3^2\)  & \(a_3^3\)  & \(\ldots\) \\
            \(\vdots\) & \(\vdots\) & \(\vdots\) & \(\vdots\) &            \\
        \end{tabular}
    \end{center}

    This listing also lists the elements of their union, \(\bigcup_{i=1}^{\infty} A_i\).

    Define \(F: \bigcup_{i=1}^{\infty} A_i \to \Z^+ \times \Z^+\) by \(F(a_i^j) = (i, j)\). It is easy to see that $F$ is
    a one-to-one correspondence (since the union is disjoint). Therefore \(\bigcup_{i=1}^{\infty} A_i\) is countable.

    Now consider the general case, where the sets are not necessarily disjoint. We can replace them with sets \(B_1,
    B_2, B_3, \ldots\) that are disjoint, and such that \(\bigcup_{i=1}^{\infty} A_i = \bigcup_{i=1}^{\infty} B_i\)
    by discarding the duplicate elements. Then by the above argument, \(\bigcup_{i=1}^{\infty} B_i\) is countable, and
    then so is \(\bigcup_{i=1}^{\infty} A_i\).
\end{proof}

\end{document}
