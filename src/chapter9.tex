\documentclass[14pt]{extarticle}

\usepackage[table]{xcolor} % colored lines for tables
%\usepackage[normalem]{ulem} % strike through text
\usepackage{amsmath,mathtools,amsfonts,amsthm,amssymb,hyperref}
\usepackage{parskip,geometry,latexsym,bookmark,mathtools,float,cancel}
\usepackage{tcolorbox,bm}

\newtheorem{defn}{Definition}
\newtheorem{thm}{Theorem}
\newtheorem{claim}{Claim}
\newtheorem{lemma}{Lemma}

\newcommand{\dps}{\displaystyle}
\newcommand{\es}{\varnothing}
\newcommand{\fbl}{\underline{\hspace{1cm}}\,\,}
\newcommand{\R}{\mathbb{R}}
\newcommand{\Q}{\mathbb{Q}}
\newcommand{\Z}{\mathbb{Z}}
\newcommand{\from}{\leftarrow}
\newcommand{\true}{{\bf t}}
\newcommand{\false}{{\bf c}}
\newcommand{\bic}{\leftrightarrow}
\newcommand{\da}{\downarrow}
\newcommand{\fa}{\forall}
\newcommand{\te}{\exists}
\newcommand{\cy}{\color{cyan}}

\newcommand{\colsq}[1]{{\color{#1} $\blacksquare$}}

\newcommand{\base}[1]{{\cy #1}} % for log bases
\newcommand{\floor}[1]{{\left\lfloor#1\right\rfloor}}
\newcommand{\ceil}[1]{{\left\lceil#1\right\rceil}}
\newcommand\Ccancel[2][black]{\renewcommand\CancelColor{\color{#1}}\cancel{#2}}
\newcommand\Cbcancel[2][black]{\renewcommand\CancelColor{\color{#1}}\bcancel{#2}}

\renewcommand{\arraystretch}{1.2}

\hypersetup{colorlinks,allcolors=blue,linktoc=all}
\geometry{a4paper}
\geometry{margin=0.42in}

\title{Solutions to Chapter 9, Susanna Epp Discrete Math 5th Edition}

\author{https://github.com/spamegg1}

\begin{document}
\maketitle
\tableofcontents

\section{Exercise Set 9.1}

\subsection{Exercise 1}
Toss two coins 30 times and make a table showing the relative frequencies of 0, 1, and 2 heads. How do your 
values compare with those shown in Table 9.1.1?

\begin{proof}
\begin{center}
\arrayrulecolor{cyan}
\begin{tabular}{|c|c|c|}
\hline
{\bf\cy Event} & {\bf\cy Freq.} & {\bf\cy Rel. freq.} \\
\hline
2 heads obtained & 7 & 23.33\% \\
\hline
1 head obtained & 16 & 53.33\% \\
\hline
0 heads obtained & 7 & 23.33\% \\
\hline
\end{tabular}
\arrayrulecolor{black} % change it back!
\end{center}
\end{proof}

\subsection{Exercise 2}
In the example of tossing two quarters, what is the probability that at least one head is obtained? that coin 
$A$ is a head? that coins $A$ and $B$ are either both
heads or both tails?

\begin{proof}
3/4, 1/2, 1/2
\end{proof}

{\bf \cy In $3-6$ use the sample space given in example 9.1.1. Write each event as a set and compute its 
probability.}

\subsection{Exercise 3}
The event that the chosen card is red and is not a face card.

\begin{proof}
\(\{1\diamondsuit, 2\diamondsuit, 3\diamondsuit, 4\diamondsuit, 5\diamondsuit, 6\diamondsuit, 7\diamondsuit, 
8\diamondsuit, 9\diamondsuit, 10\diamondsuit, 1\heartsuit, 2\heartsuit, 3\heartsuit, 4\heartsuit, 5\heartsuit\), 

\(6\heartsuit, 7\heartsuit, 8\heartsuit, 9\heartsuit, 10\heartsuit\}\), probability \(= 20/52 \approx 38.5\%\)
\end{proof}

\subsection{Exercise 4}
The event that the chosen card is black and has an even number on it.

\begin{proof}
\(\{2\spadesuit, 4\spadesuit, 6\spadesuit, 8\spadesuit, 10\spadesuit, 2\clubsuit, 4\clubsuit, 6\clubsuit, 
8\clubsuit, 10\clubsuit\}\), probability \(= 10/52 \approx 19.2\%\)
\end{proof}

\subsection{Exercise 5}
The event that the denomination of the chosen card is at least 10 (counting aces high).

\begin{proof}
\(\{10\spadesuit, J\spadesuit, Q\spadesuit, K\spadesuit, A\spadesuit, 10\diamondsuit, J\diamondsuit, Q\diamondsuit, 
K\diamondsuit, A\diamondsuit, 10\heartsuit, J\heartsuit, Q\heartsuit, K\heartsuit, A\heartsuit\), 

\(10\clubsuit, J\clubsuit, Q\clubsuit, K\clubsuit, A\clubsuit\}\), probability \(= 20/52=5/13 \approx 38.5\%\)
\end{proof}

\subsection{Exercise 6}
The event that the denomination of the chosen card is at most 4 (counting aces high).

\begin{proof}
\(\{2\clubsuit, 2\heartsuit, 2\spadesuit, 2\diamondsuit, 3\clubsuit, 3\heartsuit, 3\spadesuit, 3\diamondsuit, 
4\clubsuit, 4\heartsuit, 4\spadesuit, 4\diamondsuit\}\), probability \(= 12/52 \approx 23.0\%\)
\end{proof}

{\bf \cy In $7-10$, use the sample space given in example 9.1.2. Write each of the following events as a set and 
compute its probability.}

\subsection{Exercise 7}
The event that the sum of the numbers showing face up is 8.

\begin{proof}
\(\{26, 35, 44, 53, 62\}\), probability \(= 5/36 \approx 13.9\%\)
\end{proof}

\subsection{Exercise 8}
The event that the numbers showing face up are the same.

\begin{proof}
\(\{11, 22, 33, 44, 55, 66\}\), probability \(= 6/36 \approx 16.6\%\)
\end{proof}

\subsection{Exercise 9}
The event that the sum of the numbers showing face up is at most 6.

\begin{proof}
\(\{11, 12, 13, 14, 15, 21, 22, 23, 24, 31, 32, 33, 41, 42, 51\}\), prob. \(= 15/36 \approx 41.6\%\)
\end{proof}

\subsection{Exercise 10}
The event that the sum of the numbers showing face up is at least 9.

\begin{proof}
\(\{36, 45, 46, 54, 55, 56, 63, 64, 65, 66\}\), probability \(= 10/36 \approx 27.7\%\)
\end{proof}

\subsection{Exercise 11}
Suppose that a coin is tossed three times and the side showing face up on each toss is noted. Suppose also that on 
each toss heads and tails are equally likely. Let \(HHT\) indicate the outcome heads on the first two tosses and 
tails on the third, \(THT\) the outcome tails on the first and third tosses and heads on the second, and so forth.

\subsubsection{(a)}
List the eight elements in the sample space whose outcomes are all the possible head-tail sequences obtained in the 
three tosses.

\begin{proof}
\(\{HHH, HHT, HTH, HTT, THH, THT, TTH, TTT\}\)
\end{proof}

\subsubsection{(b)}
Write each of the following events as a set and find its probability:

(i) The event that exactly one toss results in a head.

(ii) The event that at least two tosses result in a head.

(iii) The event that no head is obtained.

\begin{proof}
(i) \(\{HTT, THT, TTH\}\), probability \(= 3/8 = 37.5\%\)

(ii) \(\{HHT, HTH, THH, HHH\}\), probability \(= 4/8 = 50.0\%\)

(iii) \(\{TTT\}\), probability \(= 1/8 = 12.5\%\)
\end{proof}

\subsection{Exercise 12}
Suppose that each child born is equally likely to be a boy or a girl. Consider a family with exactly three children. 
Let \(BBG\) indicate that the first two children born are boys and the third child is a girl, let \(GBG\) indicate 
that the first and third children born are girls and the second is a boy, and so forth.

\subsubsection{(a)}
List the eight elements in the sample space whose outcomes are all possible genders of the three children.

\begin{proof}
\(\{BBB, BBG, BGB, BGG, GBB, GBG, GGB, GGG\}\)
\end{proof}

\subsubsection{(b)}
Write each of the events in the next column as a set and find its probability.

(i) The event that exactly one child is a girl.

(ii) The event that at least two children are girls.

(iii) The event that no child is a girl.

\begin{proof}
(i) \(\{GBB, BGB, BBG\}\), probability \(= 3/8 = 37.5\%\)

(ii) \(\{GGB, GBG, BGG, GGG\}\), probability \(= 4/8 = 50\%\)

(iii) \(\{BBB\}\), probability \(= 1/8 = 12.5\%\)
\end{proof}

\subsection{Exercise 13}
Suppose that on a true/false exam you have no idea at all about the answers to three questions. You choose answers 
randomly and therefore have a 50–50 chance of being correct on any one question. Let \(CCW\) indicate that you were 
correct on the first two questions and wrong on the third, let \(WCW\) indicate that you were wrong on the first and 
third questions and correct on the second, and so forth.

\subsubsection{(a)}
List the elements in the sample space whose outcomes are all possible sequences of correct and incorrect responses 
on your part.

\begin{proof}
\(\{CCC, CCW, CWC, CWW, WCC, WCW, WWC, WWW\}\)
\end{proof}

\subsubsection{(b)}
Write each of the following events as a set and find its probability:

(i) The event that exactly one answer is correct.

(ii) The event that at least two answers are correct.

(iii) The event that no answer is correct.

\begin{proof}
(i) \(\{CWW, WCW, WWC\}\), probability \(= 3/8 = 37.5\%\)

(ii) \(\{CCW, CWC, WCC, CCC\}\), probability \(= 4/8 = 50\%\)

(iii) \(\{WWW\}\), probability \(= 1/8 = 12.5\%\)
\end{proof}

\subsection{Exercise 14}
Three people have been exposed to a certain illness. Once exposed, a person has a 50-50 chance of actually becoming ill.

\subsubsection{(a)}
What is the probability that exactly one of the people becomes ill?

\begin{proof}
probability \(= 3/8 = 37.5\%\)
\end{proof}

\subsubsection{(b)}
What is the probability that at least two of the people become ill?

\begin{proof}
probability \(= 4/8 = 50\%\)
\end{proof}

\subsubsection{(c)}
What is the probability that none of the three people becomes ill?

\begin{proof}
probability \(= 1/8 = 12.5\%\)
\end{proof}

\subsection{Exercise 15}
When discussing counting and probability, we often consider situations that may appear frivolous or of little practical 
value, such as tossing coins, choosing cards, or rolling dice. The reason is that these relatively simple examples 
serve as models for a wide variety of more complex situations in the real world. In light of this remark, 
comment on the relationship between your answer to exercise 11 and your answers to exercises $12-14$.

\begin{proof}
The answers to exercises 11, 12, 13, 14 are the same, because all 4 situations are modeled exactly the same way.
They each consist of 3 instances of the same event, each of which has 2 possible outcomes with 50\% probability (heads 
or tails, boy or girl, correct or incorrect, ill or not ill). They each have a sample space with 8 elements 
following the same pattern.
\end{proof}

\subsection{Exercise 16}
Two faces of a six-sided die are painted red, two are painted blue, and two are painted yellow. The die is rolled 
three times, and the colors that appear face up on the first, second, and third rolls are recorded.


\subsubsection{(a)}
Let \(BBR\) denote the outcome where the color appearing face up on the first and second rolls is blue and the color 
appearing face up on the third roll is red. Because there are as many faces of one color as of any other, the 
outcomes of this experiment are equally likely. List all 27 possible outcomes.

\begin{proof}
\(\{RRR, RRB, RRY, RBR, RBB, RBY, RYR, RYB, RYY,\) 

\(BRR, BRB, BRY, BBR, BBB, BBY, BYR, BYB, BYY,\)

\(YRR, YRB, YRY, YBR, YBB, YBY, YYR, YYB, YYY\}\)
\end{proof}

\subsubsection{(b)}
Consider the event that all three rolls produce different colors. One outcome in this event is \(RBY\) and another 
\(RYB\). List all outcomes in the event. What is the probability of the event?

\begin{proof}
\(\{RBY, RYB, YBR, BRY, BYR, YRB\}\), probability \(= 6/27 = 2/9 \approx 22.2\%\)
\end{proof}

\subsubsection{(c)}
Consider the event that two of the colors that appear face up are the same. One outcome in this event is \(RRB\) and 
another is \(RBR\). List all outcomes in the event. What is the probability of the event?

\begin{proof}
\(\{RRB, RBR, BRR, RRY, RYR, YRR, BBR, BRB, RBB, BBY, BYB, YBB,\)

\(YYR, YRY, RYY, YYB, YBY, BYY\}\), probability \(= 18/27 = 2/3 \approx 66.6\%\)
\end{proof}

\subsection{Exercise 17}
Consider the situation described in exercise 16.

\subsubsection{(a)}
Find the probability of the event that exactly one of the colors that appears face up is red.

\begin{proof}
\(\{RBB, RBY, RYB, RYY, BRB, BRY, BBR, BYR, YRB, YRY, YBR, YYR\}\), 

probability \(= 12/27 = 4/9 \approx 44.4\%\)
\end{proof}

\subsubsection{(b)}
Find the probability of the event that at least one of the colors that appears face up is red.

\begin{proof}
\(\{RRR, RRB, RRY, RBR, RBB, RBY, RYR, RYB, RYY, BRR, BRB, BRY,\) 

\(BBR, BYR, YRR, YRB, YRY, YBR, YYR\}\), probability \(= 20/27 \approx 74\%\)
\end{proof}

\subsection{Exercise 18}
An urn contains two blue balls (denoted \(B_1\) and \(B_2\)) and one white ball (denoted \(W\)). One ball is 
drawn, its color is recorded, and it is replaced in the urn. Then another ball is drawn, and its color is recorded.

\subsubsection{(a)}
Let \(B_1 W\) denote the outcome that the first ball drawn is \(B_1\) and the second ball drawn is \(W\). Because the 
first ball is replaced before the second ball is drawn, the outcomes of the experiment are equally likely. List all 
nine possible outcomes of the experiment.

\begin{proof}
\(\{B_1B_1, B_1B_2, B_1W, B_2B_1, B_2B_2, B_2W, WB_1, WB_2, WW\}\)
\end{proof}

\subsubsection{(b)}
Consider the event that the two balls that are drawn are both blue. List all outcomes in the event. 
What is the probability of the event?

\begin{proof}
\(\{B_1B_1, B_1B_2, B_2B_1, B_2B_2\}\), probability \(= 4/9 \approx 44.4\%\)
\end{proof}

\subsubsection{(c)}
Consider the event that the two balls that are drawn are of different colors. List all outcomes in the event. 
What is the probability of the event?

\begin{proof}
\(\{B_1W, B_2W, WB_1, WB_2\}\), probability \(= 4/9 \approx 44.4\%\)
\end{proof}

\subsection{Exercise 19}
An urn contains two blue balls (denoted \(B_1\) and \(B_2\)) and three white balls (denoted \(W_1, W_2\), and 
\(W_3\)). One ball is drawn, its color is recorded, and it is replaced in the urn. Then another ball is drawn and its 
color is recorded.

\subsubsection{(a)}
Let \(B_1 W_2\) denote the outcome that the first ball drawn is \(B_1\) and the second ball drawn is \(W_2\). 
Because the first ball is replaced before the second ball is drawn, the outcomes of the experiment are equally 
likely. List all 25 possible outcomes of the experiment.

\begin{proof}
\(\{B_1B_1, B_1B_2, B_1W_1, B_1W_2, B_1W_3\), 

\(B_2B_1, B_2B_2, B_2W_1, B_2W_2, B_2W_3\), 

\(W_1B_1, W_1B_2, W_1W_1, W_1W_2, W_1W_3\), 

\(W_2B_1, W_2B_2, W_2W_1, W_2W_2, W_2W_3\), 

\(W_3B_1, W_3B_2, W_3W_1, W_3W_2, W_3W_3\}\)
\end{proof}

\subsubsection{(b)}
Consider the event that the first ball that is drawn is blue. List all outcomes in the event. 
What is the probability of the event?

\begin{proof}
\(\{B_1B_1, B_1B_2, B_1W_1, B_1W_2, B_1W_3, B_2B_1, B_2B_2, B_2W_1, B_2W_2, B_2W_3\}\), 

probability \(= 10/25 = 40\%\)
\end{proof}

\subsubsection{(c)}
Consider the event that only white balls are drawn. List all outcomes in the event. 
What is the probability of the event?

\begin{proof}
\(\{W_1W_1, W_1W_2, W_1W_3, W_2W_1, W_2W_2, W_2W_3, W_3W_1, W_3W_2, W_3W_3\}\)

probability \(= 9/25 = 36\%\)
\end{proof}

\subsection{Exercise 20}
Refer to Example 9.1.3. Suppose you are appearing on a game show with a prize behind one of five closed doors: A, B, C, 
D, and E. If you pick the correct door, you win the prize. You pick door A. The game show host then opens one of the 
other doors and reveals that there is no prize behind it. Then the host gives you the option of staying with your 
original choice of door A or switching to one of the other doors that is still closed.

\subsubsection{(a)}
If you stick with your original choice, what is the probability that you will win the prize?

\begin{proof}
There are 5 cases. 

In Case 1, the prize is behind door A. In this case if I stick with door A, I would win.

In the remaining 4 cases, the prize is behind one of the doors B,C,D,E. In these cases if I stick with door A, I 
would lose. 

Each case has 20\% probability. If I stick with door A, I can only win in Case 1. Therefore the probability that I 
win the prize is 20\%.
\end{proof}

\subsubsection{(b)}
If you switch to another door, what is the probability that you will win the prize?

\begin{proof}
Like above, if I switch to another door, I lose in Case 1, and I have a 1/4 chance of winning in each of the other 4 
cases. Therefore the probability that I win the prize is 25\%.
\end{proof}

\subsection{Exercise 21}
\subsubsection{(a)}
How many positive two-digit integers are multiples of 3?

\begin{proof}
Between 10 and 99 (inclusive), the multiples of 3 are: \(12, 15, 18, \ldots, 93, 96, 99\). Notice that
\(12 = 3 \cdot {\cy 4}\) and \(99 = 3 \cdot {\cy 33}\). So there are as many positive two-digit integers that are 
multiples of 3 as there are integers from 4 to 33 inclusive. By Theorem 9.1.1 there are 33 - 4 + 1 = 30 such
integers.
\end{proof}

\subsubsection{(b)}
What is the probability that a randomly chosen positive two-digit integer is a multiple of 3?

\begin{proof}
There are \(99 - 10 + 1 = 90\) positive two-digit integers in all, and by part (a), 30 of these are multiples of 3. So 
the probability that a randomly chosen positive two-digit integer is a multiple of 3 is \(30/90=1/3 \approx 33.3\%\).
\end{proof}

\subsubsection{(c)}
What is the probability that a randomly chosen positive two-digit integer is a multiple of 4?

\begin{proof}
Of the integers from 10 through 99 that are multiples of 4, the smallest is 12 \((= 4 \cdot 3)\) and the largest is 96 
\((= 4 \cdot 24)\). Thus there are \(24 - 3 + 1 = 22\) two-digit integers that are multiples of 4. Hence the 
probability that a randomly chosen two-digit integer is a multiple of 4 is \(22/90 \approx 36.6\%\).
\end{proof}

\subsection{Exercise 22}
\subsubsection{(a)}
How many positive three-digit integers are multiples of 6?

\begin{proof}
They are \(102, 108, 114, \ldots, 984, 990, 996\). Notice \(102 = 6 \cdot 17\) and \(996 = 6 \cdot 166\). So there 
are \(166 - 17 + 1 = 150\) such integers.
\end{proof}

\subsubsection{(b)}
What is the probability that a randomly chosen positive three-digit integer is a multiple of 6?

\begin{proof}
There are \(999 - 100 + 1 = 900\) positive three-digit integers. So the probability is, by part (a), 
\(150/900 = 1/6 \approx 16.6\%\).
\end{proof}

\subsubsection{(c)}
What is the probability that a randomly chosen positive three-digit integer is a multiple of 7?

\begin{proof}
Multiples of 7 are \(7 \cdot 15 = 105, 7 \cdot 16 = 112, \ldots, 7 \cdot 141 = 987, 7 \cdot 142 = 994\). So there
are \(142 - 15 + 1 = 128\) such integers, then the probability is \(128/900 = 32/225 \approx 14.22\%\).
\end{proof}

\subsection{Exercise 23}
Suppose \(A[1], A[2], A[3], \ldots, A[n]\) is a one-dimensional array and \(n > 50\).

\subsubsection{(a)}
How many elements are in the array?

\begin{proof}
$n$ elements.
\end{proof}

\subsubsection{(b)}
How many elements are in the subarray \(A[4], A[5], \ldots, A[39]\)?

\begin{proof}
\(39 - 4 + 1 = 36\) elements.
\end{proof}

\subsubsection{(c)}
If \(3 \leq m \leq n\), what is the probability that a randomly chosen array element is in the subarray 
\(A[3], A[4], \ldots, A[m]\)?

\begin{proof}
There are \(m - 3 + 1 = m-2\) elements in the subarray. There are $n$ elements in the array. So the probability is 
\(\frac{m-2}{n}\).
\end{proof}

\subsubsection{(d)}
What is the probability that a randomly chosen array element is in the subarray shown below if \(n = 39\)?
\[A[\floor{n/2}], A[\floor{n/2}+1], \ldots, A[n]\]
\begin{proof}
\(\floor{39/2} = \floor{19.5} = 19\), therefore there are \(39 - 19 + 1 = 21\) elements in the subarray, and there 
are 39 elements in the array, so the probability is \(21/39 \approx 53.84\%\).
\end{proof}

\subsection{Exercise 24}
Suppose \(A[1], A[2], \ldots, A[n]\) is a one-dimensional array and \(n \geq 2\). Consider the sub-array 
\(A[1], A[2], \ldots, A[\floor{n/2}]\).

\subsubsection{(a)}
How many elements are in the sub-array (i) if $n$ is even? and (ii) if $n$ is odd?

\begin{proof}
(i) There are \(\floor{\frac{n}{2}} = \frac{n}{2}\) elements in the sub-array.

(ii) There are \(\floor{\frac{n}{2}} = \frac{n-1}{2}\) elements in the sub-array.
\end{proof}

\subsubsection{(b)}
What is the probability that a randomly chosen array element is in the sub-array (i) if $n$ is even? and 
(ii) if $n$ is odd?

\begin{proof}
There are $n$ elements in the array, so

(i) The probability that an element is in the given sub-array is \(\frac{n/2}{n} = \frac{1}{2}\),

(i) The probability that an element is in the given sub-array is \(\frac{(n-1)/2}{n} = \frac{n-1}{2n}\).
\end{proof}

\subsection{Exercise 25}
Suppose \(A[1], A[2], \ldots, A[n]\) is a one-dimensional array and \(n \geq 2\). Consider the sub-array 
\(A[\floor{n/2}], A[\floor{n/2}+1], \ldots, A[n]\).

\subsubsection{(a)}
How many elements are in the sub-array (i) if $n$ is even? and (ii) if $n$ is odd?

\begin{proof}
(i) There are \(n - \floor{\frac{n}{2}} + 1 = n - \frac{n}{2} + 1 = \frac{n+2}{2}\) elements in the sub-array.

(ii) There are \(n-\floor{\frac{n}{2}}+1 = n-\frac{n-1}{2} + 1 = \frac{n+3}{2}\) elements in the sub-array.
\end{proof}

\subsubsection{(b)}
What is the probability that a randomly chosen array element is in the sub-array (i) if $n$ is even? and 
(ii) if $n$ is odd?

\begin{proof}
There are $n$ elements in the array, so

(i) The probability that an element is in the given sub-array is \(\frac{(n+2)/2}{n} = \frac{n+2}{2n}\),

(i) The probability that an element is in the given sub-array is \(\frac{(n+3)/2}{n} = \frac{n+3}{2n}\).
\end{proof}

\subsection{Exercise 26}
What is the 27th element in the one-dimensional array \(A[42], A[43], \ldots, A[100]\)?

\begin{proof}
Let $k$ be the 27th element in the array. By Theorem 9.1.1, \(k - 42 + 1 = 27\), and so \(k = 42 + 27 - 1 = 68\). 
Thus the 27th element in the array is \(A[68]\).
\end{proof}

\subsection{Exercise 27}
What is the 62nd element in the one-dimensional array \(B[29], B[30], \ldots, B[100]\)?

\begin{proof}
Let $k$ be the 62nd element in the array. By Theorem 9.1.1, \(k - 29 + 1 = 62\), and so \(k = 29 + 62 - 1 = 90\). 
Thus the 62th element in the array is \(A[90]\).
\end{proof}

\subsection{Exercise 28}
If the largest of 56 consecutive integers is 279, what is the smallest?

\begin{proof}
Let $m$ be the smallest of the integers. By Theorem 9.1.1, \(279 - m + 1 = 56\), and so \(m = 279 - 56 + 1 = 224\). 
Thus the smallest of the integers is 224.
\end{proof}

\subsection{Exercise 29}
If the largest of 87 consecutive integers is 326, what is the smallest?

\begin{proof}
Let $m$ be the smallest of the integers. By Theorem 9.1.1, \(326 - m + 1 = 87\), and so \(m = 326 - 87 + 1 = 240\). 
Thus the smallest of the integers is 240.
\end{proof}

\subsection{Exercise 30}
How many even integers are between 1 and 1,001?

\begin{proof}
They are \(2 = 2 \cdot 1, 4 = 2 \cdot 2, \ldots, 998 = 2 \cdot 499, 1000 = 2 \cdot 500\). So there are 500 of them.
\end{proof}

\subsection{Exercise 31}
How many integers that are multiples of 3 are between 1 and 1,001?

\begin{proof}
They are \(3 = 3 \cdot 1, 6 = 3 \cdot 2, \ldots, 996 = 3 \cdot 332, 999 = 3 \cdot 333\). So there are 333 of them.
\end{proof}

\subsection{Exercise 32}
A certain non-leap year has 365 days, and January 1 occurs on a Monday.

\subsubsection{(a)}
How many Sundays are in the year?

\begin{proof}
Sundays occur on days 7, 14, 21, \(\ldots\), 364 of the year. Since \(7 = 7 \cdot 1\) and \(364 = 7 \cdot 52\),
there are 52 Sundays in the year.
\end{proof}

\subsubsection{(b)}
How many Mondays are in the year?

\begin{proof}
For each Sunday, there is a Monday in the same week. However there is also the 365th day, which comes directly
after the last Sunday, which is the 364th day. Therefore there are \(52 + 1 = 53\) Mondays in the year.
\end{proof}

\subsection{Exercise 33}
Prove Theorem 9.1.1. (Let $m$ be any integer and prove the theorem by mathematical induction on $n$.)

\begin{proof}
Let $m$ be any integer and let \(P(n)\) be the statement ``if \(m \leq n\), then there are \(n - m + 1\) integers 
from $m$ to $n$ inclusive.'' The base case is \(n=m\).

{\bf Show $P(m)$ is true:} There is only one integer from $m$ to $m$ inclusive, namely $m$ itself. And \(n - m + 1
 = m - m + 1 = 1\), so $P(m)$ is true.

{\bf Show that for any integer \(k \geq m\) if \(P(k)\) is true then \(P(k+1)\) is true:} Assume \(k \geq m\) and
assume there are \(k-m+1\) integers from $m$ to $k$ inclusive. Then there is one more integer, namely \(k+1\),
from $m$ to $k+1$ inclusive, thus there are \((k-m+1)+1 = (k+1)-m+1\) integers from $m$ to $k+1$ inclusive. So 
\(P(k+1)\) is true.
\end{proof}

\section{Exercise Set 9.2}

\subsection{Exercise 1}

\begin{proof}

\end{proof}

\subsection{Exercise 2}

\begin{proof}

\end{proof}

\subsection{Exercise 3}

\begin{proof}

\end{proof}

\subsection{Exercise 4}

\begin{proof}

\end{proof}

\subsection{Exercise 5}

\begin{proof}

\end{proof}

\subsection{Exercise 6}

\subsubsection{(a)}

\begin{proof}

\end{proof}

\subsubsection{(b)}

\begin{proof}

\end{proof}

\subsubsection{(c)}

\begin{proof}

\end{proof}

\subsubsection{(d)}

\begin{proof}

\end{proof}

\subsection{Exercise 7}

\subsubsection{(a)}

\begin{proof}

\end{proof}

\subsubsection{(b)}

\begin{proof}

\end{proof}

\subsubsection{(c)}

\begin{proof}

\end{proof}

\subsection{Exercise 8}

\begin{proof}

\end{proof}

\subsection{Exercise 9}

\subsubsection{(a)}

\begin{proof}

\end{proof}

\subsubsection{(b)}

\begin{proof}

\end{proof}

\subsubsection{(c)}

\begin{proof}

\end{proof}

\subsection{Exercise 10}

\subsubsection{(a)}

\begin{proof}

\end{proof}

\subsubsection{(b)}

\begin{proof}

\end{proof}

\subsection{Exercise 11}

\subsubsection{(a)}

\begin{proof}

\end{proof}

\subsubsection{(b)}

\begin{proof}

\end{proof}

\subsubsection{(c)}

\begin{proof}

\end{proof}

\subsection{Exercise 12}

\subsubsection{(a)}

\begin{proof}

\end{proof}

\subsubsection{(b)}

\begin{proof}

\end{proof}

\subsection{Exercise 13}

\subsubsection{(a)}

\begin{proof}

\end{proof}

\subsubsection{(b)}

\begin{proof}

\end{proof}

\subsubsection{(c)}

\begin{proof}

\end{proof}

\subsection{Exercise 14}

\subsubsection{(a)}

\begin{proof}

\end{proof}

\subsubsection{(b)}

\begin{proof}

\end{proof}

\subsubsection{(c)}

\begin{proof}

\end{proof}

\subsubsection{(d)}

\begin{proof}

\end{proof}

\subsubsection{(e)}

\begin{proof}

\end{proof}

\subsection{Exercise 15}

\subsubsection{(a)}

\begin{proof}

\end{proof}

\subsubsection{(b)}

\begin{proof}

\end{proof}

\subsection{Exercise 16}

\subsubsection{(a)}

\begin{proof}

\end{proof}

\subsubsection{(b)}

\begin{proof}

\end{proof}

\subsubsection{(c)}

\begin{proof}

\end{proof}

\subsubsection{(d)}

\begin{proof}

\end{proof}

\subsubsection{(e)}

\begin{proof}

\end{proof}

\subsection{Exercise 17}

\subsubsection{(a)}

\begin{proof}

\end{proof}

\subsubsection{(b)}

\begin{proof}

\end{proof}

\subsubsection{(c)}

\begin{proof}

\end{proof}

\subsubsection{(d)}

\begin{proof}

\end{proof}

\subsubsection{(e)}

\begin{proof}

\end{proof}

\subsection{Exercise 18}

\subsubsection{(a)}

\begin{proof}

\end{proof}

\subsubsection{(b)}

\begin{proof}

\end{proof}

\subsubsection{(c)}

\begin{proof}

\end{proof}

\subsection{Exercise 19}

\begin{proof}

\end{proof}

\subsection{Exercise 20}

\subsubsection{(a)}

\begin{proof}

\end{proof}

\subsubsection{(b)}

\begin{proof}

\end{proof}

\subsection{Exercise 21}

\subsubsection{(a)}

\begin{proof}

\end{proof}

\subsubsection{(b)}

\begin{proof}

\end{proof}

\subsubsection{(c)}

\begin{proof}

\end{proof}

\subsection{Exercise 22}

\subsubsection{(a)}

\begin{proof}

\end{proof}

\subsubsection{(b)}

\begin{proof}

\end{proof}

\subsubsection{(c)}

\begin{proof}

\end{proof}

\subsection{Exercise 23}

\begin{proof}

\end{proof}

\subsection{Exercise 24}

\begin{proof}

\end{proof}

\subsection{Exercise 25}

\begin{proof}

\end{proof}

\subsection{Exercise 26}

\begin{proof}

\end{proof}

\subsection{Exercise 27}

\begin{proof}

\end{proof}

\subsection{Exercise 28}

\begin{proof}

\end{proof}

\subsection{Exercise 29}

\begin{proof}

\end{proof}

\subsection{Exercise 30}

\subsubsection{(a)}

\begin{proof}

\end{proof}

\subsubsection{(b)}

\begin{proof}

\end{proof}

\subsection{Exercise 31}

\subsubsection{(a)}

\begin{proof}

\end{proof}

\subsubsection{(b)}

\begin{proof}

\end{proof}

\subsubsection{(c)}

\begin{proof}

\end{proof}

\subsubsection{(d)}

\begin{proof}

\end{proof}

\subsubsection{(e)}

\begin{proof}

\end{proof}

\subsection{Exercise 32}

\subsubsection{(a)}

\begin{proof}

\end{proof}

\subsubsection{(b)}

\begin{proof}

\end{proof}

\subsubsection{(c)}

\begin{proof}

\end{proof}

\subsection{Exercise 33}

\subsubsection{(a)}

\begin{proof}

\end{proof}

\subsubsection{(b)}

\begin{proof}

\end{proof}

\subsubsection{(c)}

\begin{proof}

\end{proof}

\subsection{Exercise 34}

\begin{proof}

\end{proof}

\subsection{Exercise 35}

\begin{proof}

\end{proof}

\subsection{Exercise 36}

\begin{proof}

\end{proof}

\subsection{Exercise 37}

\subsubsection{(a)}

\begin{proof}

\end{proof}

\subsubsection{(b)}

\begin{proof}

\end{proof}

\subsubsection{(c)}

\begin{proof}

\end{proof}

\subsubsection{(d)}

\begin{proof}

\end{proof}

\subsection{Exercise 38}

\subsubsection{(a)}

\begin{proof}

\end{proof}

\subsubsection{(b)}

\begin{proof}

\end{proof}

\subsection{Exercise 39}

\subsubsection{(a)}

\begin{proof}

\end{proof}

\subsubsection{(b)}

\begin{proof}

\end{proof}

\subsubsection{(c)}

\begin{proof}

\end{proof}

\subsubsection{(d)}

\begin{proof}

\end{proof}

\subsection{Exercise 40}

\begin{proof}

\end{proof}

\subsection{Exercise 41}

\begin{proof}

\end{proof}

\subsection{Exercise 42}

\begin{proof}

\end{proof}

\subsection{Exercise 43}

\begin{proof}

\end{proof}

\subsection{Exercise 44}

\begin{proof}

\end{proof}

\subsection{Exercise 45}

\begin{proof}

\end{proof}

\subsection{Exercise 46}

\begin{proof}

\end{proof}

\subsection{Exercise 47}

\subsubsection{(a)}

\begin{proof}

\end{proof}

\subsubsection{(b)}

\begin{proof}

\end{proof}

\subsubsection{(c)}

\begin{proof}

\end{proof}

\subsubsection{(d)}

\begin{proof}

\end{proof}

\section{Exercise Set 9.3}

\subsection{Exercise 1}

\subsubsection{(a)}

\begin{proof}

\end{proof}

\subsubsection{(b)}

\begin{proof}

\end{proof}

\subsection{Exercise 2}

\subsubsection{(a)}

\begin{proof}

\end{proof}

\subsubsection{(b)}

\begin{proof}

\end{proof}

\subsection{Exercise 3}

\subsubsection{(a)}

\begin{proof}

\end{proof}

\subsubsection{(b)}

\begin{proof}

\end{proof}

\subsubsection{(c)}

\begin{proof}

\end{proof}

\subsection{Exercise 4}

\begin{proof}

\end{proof}

\subsection{Exercise 5}

\subsubsection{(a)}

\begin{proof}

\end{proof}

\subsubsection{(b)}

\begin{proof}

\end{proof}

\subsection{Exercise 6}

\subsubsection{(a)}

\begin{proof}

\end{proof}

\subsubsection{(b)}

\begin{proof}

\end{proof}

\subsubsection{(c)}

\begin{proof}

\end{proof}

\subsubsection{(d)}

\begin{proof}

\end{proof}

\subsection{Exercise 7}

\subsubsection{(a)}

\begin{proof}

\end{proof}

\subsubsection{(b)}

\begin{proof}

\end{proof}

\subsubsection{(c)}

\begin{proof}

\end{proof}

\subsubsection{(d)}

\begin{proof}

\end{proof}

\subsection{Exercise 8}

\subsubsection{(a)}

\begin{proof}

\end{proof}

\subsubsection{(b)}

\begin{proof}

\end{proof}

\subsubsection{(c)}

\begin{proof}

\end{proof}

\subsubsection{(d)}

\begin{proof}

\end{proof}

\subsection{Exercise 9}

\subsubsection{(a)}

\begin{proof}

\end{proof}

\subsubsection{(b)}

\begin{proof}

\end{proof}

\subsection{Exercise 10}

\begin{proof}

\end{proof}

\subsection{Exercise 11}
\subsubsection{(a)}

\begin{proof}

\end{proof}

\subsubsection{(b)}

\begin{proof}

\end{proof}

\subsubsection{(c)}

\begin{proof}

\end{proof}

\subsection{Exercise 12}

\subsubsection{(a)}

\begin{proof}

\end{proof}

\subsubsection{(b)}

\begin{proof}

\end{proof}

\subsection{Exercise 13}

\subsubsection{(a)}

\begin{proof}

\end{proof}

\subsubsection{(b)}

\begin{proof}

\end{proof}

\subsection{Exercise 14}

\begin{proof}

\end{proof}

\subsection{Exercise 15}

\begin{proof}

\end{proof}

\subsection{Exercise 16}

\subsubsection{(a)}

\begin{proof}

\end{proof}

\subsubsection{(b)}

\begin{proof}

\end{proof}

\subsubsection{(c)}

\begin{proof}

\end{proof}

\subsection{Exercise 17}
\subsubsection{(a)}

\begin{proof}

\end{proof}

\subsubsection{(b)}

\begin{proof}

\end{proof}

\subsubsection{(c)}

\begin{proof}

\end{proof}

\subsection{Exercise 18}

\subsubsection{(a)}

\begin{proof}

\end{proof}

\subsubsection{(b)}

\begin{proof}

\end{proof}

\subsection{Exercise 19}

\begin{proof}

\end{proof}

\subsection{Exercise 20}

\subsubsection{(a)}

\begin{proof}

\end{proof}

\subsubsection{(b)}

\begin{proof}

\end{proof}

\subsubsection{(c)}

\begin{proof}

\end{proof}

\subsection{Exercise 21}

\begin{proof}

\end{proof}

\subsection{Exercise 22}

\subsubsection{(a)}

\begin{proof}

\end{proof}

\subsubsection{(b)}

\begin{proof}

\end{proof}

\subsection{Exercise 23}

\subsubsection{(a)}

\begin{proof}

\end{proof}

\subsubsection{(b)}

\begin{proof}

\end{proof}

\subsubsection{(c)}

\begin{proof}

\end{proof}

\subsection{Exercise 24}

\subsubsection{(a)}

\begin{proof}

\end{proof}

\subsubsection{(b)}

\begin{proof}

\end{proof}

\subsubsection{(c)}

\begin{proof}

\end{proof}

\subsection{Exercise 25}

\subsubsection{(a)}

\begin{proof}

\end{proof}

\subsubsection{(b)}

\begin{proof}

\end{proof}

\subsubsection{(c)}

\begin{proof}

\end{proof}

\subsubsection{(d)}

\begin{proof}

\end{proof}

\subsection{Exercise 26}

\subsubsection{(a)}

\begin{proof}

\end{proof}

\subsubsection{(b)}

\begin{proof}

\end{proof}

\subsubsection{(c)}

\begin{proof}

\end{proof}

\subsubsection{(d)}

\begin{proof}

\end{proof}

\subsubsection{(e)}

\begin{proof}

\end{proof}

\subsection{Exercise 27}

\subsubsection{(a)}

\begin{proof}

\end{proof}

\subsubsection{(b)}

\begin{proof}

\end{proof}

\subsection{Exercise 28}

\subsubsection{(a)}

\begin{proof}

\end{proof}

\subsubsection{(b)}

\begin{proof}

\end{proof}

\subsection{Exercise 29}

\subsubsection{(a)}

\begin{proof}

\end{proof}

\subsubsection{(b)}

\begin{proof}

\end{proof}

\subsubsection{(c)}

\begin{proof}

\end{proof}

\subsubsection{(d)}

\begin{proof}

\end{proof}

\subsubsection{(e)}

\begin{proof}

\end{proof}

\subsubsection{(f)}

\begin{proof}

\end{proof}

\subsubsection{(g)}

\begin{proof}

\end{proof}

\subsubsection{(h)}

\begin{proof}

\end{proof}

\subsubsection{(i)}

\begin{proof}

\end{proof}

\subsubsection{(j)}

\begin{proof}

\end{proof}

\subsection{Exercise 30}

\begin{proof}

\end{proof}

\subsection{Exercise 31}

\subsubsection{(a)}

\begin{proof}

\end{proof}

\subsubsection{(b)}

\begin{proof}

\end{proof}

\subsubsection{(c)}

\begin{proof}

\end{proof}

\subsubsection{(d)}

\begin{proof}

\end{proof}

\subsubsection{(e)}

\begin{proof}

\end{proof}

\subsection{Exercise 32}

\begin{proof}

\end{proof}

\subsection{Exercise 33}

\subsubsection{(a)}

\begin{proof}

\end{proof}

\subsubsection{(b)}

\begin{proof}

\end{proof}

\subsubsection{(c)}

\begin{proof}

\end{proof}

\subsubsection{(d)}

\begin{proof}

\end{proof}

\subsubsection{(e)}

\begin{proof}

\end{proof}

\subsubsection{(f)}

\begin{proof}

\end{proof}

\subsection{Exercise 34}

\subsubsection{(a)}

\begin{proof}

\end{proof}

\subsubsection{(b)}

\begin{proof}

\end{proof}

\subsubsection{(c)}

\begin{proof}

\end{proof}

\subsubsection{(d)}

\begin{proof}

\end{proof}

\subsection{Exercise 35}

\begin{proof}

\end{proof}

\subsection{Exercise 36}

\begin{proof}

\end{proof}

\subsection{Exercise 37}

\begin{proof}

\end{proof}

\subsection{Exercise 38}

\begin{proof}

\end{proof}

\subsection{Exercise 39}

\begin{proof}

\end{proof}

\subsection{Exercise 40}

\begin{proof}

\end{proof}

\subsection{Exercise 41}

\begin{proof}

\end{proof}

\subsection{Exercise 42}

\subsubsection{(a)}

\begin{proof}

\end{proof}

\subsubsection{(b)}

\begin{proof}

\end{proof}

\subsubsection{(c)}

\begin{proof}

\end{proof}

\subsection{Exercise 43}

\subsubsection{(a)}

\begin{proof}

\end{proof}

\subsubsection{(b)}

\begin{proof}

\end{proof}

\subsubsection{(c)}

\begin{proof}

\end{proof}

\subsection{Exercise 44}

\begin{proof}

\end{proof}

\subsection{Exercise 45}

\begin{proof}

\end{proof}

\subsection{Exercise 46}

\begin{proof}

\end{proof}

\subsection{Exercise 47}

\begin{proof}

\end{proof}

\subsection{Exercise 48}

\begin{proof}

\end{proof}

\subsection{Exercise 49}

\subsubsection{(a)}

\begin{proof}

\end{proof}

\subsubsection{(b)}

\begin{proof}

\end{proof}

\section{Exercise Set 9.4}

\subsection{Exercise 1}

\subsubsection{(a)}

\begin{proof}

\end{proof}

\subsubsection{(b)}

\begin{proof}

\end{proof}

\subsection{Exercise 2}

\subsubsection{(a)}

\begin{proof}

\end{proof}

\subsubsection{(b)}

\begin{proof}

\end{proof}

\subsection{Exercise 3}

\begin{proof}

\end{proof}

\subsection{Exercise 4}

\begin{proof}

\end{proof}

\subsection{Exercise 5}

\subsubsection{(a)}

\begin{proof}

\end{proof}

\subsubsection{(b)}

\begin{proof}

\end{proof}

\subsection{Exercise 6}

\subsubsection{(a)}

\begin{proof}

\end{proof}

\subsubsection{(b)}

\begin{proof}

\end{proof}

\subsection{Exercise 7}

\begin{proof}

\end{proof}

\subsection{Exercise 8}

\begin{proof}

\end{proof}

\subsection{Exercise 9}

\subsubsection{(a)}

\begin{proof}

\end{proof}

\subsubsection{(b)}

\begin{proof}

\end{proof}

\subsection{Exercise 10}

\begin{proof}

\end{proof}

\subsection{Exercise 11}

\begin{proof}

\end{proof}

\subsection{Exercise 12}

\begin{proof}

\end{proof}

\subsection{Exercise 13}

\begin{proof}

\end{proof}

\subsection{Exercise 14}

\begin{proof}

\end{proof}

\subsection{Exercise 15}

\begin{proof}

\end{proof}

\subsection{Exercise 16}

\begin{proof}

\end{proof}

\subsection{Exercise 17}

\begin{proof}

\end{proof}

\subsection{Exercise 18}

\begin{proof}

\end{proof}

\subsection{Exercise 19}

\begin{proof}

\end{proof}

\subsection{Exercise 20}

\subsubsection{(a)}

\begin{proof}

\end{proof}

\subsubsection{(b)}

\begin{proof}

\end{proof}

\subsection{Exercise 21}

\begin{proof}

\end{proof}

\subsection{Exercise 22}

\begin{proof}

\end{proof}

\subsection{Exercise 23}

\begin{proof}

\end{proof}

\subsection{Exercise 24}

\begin{proof}

\end{proof}

\subsection{Exercise 25}

\begin{proof}

\end{proof}

\subsection{Exercise 26}

\begin{proof}

\end{proof}

\subsection{Exercise 27}

\begin{proof}

\end{proof}

\subsection{Exercise 28}

\begin{proof}

\end{proof}

\subsection{Exercise 29}

\begin{proof}

\end{proof}

\subsection{Exercise 30}

\begin{proof}

\end{proof}

\subsection{Exercise 31}

\begin{proof}

\end{proof}

\subsection{Exercise 32}

\begin{proof}

\end{proof}

\subsection{Exercise 33}

\begin{proof}

\end{proof}

\subsection{Exercise 34}

\begin{proof}

\end{proof}

\subsection{Exercise 35}

\begin{proof}

\end{proof}

\subsection{Exercise 36}

\begin{proof}

\end{proof}

\subsection{Exercise 37}
\subsubsection{(a)}

\begin{proof}

\end{proof}

\subsubsection{(b)}

\begin{proof}

\end{proof}

\subsection{Exercise 38}

\begin{proof}

\end{proof}

\subsection{Exercise 39}

\begin{proof}

\end{proof}

\subsection{Exercise 40}

\begin{proof}

\end{proof}

\section{Exercise Set 9.5}

\subsection{Exercise 1}

\subsubsection{(a)}

\begin{proof}

\end{proof}

\subsubsection{(b)}

\begin{proof}

\end{proof}

\subsection{Exercise 2}

\subsubsection{(a)}

\begin{proof}

\end{proof}

\subsubsection{(b)}

\begin{proof}

\end{proof}

\subsection{Exercise 3}

\begin{proof}

\end{proof}

\subsection{Exercise 4}

\begin{proof}

\end{proof}

\subsection{Exercise 5}

\subsubsection{(a)}

\begin{proof}

\end{proof}

\subsubsection{(b)}

\begin{proof}

\end{proof}

\subsubsection{(c)}

\begin{proof}

\end{proof}

\subsubsection{(d)}

\begin{proof}

\end{proof}

\subsubsection{(e)}

\begin{proof}

\end{proof}

\subsection{Exercise 6}

\subsubsection{(a)}

\begin{proof}

\end{proof}

\subsubsection{(b)}

\begin{proof}

\end{proof}

\subsubsection{(c)}

\begin{proof}

\end{proof}

\subsubsection{(d)}

\begin{proof}

\end{proof}

\subsubsection{(e)}

\begin{proof}

\end{proof}

\subsection{Exercise 7}

\subsubsection{(a)}

\begin{proof}

\end{proof}

\subsubsection{(b)}

\begin{proof}

\end{proof}

\subsubsection{(c)}

\begin{proof}

\end{proof}

\subsubsection{(d)}

\begin{proof}

\end{proof}

\subsection{Exercise 8}

\subsubsection{(a)}

\begin{proof}

\end{proof}

\subsubsection{(b)}

\begin{proof}

\end{proof}

\subsubsection{(c)}

\begin{proof}

\end{proof}

\subsubsection{(d)}

\begin{proof}

\end{proof}

\subsection{Exercise 9}

\subsubsection{(a)}

\begin{proof}

\end{proof}

\subsubsection{(b)}

\begin{proof}

\end{proof}

\subsection{Exercise 10}

\begin{proof}

\end{proof}

\subsection{Exercise 11}

\subsubsection{(a)}

\begin{proof}

\end{proof}

\subsubsection{(b)}

\begin{proof}

\end{proof}

\subsubsection{(c)}

\begin{proof}

\end{proof}

\subsubsection{(d)}

\begin{proof}

\end{proof}

\subsubsection{(e)}

\begin{proof}

\end{proof}

\subsubsection{(f)}

\begin{proof}

\end{proof}

\subsubsection{(g)}

\begin{proof}

\end{proof}

\subsubsection{(h)}

\begin{proof}

\end{proof}

\subsubsection{(i)}

\begin{proof}

\end{proof}

\subsection{Exercise 12}

\begin{proof}

\end{proof}

\subsection{Exercise 13}

\subsubsection{(a)}

\begin{proof}

\end{proof}

\subsubsection{(b)}

\begin{proof}

\end{proof}

\subsubsection{(c)}

\begin{proof}

\end{proof}

\subsubsection{(d)}

\begin{proof}

\end{proof}

\subsubsection{(e)}

\begin{proof}

\end{proof}

\subsection{Exercise 14}

\subsubsection{(a)}

\begin{proof}

\end{proof}

\subsubsection{(b)}

\begin{proof}

\end{proof}

\subsubsection{(c)}

\begin{proof}

\end{proof}

\subsubsection{(d)}

\begin{proof}

\end{proof}

\subsection{Exercise 15}

\subsubsection{(a)}

\begin{proof}

\end{proof}

\subsubsection{(b)}

\begin{proof}

\end{proof}

\subsubsection{(c)}

\begin{proof}

\end{proof}

\subsubsection{(d)}

\begin{proof}

\end{proof}

\subsection{Exercise 16}

\subsubsection{(a)}

\begin{proof}

\end{proof}

\subsubsection{(b)}

\begin{proof}

\end{proof}

\subsubsection{(c)}

\begin{proof}

\end{proof}

\subsection{Exercise 17}

\subsubsection{(a)}

\begin{proof}

\end{proof}

\subsubsection{(b)}

\begin{proof}

\end{proof}

\subsubsection{(c)}

\begin{proof}

\end{proof}

\subsubsection{(d)}

\begin{proof}

\end{proof}

\subsection{Exercise 18}

\begin{proof}

\end{proof}

\subsection{Exercise 19}

\subsubsection{(a)}

\begin{proof}

\end{proof}

\subsubsection{(b)}

\begin{proof}

\end{proof}

\subsubsection{(c)}

\begin{proof}

\end{proof}

\subsection{Exercise 20}

\subsubsection{(a)}

\begin{proof}

\end{proof}

\subsubsection{(b)}

\begin{proof}

\end{proof}

\subsubsection{(c)}

\begin{proof}

\end{proof}

\subsection{Exercise 21}

\begin{proof}

\end{proof}

\subsection{Exercise 22}

\begin{proof}

\end{proof}

\subsection{Exercise 23}

\begin{proof}

\end{proof}

\subsection{Exercise 24}

\subsubsection{(a)}

\begin{proof}

\end{proof}

\subsubsection{(b)}

\begin{proof}

\end{proof}

\subsubsection{(c)}

\begin{proof}

\end{proof}

\subsubsection{(d)}

\begin{proof}

\end{proof}

\subsection{Exercise 25}

\subsubsection{(a)}

\begin{proof}

\end{proof}

\subsubsection{(b)}

\begin{proof}

\end{proof}

\subsubsection{(c)}

\begin{proof}

\end{proof}

\subsubsection{(d)}

\begin{proof}

\end{proof}

\subsubsection{(e)}

\begin{proof}

\end{proof}

\subsection{Exercise 26}

\subsubsection{(a)}

\begin{proof}

\end{proof}

\subsubsection{(b)}

\begin{proof}

\end{proof}

\subsubsection{(c)}

\begin{proof}

\end{proof}

\subsubsection{(d)}

\begin{proof}

\end{proof}

\subsubsection{(e)}

\begin{proof}

\end{proof}

\subsubsection{(f)}

\begin{proof}

\end{proof}

\subsection{Exercise 27}

\subsubsection{(a)}

\begin{proof}

\end{proof}

\subsubsection{(b)}

\begin{proof}

\end{proof}

\subsubsection{(c)}

\begin{proof}

\end{proof}

\subsubsection{(d)}

\begin{proof}

\end{proof}

\subsection{Exercise 28}

\begin{proof}

\end{proof}

\subsection{Exercise 29}

\begin{proof}

\end{proof}

\subsection{Exercise 30}

\begin{proof}

\end{proof}

\section{Exercise Set 9.6}

\subsection{Exercise 1}

\subsubsection{(a)}

\begin{proof}

\end{proof}

\subsubsection{(b)}

\begin{proof}

\end{proof}

\subsection{Exercise 2}

\subsubsection{(a)}

\begin{proof}

\end{proof}

\subsubsection{(b)}

\begin{proof}

\end{proof}

\subsection{Exercise 3}

\subsubsection{(a)}

\begin{proof}

\end{proof}

\subsubsection{(b)}

\begin{proof}

\end{proof}

\subsubsection{(c)}

\begin{proof}

\end{proof}

\subsection{Exercise 4}

\subsubsection{(a)}

\begin{proof}

\end{proof}

\subsubsection{(b)}

\begin{proof}

\end{proof}

\subsubsection{(c)}

\begin{proof}

\end{proof}

\subsection{Exercise 5}

\begin{proof}

\end{proof}

\subsection{Exercise 6}

\begin{proof}

\end{proof}

\subsection{Exercise 7}

\begin{proof}

\end{proof}

\subsection{Exercise 8}

\begin{proof}

\end{proof}

\subsection{Exercise 9}

\begin{proof}

\end{proof}

\subsection{Exercise 10}

\begin{proof}

\end{proof}

\subsection{Exercise 11}

\begin{proof}

\end{proof}

\subsection{Exercise 12}

\begin{proof}

\end{proof}

\subsection{Exercise 13}

\begin{proof}

\end{proof}

\subsection{Exercise 14}

\begin{proof}

\end{proof}

\subsection{Exercise 15}

\begin{proof}

\end{proof}

\subsection{Exercise 16}

\subsubsection{(a)}

\begin{proof}

\end{proof}

\subsubsection{(b)}

\begin{proof}

\end{proof}

\subsubsection{(c)}

\begin{proof}

\end{proof}

\subsection{Exercise 17}

\subsubsection{(a)}

\begin{proof}

\end{proof}

\subsubsection{(b)}

\begin{proof}

\end{proof}

\subsubsection{(c)}

\begin{proof}

\end{proof}

\subsubsection{(d)}

\begin{proof}

\end{proof}

\subsection{Exercise 18}

\subsubsection{(a)}

\begin{proof}

\end{proof}

\subsubsection{(b)}

\begin{proof}

\end{proof}

\subsubsection{(c)}

\begin{proof}

\end{proof}

\subsubsection{(d)}

\begin{proof}

\end{proof}

\subsection{Exercise 19}

\subsubsection{(a)}

\begin{proof}

\end{proof}

\subsubsection{(b)}

\begin{proof}

\end{proof}

\subsection{Exercise 20}

\subsubsection{(a)}

\begin{proof}

\end{proof}

\subsubsection{(b)}

\begin{proof}

\end{proof}

\subsection{Exercise 21}

\begin{proof}

\end{proof}

\section{Exercise Set 9.7}

\subsection{Exercise 1}

\begin{proof}

\end{proof}

\subsection{Exercise 2}

\begin{proof}

\end{proof}

\subsection{Exercise 3}

\begin{proof}

\end{proof}

\subsection{Exercise 4}

\begin{proof}

\end{proof}

\subsection{Exercise 5}

\begin{proof}

\end{proof}

\subsection{Exercise 6}

\begin{proof}

\end{proof}

\subsection{Exercise 7}

\begin{proof}

\end{proof}

\subsection{Exercise 8}

\begin{proof}

\end{proof}

\subsection{Exercise 9}

\begin{proof}

\end{proof}

\subsection{Exercise 10}

\subsubsection{(a)}

\begin{proof}

\end{proof}

\subsubsection{(b)}

\begin{proof}

\end{proof}

\subsubsection{(c)}

\begin{proof}

\end{proof}

\subsection{Exercise 11}

\begin{proof}

\end{proof}

\subsection{Exercise 12}

\begin{proof}

\end{proof}

\subsection{Exercise 13}

\begin{proof}

\end{proof}

\subsection{Exercise 14}

\begin{proof}

\end{proof}

\subsection{Exercise 15}

\begin{proof}

\end{proof}

\subsection{Exercise 16}

\begin{proof}

\end{proof}

\subsection{Exercise 17}

\begin{proof}

\end{proof}

\subsection{Exercise 18}

\begin{proof}

\end{proof}

\subsection{Exercise 19}

\begin{proof}

\end{proof}

\subsection{Exercise 20}

\begin{proof}

\end{proof}

\subsection{Exercise 21}

\begin{proof}

\end{proof}

\subsection{Exercise 22}

\begin{proof}

\end{proof}

\subsection{Exercise 23}

\begin{proof}

\end{proof}

\subsection{Exercise 24}

\begin{proof}

\end{proof}

\subsection{Exercise 25}

\begin{proof}

\end{proof}

\subsection{Exercise 26}

\begin{proof}

\end{proof}

\subsection{Exercise 27}

\begin{proof}

\end{proof}

\subsection{Exercise 28}

\begin{proof}

\end{proof}

\subsection{Exercise 29}

\begin{proof}

\end{proof}

\subsection{Exercise 30}

\begin{proof}

\end{proof}

\subsection{Exercise 31}

\begin{proof}

\end{proof}

\subsection{Exercise 32}

\begin{proof}

\end{proof}

\subsection{Exercise 33}

\begin{proof}

\end{proof}

\subsection{Exercise 34}

\begin{proof}

\end{proof}

\subsection{Exercise 35}

\begin{proof}

\end{proof}

\subsection{Exercise 36}

\begin{proof}

\end{proof}

\subsection{Exercise 37}

\begin{proof}

\end{proof}

\subsection{Exercise 38}

\begin{proof}

\end{proof}

\subsection{Exercise 39}

\begin{proof}

\end{proof}

\subsection{Exercise 40}

\begin{proof}

\end{proof}

\subsection{Exercise 41}

\begin{proof}

\end{proof}

\subsection{Exercise 42}

\begin{proof}

\end{proof}

\subsection{Exercise 43}

\begin{proof}

\end{proof}

\subsection{Exercise 44}

\begin{proof}

\end{proof}

\subsection{Exercise 45}

\begin{proof}

\end{proof}

\subsection{Exercise 46}

\begin{proof}

\end{proof}

\subsection{Exercise 47}

\begin{proof}

\end{proof}

\subsection{Exercise 48}

\begin{proof}

\end{proof}

\subsection{Exercise 49}

\begin{proof}

\end{proof}

\subsection{Exercise 50}

\begin{proof}

\end{proof}

\subsection{Exercise 51}

\begin{proof}

\end{proof}

\subsection{Exercise 52}

\begin{proof}

\end{proof}

\subsection{Exercise 53}

\begin{proof}

\end{proof}

\subsection{Exercise 54}

\begin{proof}

\end{proof}

\subsection{Exercise 55}

\subsubsection{(a)}

\begin{proof}

\end{proof}

\subsubsection{(b)}

\begin{proof}

\end{proof}

\subsubsection{(c)}

\begin{proof}

\end{proof}

\subsubsection{(d)}

\begin{proof}

\end{proof}

\section{Exercise Set 9.8}

\subsection{Exercise 1}

\begin{proof}

\end{proof}

\subsection{Exercise 2}

\begin{proof}

\end{proof}

\subsection{Exercise 3}

\subsubsection{(a)}

\begin{proof}

\end{proof}

\subsubsection{(b)}

\begin{proof}

\end{proof}

\subsection{Exercise 4}

\begin{proof}

\end{proof}

\subsection{Exercise 5}

\begin{proof}

\end{proof}

\subsection{Exercise 6}

\begin{proof}

\end{proof}

\subsection{Exercise 7}

\subsubsection{(a)}

\begin{proof}

\end{proof}

\subsubsection{(b)}

\begin{proof}

\end{proof}

\subsubsection{(c)}

\begin{proof}

\end{proof}

\subsubsection{(d)}

\begin{proof}

\end{proof}

\subsubsection{(e)}

\begin{proof}

\end{proof}

\subsubsection{(f)}

\begin{proof}

\end{proof}

\subsection{Exercise 8}

\begin{proof}

\end{proof}

\subsection{Exercise 9}

\subsubsection{(a)}

\begin{proof}

\end{proof}

\subsubsection{(b)}

\begin{proof}

\end{proof}

\subsubsection{(c)}

\begin{proof}

\end{proof}

\subsubsection{(d)}

\begin{proof}

\end{proof}

\subsubsection{(e)}

\begin{proof}

\end{proof}

\subsubsection{(f)}

\begin{proof}

\end{proof}

\subsection{Exercise 10}

\begin{proof}

\end{proof}

\subsection{Exercise 11}

\begin{proof}

\end{proof}

\subsection{Exercise 12}

\begin{proof}

\end{proof}

\subsection{Exercise 13}

\begin{proof}

\end{proof}

\subsection{Exercise 14}

\begin{proof}

\end{proof}

\subsection{Exercise 15}

\begin{proof}

\end{proof}

\subsection{Exercise 16}

\begin{proof}

\end{proof}

\subsection{Exercise 17}

\begin{proof}

\end{proof}

\subsection{Exercise 18}

\begin{proof}

\end{proof}

\subsection{Exercise 19}

\begin{proof}

\end{proof}

\subsection{Exercise 20}

\begin{proof}

\end{proof}

\subsection{Exercise 21}

\begin{proof}

\end{proof}

\subsection{Exercise 22}

\begin{proof}

\end{proof}

\subsection{Exercise 23}

\begin{proof}

\end{proof}

\section{Exercise Set 9.9}

\subsection{Exercise 1}

\begin{proof}

\end{proof}

\subsection{Exercise 2}

\begin{proof}

\end{proof}

\subsection{Exercise 3}

\begin{proof}

\end{proof}

\subsection{Exercise 4}

\subsubsection{(a)}

\begin{proof}

\end{proof}

\subsubsection{(b)}

\begin{proof}

\end{proof}

\subsection{Exercise 5}

\begin{proof}

\end{proof}

\subsection{Exercise 6}

\subsubsection{(a)}

\begin{proof}

\end{proof}

\subsubsection{(b)}

\begin{proof}

\end{proof}

\subsubsection{(c)}

\begin{proof}

\end{proof}

\subsection{Exercise 7}

\begin{proof}

\end{proof}

\subsection{Exercise 8}

\subsubsection{(a)}

\begin{proof}

\end{proof}

\subsubsection{(b)}

\begin{proof}

\end{proof}

\subsubsection{(c)}

\begin{proof}

\end{proof}

\subsection{Exercise 9}

\begin{proof}

\end{proof}

\subsection{Exercise 10}

\begin{proof}

\end{proof}

\subsection{Exercise 11}

\subsubsection{(a)}

\begin{proof}

\end{proof}

\subsubsection{(b)}

\begin{proof}

\end{proof}

\subsection{Exercise 12}

\begin{proof}

\end{proof}

\subsection{Exercise 13}

\subsubsection{(a)}

\begin{proof}

\end{proof}

\subsubsection{(b)}

\begin{proof}

\end{proof}

\subsection{Exercise 14}

\subsubsection{(a)}

\begin{proof}

\end{proof}

\subsubsection{(b)}

\begin{proof}

\end{proof}

\subsection{Exercise 15}

\subsubsection{(a)}

\begin{proof}

\end{proof}

\subsubsection{(b)}

\begin{proof}

\end{proof}

\subsubsection{(c)}

\begin{proof}

\end{proof}

\subsubsection{(d)}

\begin{proof}

\end{proof}

\subsection{Exercise 16}

\subsubsection{(a)}

\begin{proof}

\end{proof}

\subsubsection{(b)}

\begin{proof}

\end{proof}

\subsection{Exercise 17}

\begin{proof}

\end{proof}

\subsection{Exercise 18}

\begin{proof}

\end{proof}

\subsection{Exercise 19}

\begin{proof}

\end{proof}

\subsection{Exercise 20}

\begin{proof}

\end{proof}

\subsection{Exercise 21}

\begin{proof}

\end{proof}

\subsection{Exercise 22}

\begin{proof}

\end{proof}

\subsection{Exercise 23}

\subsubsection{(a)}

\begin{proof}

\end{proof}

\subsubsection{(b)}

\begin{proof}

\end{proof}

\subsubsection{(c)}

\begin{proof}

\end{proof}

\subsection{Exercise 24}

\subsubsection{(a)}

\begin{proof}

\end{proof}

\subsubsection{(b)}

\begin{proof}

\end{proof}

\subsection{Exercise 25}

\subsubsection{(a)}

\begin{proof}

\end{proof}

\subsubsection{(b)}

\begin{proof}

\end{proof}

\subsubsection{(c)}

\begin{proof}

\end{proof}

\subsubsection{(d)}

\begin{proof}

\end{proof}

\subsection{Exercise 26}

\begin{proof}

\end{proof}

\subsection{Exercise 27}

\begin{proof}

\end{proof}

\subsection{Exercise 28}

\subsubsection{(a)}

\begin{proof}

\end{proof}

\subsubsection{(b)}

\begin{proof}

\end{proof}

\subsubsection{(c)}

\begin{proof}

\end{proof}

\subsubsection{(d)}

\begin{proof}

\end{proof}

\subsection{Exercise 29}

\subsubsection{(a)}

\begin{proof}

\end{proof}

\subsubsection{(b)}

\begin{proof}

\end{proof}

\subsubsection{(c)}

\begin{proof}

\end{proof}

\subsubsection{(d)}

\begin{proof}

\end{proof}

\subsection{Exercise 30}

\subsubsection{(a)}

\begin{proof}

\end{proof}

\subsubsection{(b)}

\begin{proof}

\end{proof}

\subsubsection{(c)}

\begin{proof}

\end{proof}

\subsubsection{(d)}

\begin{proof}

\end{proof}

\subsection{Exercise 31}

\subsubsection{(a)}

\begin{proof}

\end{proof}

\subsubsection{(b)}

\begin{proof}

\end{proof}

\subsubsection{(c)}

\begin{proof}

\end{proof}

\subsection{Exercise 32}

\subsubsection{(a)}

\begin{proof}

\end{proof}

\subsubsection{(b)}

\begin{proof}

\end{proof}

\subsubsection{(c)}

\begin{proof}

\end{proof}

\subsubsection{(d)}

\begin{proof}

\end{proof}

\subsubsection{(e)}

\begin{proof}

\end{proof}

\subsection{Exercise 33}

\begin{proof}

\end{proof}

\subsection{Exercise 34}

\begin{proof}

\end{proof}

\end{document}
